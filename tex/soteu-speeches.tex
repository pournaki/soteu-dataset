\documentclass[a4paper,11pt]{article}
\setlength{\parindent}{0cm}
\usepackage[margin=1in]{geometry}
\usepackage{eurosym}
\usepackage{titlesec}
\title{State of the European Union Speeches}
\date{}
\titleformat*{\section}{\normalsize\bfseries}
\usepackage{hyperref}
\usepackage{url}
\hypersetup{
    colorlinks,
    citecolor=black,
    filecolor=black,
    linkcolor=black,
    urlcolor=black
}
\begin{document}
\maketitle
\tableofcontents
\newpage

\section{Speech 1 - Barroso - 2010-09-07}
\url{https://ec.europa.eu/commission/presscorner/detail/en/SPEECH_10_411}\\[3mm]
President,

Honourable Members,

It is a great privilege to deliver the first State of the Union address before this House.

From now on the State of the Union address will be the occasion when we will chart our work for the next 12 months. Many of the decisions we will take this year will have long-term implications. They will define the kind of Europe we want. They will define a Europe of opportunity where those that aspire are elevated and those in need are not neglected. A Europe that is open to the world and open to its people. A Europe that delivers economic, social and territorial cohesion.

Over the last year, the economic and financial crisis has put our Union before one of its greatest challenge ever. Our interdependence was highlighted and our solidarity was tested like never before.

As I look back at how we have reacted, I believe that we have withstood the test. We have provided many of the answers needed – on financial assistance to Member States facing exceptional circumstances, on economic governance, on financial regulation, on growth and jobs. And we have been able to build a base camp from which to modernise our economies. Europe has shown it will stand up and be counted. Those who predicted the demise of the European Union were proved wrong. The European institutions and the Member States have demonstrated leadership. My message to each and every European is that you can trust the European Union to do what it takes to secure your future.

The economic outlook in the European Union today is better than one year ago, not least as a result of our determined action. The recovery is gathering pace, albeit unevenly within the Union. Growth this year will be higher than initially forecast. The unemployment rate, whilst still much too high, has stopped increasing. Clearly, uncertainties and risks remain, not least outside the European Union.

We should be under no illusions. Our work is far from finished. There is no room for complacency. Budgetary expansion played its role to counter the decline in economic activity. But it is now time to exit. Without structural reforms, we will not create sustainable growth. We must use the next 12 months to accelerate our reform agenda. Now is the time to modernise our social market economy so that it can compete globally and respond to the challenge of demography. Now is the time to make the right investments for our future.

This is Europe's moment of truth. Europe must show it is more than 27 different national solutions. We either swim together, or sink separately. We will only succeed if, whether acting nationally, regionally or locally we think European.

Today, I will set out what I see as the priorities for our work together over the coming year. I cannot now cover every issue of European policy or initiative we will take. I am sending you through President Buzek a more complete programme document.

Essentially, I see five major challenges for the Union over the next year:

dealing with the economic crisis and governance;

restoring growth for jobs by accelerating the Europe 2020 reform agenda;

building an area of freedom, justice and security;

launching negotiations for a modern EU budget, and

pulling our weight on the global stage.

Let me start with the economic crisis and governance. Earlier this year, we acted decisively when euro area members and the euro itself needed our help.

We have learned hard lessons. Now we are making important progress on economic governance. The Commission has put its ideas on the table in May and in June. They have been well received, in this Parliament, and in the Task Force chaired by President of the European Council. They are the basis around which a consensus is being developed. We will present the most urgent legislative proposals on 29 September, so as not to lose the momentum.

Unsustainable budgets make us vulnerable. Debt and deficit lead to boom and bust. And they unravel the social safety net. Money that's spent on servicing debt is money that cannot be spent on the social good. Nor to prepare ourselves for the costs of an ageing population. A debt generation makes an unsustainable nation. Our proposals will strengthen the Stability and Growth Pact through increased surveillance and enforcement.

And we need to tackle severe macro-economic imbalances, especially in the Euro area. That is why we have made proposals early on to detect asset bubbles, lack of competitiveness and other sources of imbalances.

I now see a willingness of governments to accept stronger monitoring, backed up by incentives for compliance and earlier sanctions. The Commission will strengthen its role as independent referee and enforcer of the new rules.

We will match monetary union with true economic union.

If implemented as we propose, these reforms will also guarantee the long-term stability of the euro. It is key to our economic success.

For the economy to grow, we also need a strong and sound financial sector. A sector that serves the real economy. A sector that prides itself on proper regulation and proper supervision.

We took action to increase bank transparency. Today we are better than one year ago. With the publication of the stress test results, banks should now be able to lend to each other, so that credit can flow to Europe's citizens and companies.

We have proposed to protect people's savings up to \euro100,000. We will propose to ban abusive naked short selling. We will tackle credit default swaps. The days of betting on someone else's house burning down are over. We continue to insist that banks, not taxpayers, must pay up front to cover the costs of their own risks of failure. We are legislating to outlaw bonuses for quick-wins today that become big losses tomorrow. As part of this approach, I am also defending taxes on financial activities and we will come with proposals this autumn.

The political deal on the financial supervision package just concluded is very good news. The Commission proposals based on the de Larosière report will give us an effective European supervision system. I want to thank the Parliament for the constructive role it has played and I hope it will give its final agreement this month.

We will also go further on regulation. Initiatives on derivatives, further measures on credit rating agencies and a framework for bank resolution and crisis management will soon be before you. Our goal is to have a reformed financial sector in place by the end of 2011.

Sound government finances and responsible financial markets give us the confidence and economic strength for sustainable growth. We need to move beyond the debate between fiscal consolidation and growth. We can have both.

Honourable Members,

Sound public finances are a means to an end: growth for jobs. Our goal is growth, sustainable growth, inclusive growth. This is our overarching priority. This is where we need to invest.

Europe 2020 starts now. We must frontload and accelerate the most growth-promoting reforms of our agenda. This could raise growth levels by over a third by 2020.

This means concentrating on three priorities: getting more people in jobs, boosting our companies' competitiveness and deepening the single market.

Let me start with people and jobs.

Over 6.3 million people have lost their jobs since 2008. Each one of them should have the chance to get back into employment. Europe's employment rates are at 69\% on average for those aged between 20 and 64. We have agreed these should rise to 75\% by 2020, bringing in particular more women and older workers into the work force.

Most of the competences for employment policy remain with Member States. But we won't stand on the sidelines. I want a European Union that helps its people to seize new opportunities; and I want a Union that is social and inclusive. This is the Europe we will build if Member States, the European institutions and the social partners move ahead on our common reform agenda.

It should be centred on skills and jobs and investment in life-long learning.

And it should focus on unlocking the growth potential of the single market, to build a stronger single market for jobs.

The opportunities exist. We have very high levels of unemployment but Europe has now 4 million job vacancies. The Commission will propose later this year a "European Vacancy Monitor". It will show people where the jobs are in Europe and which skills are needed. We will also come forward with plans for a European skills passport.

We must also tackle problems of poverty and exclusion. We must make sure that the most vulnerable are not left behind. This is the focus of our "Platform Against Poverty". It will bring together European action for vulnerable groups such as children and old people.

As more and more people travel, study or work abroad, we will also strengthen citizens' rights as they move across borders. The Commission will address persisting obstacles as early as this autumn.

Honourable Members,

Growth must be based on our companies' competitiveness.

We should continue to make life easier for our Small and Medium-Sized Enterprises. They provide two out of every three private sector jobs. Among their main concerns are innovation and red tape. We are working on both.

Just before the summer, the Commission has announced the biggest ever package from the Seventh Research Framework Programme, worth \euro6.4 billion. This money will go to SMEs as well as to scientists.

Investing in innovation also means promoting world class universities in Europe. I want to see them attracting the brightest and the best, from Europe and the rest of the world. We will take an initiative on the modernisation of European universities. I want to see a Europe that is strong in science, education and culture.

We need to improve Europe's innovation performance not only in universities. Along the whole chain, from research to retail, notably through innovation partnerships. We need an Innovation Union. Next month, the Commission will set out how to achieve this.

Another key test will be whether Member States are ready to make a breakthrough on a patent valid across the whole European Union. Our innovators are often paying ten times the price faced by their competitors in the United States or in Japan. Our proposal is on the table. It would reduce the cost fundamentally and double the coverage. After decades of discussion, it is time to decide.

We will also act further on red tape. SMEs are being strangled in regulatory knots. 71\% of CEOs say that the biggest barrier to their success is bureaucracy. The Commission has put proposals on the table to generate annual savings of \euro38 billion for European companies.

Stimulating innovation, cutting red tape and developing a highly-skilled workforce: these are ways to ensure that European manufacturing continues to be world class. A thriving industrial base in Europe is of paramount importance for our future. Next month, the Commission will present a new industrial policy for the globalisation era.

We have the people, we have the companies. What they both need is an open and modern single market.

The internal market is Europe's greatest asset, and we are not using it enough. We need to deepen it urgently.

Only 8\% of Europe's 20 million SMEs engage in cross-border trade, still fewer in cross-border investment. And even with the internet, over a third of consumers lack the confidence to make cross-border purchases.

At my request, Mario Monti presented an expert report and has identified 150 missing links and bottlenecks in the internal market.

Next month we will set out how to deepen the Single Market in a comprehensive and ambitious Single Market Act.

Energy is a key driver for growth and a central priority for action: we need to complete the internal market of energy, build and interconnect energy grids, and ensure energy security and solidarity. We need to do for energy what we have done for mobile phones: real choice for consumers in one European marketplace.

This will give us a real energy community in Europe.

We need to make frontiers irrelevant for pipelines or power cables.

To have the infrastructure for solar and wind energy.

To ensure that across the whole of Europe, we have a common standard so that charging electric car batteries becomes as natural as filling up the tank.

Over the next year, we will bring forward an energy action plan, an infrastructure package and an energy efficiency action plan to put this vision in place. I myself will travel to the Caspian region later this year to promote the Southern Corridor as a means of enhancing our security of supply.

To build a resource-efficient Europe, we need to look beyond energy. In the 20th century the world enjoyed phenomenal resource-intensive growth. We saw in the 20th century globally a four-fold growth in population accompanied by a 40-fold growth in economic output. But in the same period we also increased our use of fossil fuels 16 times, our fishing catches 35 times, our water use 9 times. And our carbon emissions increased 17 times.

That means we have to deliver on our climate and energy package, as a core driver for change. This means integrating the different strands of policy on climate change, energy, transport and environment into a coherent approach on resource efficiency and a low carbon future.

A forward-looking agricultural sector will play a major role in European measures to address some of the biggest challenges ahead, such as global food security, biodiversity loss and the sustainable management of natural resources. So will our maritime policy.

All of this will not only strengthen our economy tomorrow: it will provide new openings today. Jobs in the eco-industry have been increasing by 7\% a year since 2000. I want to see 3 million "green jobs" by 2020, 3 million green collar workers that complement our blue and white collar workers.

We need sustainable growth, and we need smart growth. Half of European productivity growth over the last 15 years was driven by information and communication technologies. This trend is set to intensify. Our European Digital Agenda will deliver a single digital market worth 4\% of EU GDP by 2020.

Honourable Members,

Everything we do is for the citizens of Europe. A fundamental dimension of our European project is precisely building an area of freedom, security and justice.

We are working hard to implement the Stockholm action plan. We will make a real push on asylum and migration.

Legal migrants will find in Europe a place where human values are respected and enforced. At the same time, we will crack down on the exploitation of illegal immigrants within Europe and at our borders. The Commission will make new proposals on policing our external borders.

And we will bring forward an internal security strategy to tackle threats of organised crime and terrorism.

Europeans will find that their fundamental rights and obligations exist wherever they go. Everyone in Europe must respect the law, and the governments must respect human rights, including those of minorities. Racism and xenophobia have no place in Europe. On such sensitive issues, when a problem arises, we must all act with responsibility. I make a strong appeal not to re-awaken the ghosts of Europe's past.

An area of freedom, liberty and security, will create a place where Europeans can prosper.

Honourable Members,

Another challenge is sorting out the future budget of the European Union.

Next month, we will come forward with the Commission's first ideas for the budget review. It shall launch an open debate without taboos to prepare our legislative proposals that will be presented in the second quarter of next year.

We need to spend our money where we get most value for it. And we should invest it where it leverages growth and delivers on our European agenda. The quality of spending should be the yardstick for us all.

So it is not only important to discuss the quantity, but also the quality of spending and investment.

I believe Europe offers real added value. That is why I will be pushing for an ambitious post-2013 budget for Europe.

I believe we should pool our means to back our policy priorities.

The issue is not about spending more or less, but spending more intelligently, by looking at European and national budgets together. The EU budget is not for Brussels – it is for the people that you represent: for the unemployed workers being retrained by the Social Fund; for the students that participate in the Erasmus programme; for the regions that benefit from the Cohesion Fund.

Energy interconnections, research, and development aid are obvious examples where a Euro spent at European level gets you more than a Euro spent at national level. Some Member States are seeing this logic even in areas of core national competence, like defence. They recognize that huge savings could be made if they pool some of their means and activities. Pooling money at the European level allows Member States to cut their costs, avoid overlaps and get a better return on their investment.

That's why we should also explore new sources of financing for major European infrastructure projects. For instance, I will propose the establishment of EU project bonds, together with the European Investment Bank. We will also further develop Public Private Partnerships.

As this Parliament has made clear, we must also address the issue of own resources. The present system is stretched to its limits – propped up by a byzantine set of corrections. Our citizens deserve a fairer and more efficient and transparent system. Some will not agree with all the ideas we will raise; I find it extraordinary that some are already rejecting them, even before knowing what they will be.

I know that one issue of interest to this Parliament is the duration of the next budget. Various options exist. I would like to look at a 10-year framework, with a mid-term review of the financial dimension after five years – a "five plus five" option. This will give us longer term planning and a clearer link with the mandates of both our institutions.

Of course, part of a credible European budget is the rigorous pursuit of savings. I am looking at the administrative costs within the Commission and other Community bodies like Agencies. We need to eliminate all pockets of inefficiency. We will build on recommendations from the Court of Auditors to improve financial management.

Honourable Members,

The final challenge I want to address today is how we pull our weight on the global stage.

When we deal with our every day problems, we sometimes lose perspective and forget our achievements. A peaceful and successful transition to a European Union that has doubled in size and is negotiating further accessions. A sound currency, the euro, that is a major currency of the world. A strong partnership with our neighbourhood that strengthens us all. If we act decisively, then we have nothing to fear from the 21st century.

As the strategic partnerships of the 21st century emerge, Europe should seize the chance to define its future. I am impatient to see the Union play the role in global affairs that matches its economic weight. Our partners are watching and are expecting us to engage as Europe, not just as 27 individual countries. If we don't act together, Europe will not be a force in the world, and they will move on without us: without the European Union but also without its Member States. This is why, in my political guidelines, I called for Europe to be a global player, a global leader – a key task and test for our generation.

Together with High Representative and Vice-President Ashton, I will present our vision of how we can maximise Europe's role in the world. With the European External Action Service, we have the means to match our aspirations.

In our globalized world, the relationships we build with strategic partners determine our prosperity. To be effective on the international stage, we need the weight of the European Union. Size matters, now more than ever.

A good example is the fight against climate change. Copenhagen showed that, while others did not match our ambition, we did not help ourselves by not speaking with one voice. Negotiations may have stalled but climate change has not. I want us to intensify our engagement with international partners to turn their press releases into credible commitments to cut emissions and push forward with fast-start funding.

The next two months will see crucial Summits with strategic partners. The more we are able to establish a common agenda with a clearly defined European interest, the more we will achieve. For example, I see huge potential in developing a transatlantic agenda for growth and jobs.

Where we are already punching our weight is the G20, the forum where the key economic global players address common challenges. When President Van Rompuy and I go to Seoul in November and represent the European Union, we want to see concrete results:

Further progress in global economic coordination.

More stable and responsible financial markets and agreement on reform of international financial institutions.

More effective global financial safety nets.

More progress on a G20 development agenda.

We will continue to show leadership in this forum and work closely with the French G8/G20 Presidency next year.

We also want to see support for the Doha Round. Trade boosts growth and prosperity. We will also pursue bilateral and regional Free Trade Agreements. In October, the Commission will present a renewed trade policy to drive new benefits for Europe.

Being open to the world also means standing side by side with developing countries, especially with Africa. When I go to the Millennium Development Goals High-Level Event in New York in 2 weeks' time, I intend to commit, with your support and on behalf of the European Union, an extra \euro1 billion to the Millennium Development Goals.

Being a global player also means standing up for our values. Human rights are not negotiable. I am shocked about how the rights of women are being infringed in many countries. I am appalled when I hear that Sakineh Mohammadi Ashtiani is sentenced to death by stoning. This is barbaric beyond words. In Europe we condemn such acts which have no justification under any moral or religious code.

Our values also mean that we must come to the aid of those facing a crisis situation, anywhere around the world.

Our humanitarian aid to Pakistan is the latest example of Europe's solidarity in action. It is a striking example of the need to present the different contributions of the Commission and the Member States as a truly European aid package. The Member States have the helicopters; they have the civil protection teams. We now need to pool them to create a real European crisis response capacity. This is what the Commission will propose in October. And I urge the Member States to show they are serious about the Union punching its weight in this area.

We are making progress on a common foreign policy. But let's be under no illusions: we will not have the weight we need in the world without a common defence policy. I believe now is the moment to address this challenge.

Honourable Members,

We are still bedding down the new institutional set-up of Europe created by the Lisbon Treaty.

What really matters is what the institutions deliver to the people. What matters is the difference Europe makes in their daily lives.

The secret of Europe's success is its unique Community model. More than ever, the Commission must drive the political agenda with its vision and proposals.

I have called for a special relationship between the Commission and Parliament, the two Community institutions par excellence. I am intensifying my political cooperation with you.

Europe is not only Brussels or Strasbourg. It is our regions. It is the cities, towns and villages you come from. When you walk round your constituencies, you can point to the European projects that are so important for their prosperity.

At the end of the day, we are all in the same boat, the European institutions, the Member states, the regions. The Union will not achieve its objectives in Europe without the Member States. And the Member States will not achieve their objectives in the world without the European Union.

Honourable Members,

The citizens of Europe expect us to take the action needed to get out of this crisis.

We must show them that the common efforts we are making today will lead to new jobs, new investments, and a Europe fit for the future.

I am confident that Europe has what it takes. We will get the results we are reaching for.

One thing is certain, it is not with pessimism that we will win this battle. It is with confidence, with a strong common will.

Today, I have outlined how I see the European Union doing that.

I have committed to deliver the proposals to build our economic union.

I have made the case to fast-track our reform agenda.

I have set out how to modernise our social market economy to deliver growth and jobs in a smart, sustainable and inclusive economy through our Europe 2020 flagship initiatives.

I have set out how to achieve a common energy policy in Europe.

I have defended the need for an area of freedom, security and justice, where Europeans will find that their fundamental rights and obligations exist wherever they go.

I have made clear that the Commission will strive for an ambitious budget.

I have proposed to develop EU project bonds to finance major European projects.

I have announced our reinforced commitment to the Millennium Development Goals.

I have made the case clear of why we need a common crisis response capacity and a also a common foreign and a common defence policy.

And I have urged European leaders to act together if they want Europe to be a global player and defend the European interest.

It is indeed a transformational, an ambitious and challenging agenda.

For Europe to succeed, the Commission needs your support for a stronger, a fairer Europe for the benefits of our citizens

Thank you.
 \newpage\section{Speech 2 - Barroso - 2011-09-28}
\url{https://ec.europa.eu/commission/presscorner/detail/en/SPEECH_11_607}\\[3mm]
Mr President,

Honourable Members,

Minister,

We must be honest and clear in our analysis of the state of the Union.

We are facing the biggest challenge in the history of our Union.

This crisis is financial, economic and social. But it is also a crisis of confidence. A crisis of confidence in our leaders, in Europe itself, and in our capacity to find solutions.

The roots of the crisis are well-known. Europe has not met the challenges of competitiveness. Some of our Member States have lived beyond their means. Some behaviours in the financial markets have been irresponsible and inadmissible. We have allowed imbalances between our Member States to grow, particularly in the euro area.

Tectonic shifts in the world order and the pressures of globalisation, have made matters even worse.

The result is clear: concern in our societies. Fear among our citizens for the future. A growing danger of a retreat into national, not to say nationalist, feeling.

Populist responses are calling into question the major successes of the European Union: the euro, the single market, even the free movement of persons.

Today we can say that the sovereign debt crisis today is, above all, a crisis of political confidence. And our citizens, but also people in the outside world, are observing us and wondering – are we really a Union? Do we really have the will to sustain the single currency?

Are the most vulnerable Member States really determined to carry out essential reforms?

Are the most prosperous Member States really ready to show solidarity?

Is Europe really capable of achieving growth and creating jobs?

I assert here today:

Yes, the situation is serious. But there are solutions to the crisis.

Europe has a future, if we restore confidence.

And to restore confidence we need stability and growth. But also political will, political leadership.

Together we must propose to our citizens a European renewal.

We must translate into deeds what was stated in the Berlin Declaration, signed by the Commission, by Parliament and by the European Council on the occasion of the 50th anniversary of the signature of the Rome Treaties. It was said then: ‘Wir leben heute miteinander, wie es nie zuvor möglich war. Wir Bürgerinnen und Bürger der Europäischen Union sind zu unserem Glück vereint.' - ‘Today we live together as was never possible before. We, the citizens of the European Union, have united for the better.' It is a declaration. And words count. This expression of will must be translated into everyday courage.

Working with our institutions, and not working against them, we can succeed.

For some, the main consideration is the need for stability. For others, it is growth.

I say we need both.

Some preach discipline. Others, solidarity.

We need both.

The time for piecemeal solutions is over. We need to set our minds on global solutions. A greater ambition for Europe.

Today we are at a turning point in our history. A moments when, if we do not integrate further, we risk fragmentation.

It is therefore a question of political will, a test for our whole generation.

And I say to you, yes, it is possible to emerge from this crisis. It is not only possible, but it is necessary. And political leadership is about making possible that which is necessary.

Honourable members,

Let me start with Greece. Greece is, and will remain, a member of the euro area. Greece must implement its commitments in full and on time. In turn, the other euro area members have pledged to support Greece and each other. As stated at the euro area Summit on 21 July: "We are determined to continue to provide support to countries under programmes until they have regained market access, provided they successfully implement those programmes."

That is why I created the Task Force for Greece.

We have just launched an action plan based on two major pillars:

    Around 100 viable and high-quality projects, investing in all Greek regions, to make the best use of Greece's remaining allocation of the structural funds.

    And a major drive to reduce bureaucratic procedures for European co-funded projects.

\euro 15 billion remain to be spent in Greece from the structural funds. This will support the Greek economy with an urgent programme of technical assistance to the Greek administration.

A programme of \euro 500 million Euros to guarantee European Investment Bank loans to Greek SMEs is already under way. The Commission is also considering a wider guarantee mechanism to help banks lend again to the real economy.

All of this represents a huge support to Greece's fight back and Greece will have to deliver concrete results. It must break with counter productive practices and resist vested interests.

But we have to be clear about this. This is not a sprint, but a marathon.

The task of building a Union of stability and responsibility is not only about Greece.

The economic outlook that we face is very difficult. We are confronted with the negative effects of an ongoing global re-assessment of risks. It is therefore our responsibility to rebuild confidence and trust in the euro and our Union as a whole.

And we can do this by showing that we are able to take all the decisions needed to run a common currency and an integrated economy in a competitive, inclusive and resource-efficient way. For this we need to act in the short, in the medium and the long term.

The first step is to quickly fix the way we respond to the sovereign debt crisis.

This will require stronger mechanisms for crisis resolution. We need credible firepower and effective firewalls for the euro.

We have to build on the EFSF and the upcoming European Stability Mechanism.

The EFSF must immediately be made both stronger and more flexible. This is what the Commission proposed already in January. This is what Heads of State and Government of the euro area agreed upon on 21 July. Only then, when you ratify this, will the EFSF be able to:

    deploy precautionary intervention;

    intervene to support the recapitalisation of banks,

    intervene in the secondary markets to help avoid contagion 

Once the EFSF is ratified, we should make the most efficient use of its financial envelope. The Commission is working on options to this end.

Moreover we should do everything possible to accelerate the entry into force of the ESM.

And naturally we trust that the European Central Bank – in full respect of the Treaty – will do whatever is necessary to ensure the integrity of the euro area and to ensure its financial stability.

But we cannot stop there. We must deepen economic coordination and integration, particularly in the euro area.

This is at least as big a political task as an economic one.

Today, you will vote on the so-called "six-pack" proposals that we put in front of you and the Council one year ago. This "six-pack" reforms the Stability and Growth Pact and widens surveillance to macro-economic imbalances. We are now back very close to what the Commission originally put on the table. You have played a decisive role in keeping the level of ambition of these proposals, and I really want to thank you and congratulate you for that.

This legislation will give us much stronger enforcement mechanisms. We can now discuss Member States' budgetary plans before national decisions are taken. This mix of discipline and integration holds the key to the future of the euro area. Only with more integration and discipline we can have a really credible euro area.

Honourable members,

These are indeed important steps forward, but we must go further. We need to complete our monetary union with an economic union. We need to achieve the tasks of Maastricht.

It was an illusion to think that we could have a common currency and a single market with national approaches to economic and budgetary policy. Let's avoid another illusion that we can have a common currency and a single market with an intergovernmental approach.

For the euro area to be credible – and this not only the message of the federalists, this is the message of the markets – we need a truly Community approach. We need to really integrate the euro area, we need to complete the monetary union with real economic union. And this truly Community approach can be built how? In the coming weeks, the Commission will build on the six-pack and present a proposal for a single, coherent framework to deepen economic coordination and integration, particularly in the euro area. This will be done in a way that ensures the compatibility between the euro area and the Union as a whole. We do not want the euro area to break of course the great acquis of the single market and all our four freedoms.

At the same time, we can pool decision making to enhance our competitiveness. This could be done by integrating the Euro Plus Pact into this framework, in full respect of the national implementation competences.

For all of this to work, we need more than ever the independent authority of the Commission, to propose and assess the actions that the Member States should take. Governments, let's be frank, cannot do this by themselves. Nor can this be done by negotiations between governments.

Indeed, within the Community competences, the Commission is the economic government of the Union, we certainly do not need more institutions for this.

For a reason the Treaties have created supra-national institutions. For a reason the European Commission, the European Central Bank, the European Court of Justice were created. The Commission is the guarantor of fairness. Moreover, the Commission, which naturally works in partnership with the Member States, is voted by and accountable to this House. The directly elected Parliament both of the euro area and of the European Union as a whole.

Honourable members,

It is also time to have unified external representation of the euro area. In accordance with the Treaty the Commission will make proposals for this purpose.

A Union of stability and responsibility built on this basis and with common approach will also allow the Member States to seize fully the advantages of a bigger market for the issuance of sovereign debt.

Once the euro area is fully equipped with the instruments necessary to ensure both integration and discipline, the issuance of joint debt will be seen as a natural and advantageous step for all. On condition that such Eurobonds will be "Stability Bonds": bonds that are designed in a way that rewards those who play by the rules, and deters those who don't. As I already announced to this house, the Commission will present options for such "Stability Bonds" in the coming weeks.

Some of these options can be implemented within the current Treaty, whereas fully fledged 'Eurobonds' would require Treaty change. And this is important because, Honourable Members, we can do a lot within the existing Treaty of Lisbon. And there is no excuse for not doing it, and for not doing it now.

But it may be necessary to consider further changes to the Treaty.

I am also thinking particularly of the constraint of unanimity. The pace of our joint endeavour cannot be dictated by the slowest. And today we have a Union where it is the slowest member that dictates the speed of all the other Member States. This is not credible also from the markets' point of view, this is why we need to solve this problem of decision making. A Member State has of course the right not to accept decisions. That is a question, as they say, of national sovereignty. But a Member State does not have the right to block the moves of others, the others also have their national sovereignty and if they want to go further, they should go further.

Our willingness to envisage Treaty change should not be a way or an excuse to delay the reforms that are necessary today but I believe that this longer term perspective will reinforce the credibility of our decisions now.

A Union of stability and responsibility means swiftly completing the work on a new system of regulation for the financial sector. We need well-capitalised, responsible banks lending to the real economy.

Much has been said about the alleged vulnerability of some of our banks. European banks have substantially strengthened their capital positions over the past year. They are now raising capital to fill the remaining gaps identified by the stress tests in summer. This is necessary to limit the damage to financial market turbulence on the real economy and on jobs.

Over the last three years, we have designed a new system of financial regulation.

Let's remember, we have already tabled 29 pieces of legislation. You have already adopted several of them, including the creation of independent supervising authorities, which are already working. Now it is important to approve our proposals for new rules on:

    derivatives;

    naked short selling and credit default swaps;

    fair remuneration for bankers.

These propositions are there, they should be adopted by the Council and by the Parliament. The Commission will deliver the remaining proposals by the end of this year, namely rules on:

    credit rating agencies;

    bank resolution;

    personal responsibility of financial operatives.

So we will be the first constituency in the G20 to have delivered on our commitment to global efforts for financial regulation.

Honourable members,

In the last three years, Member States - I should say taxpayers - have granted aid and provided guarantees of \euro 4.6 trillion to the financial sector. It is time for the financial sector to make a contribution back to society. That is why I am very proud to say that today, the Commission adopted a proposal for the Financial Transaction Tax. Today I am putting before you a very important text that if implemented may generate a revenue of about \euro 55 billion per year. Some people will ask "Why?". Why? It is a question of fairness. If our farmers, if our workers, if all the sectors of the economy from industry to agriculture to services, if they all pay a contribution to the society also the banking sector should make a contribution to the society.

And if we need – because we need – fiscal consolidation, if we need more revenues the question is where these revenues are coming from. Are we going to tax labour more? Are we going to tax consumption more? I think it is fair to tax financial activities that in some of our Member States do not pay the proportionate contribution to the society.

It is not only financial institutions who should pay a fair share. We cannot afford to turn a blind eye to tax evasion. So it is time to adopt our proposals on savings tax within the European Union. And I call on the Member States to finally give the Commission the mandate we have asked for to negotiate tax agreements for the whole European Union with third countries.

Honourable members,

Stability and responsibility are not enough on their own. We need stability but we also need growth. We need responsibility but we also need solidarity.

The economy can only remain strong if it delivers growth and jobs. That's why we must unleash the energy of our economy, especially the real economy.

The forecasts today point to a strong slowdown.

But significant growth in Europe is not an impossible dream. It will not come magically tomorrow. But we can create the conditions for growth to resume. We have done it before. We must and we can do it again.

It is true that we do not have much room for a new fiscal stimulus.

But that does not mean that we cannot do more to promote growth.

First, those who have fiscal space available must explore it – but in a sustainable way.

Second, all member states need to promote structural reforms so that we can increase our competitiveness in the world and promote growth.

Together, we can and must tap the potential of the Single Market, exploit all the benefits of trade and mobilise investment at the Union level.

Let me start with the Single Market.

Full implementation of the Services Directive alone could, according to our estimates, deliver up to \euro 140 billion in economic gains.

But today, two years after the deadline for implementation, several Member States have still not adopted the necessary laws.

So we are not benefiting from all the possible gains from having a true services liberalisation in Europe.

But we can also do more.

We must adopt what is on the table. We have adopted the Single Market Act in the European Commission. A number of key initiatives are ready.

We are close to having a European patent which would cut the cost of protection to 20\% of current costs. I expect this is to be concluded by the end of this year.

Moreover, for the Single Market Act, we should consider a fast track legislative procedure. By the way, in many areas we should take a fast track legislative procedure because we are living in real emergency times. This will allow us to respond to these extraordinary circumstances.

And growth in the future will depend more and more on harnessing information technology. We need a digital single market, which will benefit each and every European by around \euro1500 per year – by using the possibilities of e-commerce to ending, for instance, mobile roaming charges.

An extra 10 \% in broadband penetration would bring us between 1 and 1.5 \% of extra annual growth.

In a competitive world we must be also well-educated with skills to face these new challenges. We must innovate. And we must act in a sustainable way.

We have already presented detailed proposals on innovation, resource-efficiency and how we can strengthen our industrial base.

Modern industrial policy is about investing in research and innovation.

We need to accelerate the adoption of our efforts to boost the use of venture capital to fund young, innovative companies across Europe.

Sustainable jobs will come if we focus on innovation and new technologies, including green technologies. We must see that "green" and growth go together.

For example, the renewables sector has already created 300,000 jobs in past 5 years in the European Union. The global green technology market will triple over the next decade.

We must focus our action on where it makes a real impact. Growth of the future means we must actively pursue also our smart regulation agenda, which will give a saving of \euro 38 billion for European companies, particularly for SMEs. But Member States must also do their part in reducing the administrative burden.

But we also need investment. These reforms are important but we also need some kind of investment at European level.

A Union of growth and solidarity needs modern, interconnected infrastructures.

We have proposed for the next Multi-Annual Financial Framework (MFF) to create a facility to connect Europe – in energy, in transport, in digital.

This innovative part of our MFF proposal has to be seen together with another very important innovative idea: the project bond.

In the coming weeks the Commission will publish its proposals for EU project bonds. We are also proposing pilot projects, so that we can fund that growth. We can do it even before the MFF is adopted. In this way we can frontload some of the major infrastructure investments Europe needs.

The Union and its Member States should urgently consider how to allow our own policy-driven bank, the European Investment Bank to do more – and possibly much more – to finance long-term investment.

To do so, we need to explore ways to reinforce the EIB's resources and capital base so that it can lend to the real economy.

In the year 2000, there was \euro 22 billion of venture capital in Europe. In 2010 there was only \euro 3 billion. If we want to promote entrepreneurship we must reverse this decline and we need that support namely for SMEs.

We can also get more growth out of the Structural Funds, by increasing absorption capacity, using the Structural Funds to support macroeconomic performance. They are essential for innovation, for training and employment, and for SMEs.

I would also like to urge this House to adopt by the end of the year the proposals we made in August to increase cofinancing rates to those countries with assistance programmes. This will inject essential funding into these economies, while reducing pressure on national budgets.

Honourable members,

Reforms to our labour markets, public finances and pension systems require a major effort from all parts of society.

We all know these changes are necessary, so that we can reform our social market economy and keep our social model. But it is imperative that we hold on to our values – values of fairness, of inclusiveness and of solidarity.

Right now we need to give concrete hope to the 1 in 5 of our young people who cannot find work. In some countries, the situation of our young people is simply dramatic. I want to call on companies to make a special effort to provide internships and apprenticeships for young people. These can be supported by the European Social Fund.

By getting businesses, the social partners, national authorities and the Union level working in a "Young Opportunities Initiative", we can make a difference. This I believe is the most urgent social matter to respond to the anxiety of our young people that cannot find a job and it is much better to have an apprenticeship, a traineeship, than to be with that anxiety in the streets expressing that lack of confidence in the Union as a whole.

We must accelerate the most urgent parts of our Growth and Jobs Plan, Europe 2020. The Commission will focus on the situation of young people in each and every Member State in its Country-specific recommendations for next year.

I believe we must give our future a real chance.

Right now we also need to act to help the 80 million Europeans at risk of poverty. This means that the Council must finally approve our proposal to safeguard the programme for the supply of food for the most deprived persons. I would like to thank this Parliament for the political support it has given to our proposed solution.

Honourable Members,

Fifty years ago, 12 countries in Europe came together to sign the Social Charter. It was exactly in October 50 years ago. Today, that Charter has 47 signatories, including all our Member States.

To guarantee these fundamental values in Europe, I believe we need to boost the quality of social dialogue at European level. The renewal of Europe can only succeed with the input and the ownership of all the social partners – of trade unions, of workers, of businesses, civil society in general.

We should remember that our Europe is a Europe of citizens. As citizens, we all gain through Europe. We gain a European identity and citizenship apart from our national citizenship. European citizenship adds a set of rights and opportunities. The opportunity to freely cross borders, to study and work abroad. Here again, we must all stand up and preserve and develop these rights and opportunities. Just as the Commission is doing now with our proposals on Schengen. We will not tolerate a rolling back of our citizens' rights. We will defend the freedom of circulation and all the freedoms in our Union.

Honourable Members,

The Commission's activities, as you well know, cover many other fields. I cannot discuss them all here, but they are mentioned in the letter which I sent to the Parliament's President and which you have all received.

Before I conclude, however, let me speak about the European Union's external responsibilities. I want to see an open Europe, a Europe engaging with the world.

European action in the world is not only the best guarantee for our citizens and for the defence of our interests and our values: it is also indispensable to the world. Today it is fashionable to talk of a G2. I believe the world does not want a G2. It is not in the interests of the Two themselves. We know the tension that bipolarity created during the Cold War. If we want to have a just world and an open world, I believe that Europe is more necessary than ever.

The rapidly-changing world needs a Europe that assumes its responsibilities. An influential Europe, a Europe of 27 - with the accession of Croatia soon to be 28. A Europe that continues to show the way, whether in matters of trade or of climate change. At a time when major events await us, from Durban to Rio +20, Europe must retain its position of leadership on these questions.

Let us also turn our attention to our southern neighbours. The Arab Spring is a profound transformation which will have lasting consequences not only for those peoples but also for Europe. Europe should be proud. We were the first to stand alongside those Tunisians, Egyptians and Libyans who wanted democracy and freedom. Europe is supporting these legitimate aspirations, namely through our Partnership for Democracy and Shared Prosperity.

The Arab Spring should give hope for peace throughout the region.l Europe wishes to see a Palestinian State living in peace alongside the State of Israel.

Let us also turn our attention to our eastern neighbours. On Friday I shall take part in the Eastern Partnership Summit in Warsaw. I shall go there with the ambition to forge a closer political relationship and tighter economic integration between us and our partners in the region. The EU has extraordinary transformational power. It is an inspiration for many people in the world, and if those countries embark on a thorough process of reform we can help them. We can further political and economic ties.

Finally, let us not forget the most deprived of all and let us live up to our commitments in attaining the Millennium Development Goals.

We must also be realistic and recognise that, if Europe is to exert its influence fully, if Europe really wants to be a power, we must strengthen the Common Foreign and Security Policy. It must be credible. It must be based on a common security and defence dimension if we are really to count in the world.

Long gone is the time when people could oppose the idea of European defence for fear that it might harm the Transatlantic relationship. As you have noticed, today it is the Americans themselves who are asking us to do more as Europeans. The world has changed, the world is still changing fundamentally. Do we really want to count in the world?

Hence, at a time when defence budgets are under pressure, we must do more together with the means at our disposal.

The Commission is assuming playing its part: we are working towards a single defence market. We are using our under the Treaty with a view to developing a European defence industrial base.

Honourable Members,

Let us not be naive: the world is changing and if Europe is to count in the world and defend its citizens' interests we need the political dimension and the defence dimension to give us weight and a say in the world's future.

Honourable Members,

I conclude.

At the end of our mandate, in 2014, it will be exactly a century since the Great War broke out on our continent. A dark period which was followed by the Second World War, one of the most dramatic pages in the history of Europe and the world. Today such horrors are unimaginable in Europe, largely because we have the European Union. Thanks to the European vision, we have built a guarantee of peace in our continent through economic and political integration. That is why we cannot allow this great work to be placed in jeopardy. It was a gift from previous generations. It will not be our generation that calls it into question. And let us be clear: if we start to break up Europe, if we start to backtrack on Europe's major achievements, we will doubtless have to face the risk of fragmentation.

As I said, the root of the crisis we are now facing is a political problem. It is a test of our willingness to live together. That is why we have built common institutions. That is why we must safeguard the European interest.

The reality today is that intergovernmental cooperation is not enough to pull Europe out of this crisis, to give Europe a future. On the contrary, certain forms of intergovernmentalism could lead to renationalisation and fragmentation. Certain forms of intergovernmentalism could be the death of the united Europe we wish for.

Let us not forget that the decisions we take now, or fail to take, are going to shape our future. I feel hurt when I hear people in other parts of the world, with a certain condescension, telling us Europeans what we should do. I think, frankly, we have problems, very serious problems, but I also think we do not have to apologise for our democracies. We do not have to apologise for our social market economy. We should ask our institutions, but also our Member States, Paris, Berlin, Athens, Lisbon and Dublin, to show a burst of pride in being European, a burst of dignity, and say to our partners: ‘Thanks for the advice, but we can overcome this crisis together'. I feel that pride in being European.

And pride in being European is not just about our great culture, our great civilisation, everything to which we have given birth. It is not pride only in the past, it is pride in our future. That is the confidence that we have to re-create among ourselves. It is possible.

Some say it is very difficult, it is impossible. I would remind them of the words of a great man, a great African, Nelson Mandela: ‘It always seems impossible until it is done'. Let's do it. We can do it with confidence. We can do it, we can renew our Europe.

Thank you for your attention.
 \newpage\section{Speech 3 - Barroso - 2012-09-12}
\url{https://ec.europa.eu/commission/presscorner/detail/en/SPEECH_12_596}\\[3mm]
Mr President,

Honourable Members,

1. Analysis of the situation

It is an honour to stand before you today to deliver this third State of the Union address.

At a time when the European Union continues to be in crisis.

A financial and economic crisis. A social crisis. But also a political crisis, a crisis of confidence.

At its root, the crisis results from:

    Irresponsible practices in the financial sector;

    Unsustainable public debt, and also;

    A lack of competitiveness in some Member States.

On top of that, the Euro faces structural problems of its own. Its architecture has not been up to the job. Imbalances have built up.

This is now being corrected. But it is a painful, difficult, effort. Citizens are frustrated. They are anxious. They feel their way of life is at risk.

The sense of fairness and equity between Member States is being eroded. And without equity between Member States, how can there be equity between European citizens?

Over the last four years, we have made many bold decisions to tackle this systemic crisis. But despite all these efforts, our responses have not yet convinced citizens, markets or our international partners.

Why? Because time and again, we have allowed doubts to spread. Doubts over whether some countries are really ready to reform and regain competitiveness. Doubts over whether other countries are really willing to stand by each other so that the Euro and the European project are irreversible.

On too many occasions, we have seen a vicious spiral. First, very important decisions for our future are taken at European summits. But then, the next day, we see some of those very same people who took those decisions undermining them. Saying that either they go too far, or that they don't go far enough. And then we get a problem of credibility. A problem of confidence.

It is not acceptable to present these European meetings as if they were boxing events, claiming a knockout victory over a rival. We cannot belong to the same Union and behave as if we don't. We cannot put at risk nine good decisions with one action or statement that raises doubts about all we have achieved.

This, Honourable Members, reveals the essence of Europe's political crisis of confidence. If Europe's political actors do not abide by the rules and the decisions they have set themselves, how can they possibly convince others that they are determined to solve this crisis together?

Mr President,

Honourable Members,

2. The challenge – a new thinking for Europe

A crisis of confidence is a political crisis. And, the good thing is that, in a democracy, there is no political problem for which we cannot find a political solution.

That is why, here today, I want to debate with you the fundamental political questions - where we are now and how we must move forward. I want to focus on the political direction and the vision that shall inspire our policy decisions.

I will of course not list all these individual decisions. You are receiving the letter I addressed to the President of the European Parliament, and that sets out the Commission's immediate priorities. We will discuss them with you before adopting the Commission Work Programme later in the autumn.

My message to you today is this: Europe needs a new direction. And, that direction can not be based on old ideas. Europe needs a new thinking.

When we speak about the crisis, and we all speak about the crisis, have we really drawn all the consequences for our action? When we speak about globalisation, and we all speak a lot about globalisation, have we really considered its impact on the role of each of our Member States?

The starting point for a new thinking for Europe is to really draw all the consequences of the challenges that we are facing and that are fundamentally changing our world.

The starting point is to stop trying to answer the questions of the future with the tools of the past.

Since the start of the crisis, we have seen time and again that interconnected global markets are quicker and therefore more powerful than fragmented national political systems. This undermines the trust of citizens in political decision making. And it is fuelling populism and extremism in Europe and elsewhere.

The reality is that in an interconnected world, Europe's Member States on their own are no longer able to effectively steer the course of events. But at the same time, they have not yet equipped their Union - our Union —with the instruments needed to cope with this new reality. We are now in a transition, in a defining moment. This moment requires decisions and leadership.

Yes, globalisation demands more European unity.

More unity demands more integration.

More integration demands more democracy, European democracy.

In Europe, this means first and foremost accepting that we are all in the same boat.

It means recognising the commonality of our European interests.

It means embracing the interdependence of our destinies.

And it means demanding a true sense of common responsibility and solidarity.

Because when you are on a boat in the middle of the storm, absolute loyalty is the minimum you demand from your fellow crew members.

This is the only way we will keep up with the pace of change. It is the only way we will get the scale and efficiency we need to be a global player. It is the only way to safeguard our values, because it is also a matter of values, in a changing world.

In the 20th century, a country of just 10 or 15 million people could be a global power. In the 21st-century, even the biggest European countries run the risk of irrelevance in between the global giants like the US or China.

History is accelerating. It took 155 years for Britain to double its GDP per capita, 50 years for the US, and just 15 years for China. But if you look at some of our new Member States, the economic transformation going on is no less impressive.

Europe has all the assets it takes. In fact much more so than previous generations faced with similar or even greater challenges.

But we need to act accordingly and mobilize all these resources together.

It is time to match ambitions, decisions, and actions.

It is time to put a stop to piecemeal responses and muddling through.

It is time to learn the lessons from history and write a better future for our Europe.

Mr President,

Honourable Members,

3. Response to the situation – the 'decisive deal for Europe'

What I demand and what I present to you today is a Decisive Deal for Europe.

A decisive deal to project our values, our freedom and our prosperity into the future of a globalized world. A deal that combines the need to keep our social market economies on one hand and the need to reform them on the other. A deal that will stabilise the EMU, boost sustainable growth, and restore competitiveness. A deal that will establish a contract of confidence between our countries, between Member States and the European institutions, between social partners, and between the citizens and the European Union.

The Decisive Deal for Europe means that:

We must leave no doubt about the integrity of the Union or the irreversibility of the Euro. The more vulnerable countries must leave no doubts about their willingness to reform. About their sense of responsibility. But the stronger countries must leave no doubts about their willingness to stick together. About their sense of solidarity. We must all leave no doubts that we are determined to reform. To REFORM TOGETHER.

The idea that we can grow without reform, or that we can prosper alone is simply false. We must recognise that we are in this together and must resolve it together.

This decisive deal requires the completion of a deep and genuine economic union, based on a political union.

a) Economic union:

Let me start with Europe's economy.

Firstly, we need growth. Sustainable growth

Growth is the lifeblood of our European social market model: it creates jobs and supports our standard of living. But we can only maintain growth if we are more competitive.

At the national level it means undertaking structural reforms that have been postponed for decades. Modernising public administration. Reducing wasteful expenditure. Tackling vested interests and privileges. Reforming the labour market to balance security with flexibility. And ensuring the sustainability of social systems.

At the European level, we need to be more decisive about breaking down barriers, whether physical, economic or digital.

We need to complete the single market.

We need to reduce our energy dependence and tap the renewable energy potential.

Promoting competitiveness in sectors such as energy, transport or telecoms could open up fresh competition, promote innovation and drive down prices for consumers and businesses.

The Commission will shortly present a Single Market Act II. To enable the single market to prosper, the Commission will continue to be firm and intransigent in the defence of its competition and trade rules. Let me tell you frankly, If it was left to the Member States, I can tell you they will not resist pressure from big corporations or large external powers.

We need to create a European labour market, and make it as easy for people to work in another country as it is as home.

We need to explore green growth and be much more efficient in our use of resources.

We have to be much more ambitious about education, research, innovation and science.

Europe is a world leader in key sectors such as aeronautics, automotives, pharmaceuticals and engineering, with global market shares above a third. Industrial productivity increased by 35\% over the last decade despite the economic slowdown. And today, some 74 million jobs depend on manufacturing. Every year start-up firms in the EU create over 4 million jobs. We need to build on this by investing in our new industrial policy and creating a business environment that encourages entrepreneurship and supports small businesses.

This means making the taxation environment simpler for businesses and more attractive for investors. Better tax coordination would benefit all Member States.

We also need a pro-active trade policy by opening up new markets.

This is the potential of Europe's economy. This is the goldmine that is yet to be fully explored. Fully implementing the Growth Compact agreed at the June European Council can take us a long way.

And we could go further, with a realistic but yet ambitious European Union budget dedicated to investment, growth and reform. Let's be clear. The European budget is the instrument for investment in Europe and growth in Europe. The Commission and this Parliament, indeed all pro-European forces, because most member States support our proposal, must now stand together in support of the right multi-annual financial framework that will take us to 2020. It will place little burden on Member States, especially with our proposed new own resources system. But it would give a great boost to their economies, their regions, their researchers, their students, their young people who seek employment, or their SMEs.

It is a budget for growth, for economic, social and territorial cohesion between Member States and within Member States.

It is a budget that will help complete the single market by bridging gaps in our energy, transport and telecoms infrastructure through the Connecting Europe Facility.

It is a budget for a modern, growth-oriented agriculture capable of combining food security with sustainable rural development.

It is a budget that will promote a research intensive and innovative Europe through Horizon 2020. Because we need this European scale for research

This will be a real test of credibility for many of our some Member States. I want to see if the same member States who are all the time talking about investment and growth will now support a budget for growth at the European level.

The budget is also the tool to support investment in our growth agenda, Europe 2020, which we need now more than ever before.

Europe 2020 is the way to modernise and preserve the European social market economy.

Honourable Members,

Our agenda of structural reform requires a major adjustment effort. It will only work if it is fair and equitable. Because inequality is not sustainable.

In some parts of Europe we are seeing a real social emergency.

Rising poverty and massive levels of unemployment, especially among our young people.

That is why we must strengthen social cohesion. It is a feature that distinguishes European society from alternative models.

Some say that, because of the crisis, the European Social model is dead. I do not agree.

Yes, we need to reform our economies and modernise our social protection systems. But an effective social protection system that helps those in need is not an obstacle to prosperity. It is indeed an indispensable element of it. Indeed, it is precisely those European countries with the most effective social protection systems and with the most developed social partnerships, that are among the most successful and competitive economies in the world.

Fairness and equity means giving a chance to our young people. We are already doing a lot. And before the end of the year, the Commission will launch a Youth Package that will establish a youth guarantee scheme and a quality framework to facilitate vocational training.

Fairness and equity also means creating better and fairer taxation systems.

Stopping tax fraud and tax evasion could put extra billions into the public purse across Europe.

This is why the Commission will fight for an agreement on the revised savings tax directive, and on mandates to negotiate stronger savings tax agreements with third countries. Their completion would be a major source of legitimate tax revenues.

And the Commission will continue to fight for a fair and ambitious Financial Transactions Tax that would ensure that taxpayers benefit from the financial sector, not just that the financial sector benefits from taxpayers. Now that it is clear that agreement on this can only happen through enhanced cooperation, the Commission will do all it can to move this forward rapidly and effectively with those Member States that are willing. Because this is about fairness. And fairness is an essential condition to make the necessary economic reforms socially and politically acceptable. And above all fairness is a question of justice, social justice.

Mr President,

Honourable Members,

In the face of the crisis, important decisions have been taken. Across the European Union, reform and consolidation measures are being implemented. Joint financial backstops are being put in place, and the European institutions have consistently shown that they stands by the Euro.

The Commission is very aware that in the Member States implementing the most intense reforms, there is hardship and there are – sometimes very painful – difficult adjustments. But it is only through these reforms that we can come to a better future. They were long overdue. Going back to the status quo ante is simply impossible.

The Commission will continue to do all it can to support these Member States and to help them boost growth and employment, for instance through the re-programming of structural funds.

Allow me to say a word on Greece. I truly believe that we have a chance this autumn to come to the turning point. If Greece banishes all doubts about its commitment to reform. But also if all other countries banish all doubts about their determination to keep Greece in the Euro area, we can do it.

I believe that if Greece stands by its commitments it should stay in the Euro area, as a member of the European family.

Securing the stability of the Euro area is our most urgent challenge. This is the joint responsibility of the Member States and the Community Institutions. The ECB cannot and will not finance governments. But when monetary policy channels are not working properly, the Commission believes that it is within the mandate of the ECB to take the necessary actions, for instance in the secondary markets of sovereign debt. Indeed, the ECB has not only the right but also the duty to restore the integrity of monetary policy. It is of course for the ECB, as an independent institution, to determine what actions to carry out and under what conditions. But all actors, and I really mean all actors, should respect the ECB's independence.

Honourable Members,

I have spoken about the economic measures that we must implement as a matter of urgency. This is indispensable. But it is not sufficient. We must go further.

We must complete the economic and monetary union. We must create a banking union and a fiscal union and the corresponding institutional and political mechanisms.

Today, the Commission is presenting legislative proposals for a single European supervisory mechanism. This is the stepping stone to a banking union.

The crisis has shown that while banks became transnational, rules and oversight remained national. And when things went wrong, it was the taxpayers who had to pick up the bill.

Over the past four years the EU has overhauled the rulebook for banks, leading the world in implementing the G20 commitments. But mere coordination is no longer adequate – we need to move to common supervisory decisions, namely within the Euro area.

The single supervisory mechanism proposed today will create a reinforced architecture, with a core role for the European Central Bank, and appropriate articulation with the European Banking Authority, which will restore confidence in the supervision of the banks in the Euro area.

It will be a supervision for all Euro area banks. Supervision must be able to look everywhere because systemic risks can be anywhere, not just in so-called systemically relevant banks. Of course, this in a system that fully engages the national supervisors.

The package comprises two legal texts, one on the ECB and the other on the EBA, which go together. It is clear that this parliament will have a crucial role to play in the adoption of the new mechanism, and after that in its democratic oversight.

This is a crucial first step towards the banking union I proposed before this House in June. Getting the European supervisor in place is the top priority for now, because it is the precondition for the better management of banking crises, from banking resolution to deposit insurance.

In parallel the Commission will continue to work on the reform of the banking sector, to make sure it plays its role in the responsible financing of the real economy. That means improving long term financing for SMEs and other companies. It means rules on reference indices, so we do not again see the manipulation of bank interest rates affecting companies and mortgage holders alike. It means legislation to ensure that banks give a fair deal to consumers and another look at the structure of banking activities to eliminate inherent risks.

In all of this, the role of this Parliament is essential. The Commission endeavours to work in close partnership with you.

But there is a second element of a deeper economic union it is the move towards a fiscal union.

The case for it is clear: the economic decisions of one Member State impact the others. So we need stronger economic policy co-ordination.

We need a stronger and more binding framework for the national decision making for key economic policies, as the only way to prevent imbalances. While much has been done here, for instance through the six-pack and the Country-Specific Recommendations, further steps are crucial to combine specific conditions with specific incentives and to really make the economic and monetary union sustainable.

To deliver lasting results, we need to develop a fully equipped Community economic governance together with a genuine, credible Community fiscal capacity.

We do not need to separate institutions or to create new institutions for that. Quite the contrary: for this to be effective and quick, the best way is to work with and through the existing institutions: The European Commission as the independent European authority, and overseen by the European Parliament as the parliamentary representation at the European level.

And it is in such a framework that over time, steps for genuine mutualisation of debt redemption and debt issuance can take their place.

So economic reform coupled with a genuine economic and monetary union: these are the engines to get our boat moving forward.

The Commission will publish a blueprint for deepening the economic and monetary union still this autumn.

This blueprint will be presented to this House. Because these questions must be discussed with and by the representatives of the people

At the same time, it will inform the debate at the December European Council that will be prepared by the report that the President of the European Council, myself and the Presidents of the European Central Bank and the Eurogroup have been asked to present.

Our blueprint will identify the tools and instruments, and present options for legal drafting that would give effect to them, from policy coordination to fiscal capacity to debt redemption. And, where necessary – as in the case of jointly and severally guaranteed public debt – it would identify the treaty changes necessary, because some of these changes require modifications to the Treaty. It will present a blue-print for what we need to accomplish not only in the next few weeks and months, but in the next years.

Mr President,

Honourable Members,

b) Political union:

Ultimately, the credibility and sustainability of the Economic and Monetary Union depends on the institutions and the political construct behind it.

This is why the Economic and Monetary Union raises the question of a political union and the European democracy that must underpin it.

If we want economic and monetary union to succeed, we need to combine ambition and proper sequencing. We need to take concrete steps now, with a political union as a horizon.

I would like to see the development of a European public space, where European issues are discussed and debated from a European standpoint. We cannot continue trying to solve European problems just with national solutions.

This debate has to take place in our societies and among our citizens. But, today, I would like to make an appeal also to European thinkers. To men and women of culture, to join this debate on the future of Europe. And I make this appeal to you. This is the house of European democracy. We must strengthen the role of the European Parliament at the European level.

And we need to promote a genuine complementarity and cooperation between the European and national parliaments.

This also cannot be done without strengthening European political parties. Indeed, we have very often a real disconnect between political parties in the capitals and the European political parties here in Strasbourg. This is why we have to recognise the political debate is cast all too often as if it were just between national parties. Even in the European elections we do not see the name of the European political parties on the ballot box, we see a national debate between national political parties. This is why we need a reinforced statute for European political parties. I am proud to announce that the Commission has adopted a proposal for this today.

An important means to deepen the pan-European political debate would be the presentation by European political parties of their candidate for the post of Commission President at the European Parliament elections already in 2014. This can be done without Treaty change. This would be a decisive step to make the possibility of a European choice offered by these elections even clearer. I call on the political parties to commit to this step and thus to further Europeanise these European elections.

Mr President,

Honourable Members,

A true political European Union means we must concentrate European action on the real issues that matter and must be dealt with at the European level. Let's be frank about this not everything can be at the same time a priority. Here, some self-criticism can probably be applied

Proper integration is about taking a fresh look at where is the most appropriate level of action. Subsidiarity is an essential democratic concept and should be practiced.

A political union also means that we must strengthen the foundations on which our Union is built: the respect for our fundamental values, for the rule of law and democracy.

In recent months we have seen threats to the legal and democratic fabric in some of our European states. The European Parliament and the Commission were the first to raise the alarm and played the decisive role in seeing these worrying developments brought into check.

But these situations also revealed limits of our institutional arrangements. We need a better developed set of instruments– not just the alternative between the "soft power" of political persuasion and the "nuclear option" of article 7 of the Treaty.

Our commitment to upholding the rule of law is also behind our intention to establish a European Public Prosecutor's Office, as foreseen by the Treaties. We will come with a proposal soon.

Mr President,

Honourable Members,

A political union also means doing more to fulfil our global role. Sharing sovereignty in Europe means being more sovereign in a global world.

In today's world, size matters.

And values make the difference.

That is why Europe's message must be one of freedom, democracy, of rule of law and of solidarity. In short, our values European values.

More than ever our citizens and the new world order need an active and influential Europe. This is not just for us, for the rest of the world it is important that we succeed. A Europe that stands by its values. And a Europe that stands up for its belief that human rights are not a luxury for the developed world, they should be seen as universal values

The appalling situation in Syria reminds us that we can not afford to be by-standers. A new and democratic Syria must emerge. We have a joint responsibility to make this happen. And to work with those in the global order who need to give also their co-operation to this goal

The world needs an EU that keeps its leadership at the forefront of development and humanitarian assistance. That stands by open economies and fights protectionism. That leads the fight against climate change.

The world needs a Europe that is capable of deploying military missions to help stabilize the situation in crisis areas. We need to launch a comprehensive review of European capabilities and begin truly collective defense planning. Yes, we need to reinforce our Common Foreign and Security Policy and a common approach to defense matters because together we have the power, and the scale to shape the world into a fairer, rules based and human rights' abiding place.

Mr President,

Honourable Members

4. Treaty change, 17/27 dimension and expanding public debate

a) Federation of nation states - Treaty change

A deep and genuine economic and monetary union, a political union, with a coherent foreign and defence policy, means ultimately that the present European Union must evolve.

Let's not be afraid of the words: we will need to move towards a federation of nation states. This is what we need. This is our political horizon.

This is what must guide our work in the years to come.

Today, I call for a federation of nation states. Not a superstate. A democratic federation of nation states that can tackle our common problems, through the sharing of sovereignty in a way that each country and each citizen are better equipped to control their own destiny. This is about the Union with the Member States, not against the Member States. In the age of globalisation pooled sovereignty means more power, not less.

And, I said it on purpose a federation of nation states because in these turbulent times these times of anxiety, we should not leave the defence of the nation just to the nationalists and populists. I believe in a Europe where people are proud of their nations but also proud to be European and proud of our European values.

Creating this federation of nation states will ultimately require a new Treaty.

I do not say this lightly. We are all aware how difficult treaty change has become.

It has to be well prepared.

Discussions on treaty change must not distract or delay us from doing what can and must be done already today.

A deep and genuine economic and monetary union can be started under the current Treaties, but can only be completed with changes in the treaties So let's start it now but let's have the horizon for the future present in our decisions of today.

We must not begin with treaty change. We must identify the policies we need and the instruments to implement them. Only then can we decide on the tools that we lack and the ways to remedy this.

And then there must be a broad debate all over Europe. A debate that must take place before a convention and an IGC is called. A debate of a truly European dimension.

The times of European integration by implicit consent of citizens are over. Europe can not be technocratic, bureaucratic or even diplomatic. Europe has to be ever more democratic. The role of the European parliament is essential. This is why the European elections of 2014 can be so decisive.

Before the next European Parliament elections in 2014, the Commission will present its outline for the shape of the future European Union. And we will put forward explicit ideas for Treaty change in time for a debate.

We will set out the objectives to be pursued, the way the institutions that can make the European Union more open and democratic, the powers and instruments to make it more effective, and the model to make it a union for the peoples of Europe. I believe we need a real debate and in a democracy the best way to debate is precisely in elections at the European level on our future and our goals;

b) 17/27 dimension

Mr President, Honourable Memberss

This is not just a debate for the Euro area in its present membership.

While deeper integration is indispensable for the Euro area and its members, this project should remain open to all Member States.

Let me be very clear: in Europe, we need no more walls dividing us!. Because the European Union is stronger as a whole in keeping the integrity of its single market, its membership and in its institutions.

No one will be forced to come along. And no one will be forced to stay outThe speed will not be dictated by the slowest or the most reluctant

This is why our proposals will be based on the existing Union and its institutions, On the Community method. Let's be clear – there is only one European Union. One Commission. One European Parliament. More democracy, more transparency, more accountability, is not created by a proliferation of institutions that would render the EU more complicated, more difficult to read less coherent and less capable to act.

c) Expanding public debate:

This is honourable members the magnitude of the decisions that we will need to make over time.

That's why I believe we need a serious discussion between the citizens of Europe about the way forward.

About the possible consequences of fragmentation. Because what can happen some times is to have, through unintended consequences, to have fragmentation when we do not want it.

About what we could achieve if leaders avoid national provincialism what we can achieve together.

We must use the 2014 election to mobilise all pro-European forces. We must not allow the populists and the nationalists to set a negative agenda. I expect all those who call themselves Europeans to stand up and to take the initiative in the debate. Because even more dangerous than the scepticism of the anti-Europeans, is the indifference or the pessimism of the pro-Europeans.

Mr President,

Honourable Members,

5. Conclusion: is this realistic?

To sum up, what we need is a decisive deal to complete the EMU, based on a political commitment to a stronger European Union.

The sequence I put before you today is clear.

We should start by doing all we can to stabilise the euro area and accelerate growth in the EU as a whole. The Commission will present all the necessary proposals and we have started today with the single supervisor to create a banking union, in line with the current Treaty provisions.

Secondly, we will present our blueprint on a deep and genuine economic and monetary union, including the political instruments, and this will be done still this autumn

We will present here again all proposals in line with the current Treaty provisions.

And thirdly, where we cannot move forward under the existing treaties, we will present explicit proposals for the necessary Treaty changes ahead of the next European Parliamentary election in 2014, including elements for reinforced democracy and accountability

This is our project. A project which is step by step but with a big ambition for the future with a Federation as our horizon for Europe.

Many will say that this is too ambitious, that it is not realistic.

But let me ask you - is it realistic to go on like we have been doing? Is it realistic to see what we are seeing today in many European countries? Is it realistic to see taxpayers paying banks and afterwards being forced to give banks back the houses they have paid for because they can not pay their mortgages? Is it realistic to see more than 50\% of our young people without jobs in some of our Member States? Is it realistic to go on trying to muddle through and just to accumulate mistakes with unconvincing responses? Is it realistic to think that we can win the confidence of the markets when we show so little confidence in each other?

To me, it is this reality that is not realistic. This reality cannot go on.

The realistic way forward is the way that makes us stronger and more united. Realism is to put our ambition at the level of our challenges. We can do it! Let's send our young people a message of hope. If there is a bias, let it be a bias for hope. We should be proud to be Europeans. Proud of our rich and diverse culture. In spite of our current problems, our societies are among the most human and free in the world.

We do not have to apologise for our democracy our social market economy and for our values. With high levels of social cohesion. Respect for human rights and human dignity. Equality between men and women and respect for our environment. These European societies, with all its problems, are among the most decent societies in human history and I think we should be proud of that. In our countries two or three girls do not go to prison because they sing and criticise the ruler of their country. In our countries people are free and are proud of that freedom and people understand what it means to have that freedom. In many of our countries, namely the most recent Member States, there is a recent memory of what was dictatorship and totalitarianism.

So Previous generations have overcome bigger challenges. Now it is for this generation to show they are up to the task.

Now is the moment for all pro-Europeans to leave business as usual behind and to embrace the business of the future. The European Union was built to guarantee peace. Today, this means making our Union fit to meet the challenges of globalization.

That is why we need a new thinking for Europe, a decisive deal for Europe. That is why we need to guide ourselves by the values that are at the heart of the European Union. Europe I believe has a soul. This soul can give us the strength and the determination to do what we must do.

You can count on the European Commission. I count on you, the European Parliament. Together, as Community institutions we will build a better, stronger and a more united Europe, a citizens' Union for the future of Europe but also the future of the world.

Thank you for your attention.
 \newpage\section{Speech 4 - Barroso - 2013-09-11}
\url{https://ec.europa.eu/commission/presscorner/detail/en/SPEECH_13_684}\\[3mm]
Mr. President,

Presidency of the Council,

Honourable Members,

Ladies and gentlemen,

In 8 months' time, voters across Europe will judge what we have achieved together in the last 5 years.

In these 5 years, Europe has been more present in the lives of citizens than ever before. Europe has been discussed in the coffee houses and popular talk shows all over our continent.

Today, I want to look at what we have done together. At what we have yet to do. And I want to present what I believe are the main ideas for a truly European political debate ahead of next year's elections.

Honourable Members,

As we speak, exactly 5 years ago, the United States government took over Fannie Mae and Freddie Mac, bailed out AIG, and Lehman Brothers filed for bankruptcy protection.

These events triggered the global financial crisis. It evolved into an unprecedented economic crisis. And it became a social crisis with dramatic consequences for many of our citizens. These events have aggravated the debt problem that still distresses our governments. They have led to an alarming increase in unemployment, especially amongst young people. And they are still holding back our households and our companies.

But Europe has fought back. In those 5 years, we have given a determined response. We suffered the crisis together. We realised we had to fight it together. And we did, and we are doing it.

If we look back and think about what we have done together to unite Europe throughout the crisis, I think it is fair to say that we would never have thought all of this possible 5 years ago.

We are fundamentally reforming the financial sector so that people's savings are safe.

We have improved the way governments work together, how they return to sound public finances and modernise their economies.

We have mobilised over 700 billion euro to pull crisis-struck countries back from the brink, the biggest effort ever in stabilisation between countries.

I still vividly remember my meeting last year with chief economists of many of our leading banks. Most of them were expecting Greece to leave the euro. All of them feared the disintegration of the euro area. Now, we can give a clear reply to those fears: no one has left or has been forced to leave the euro. This year, the European Union enlarged from 27 to 28 member states. Next year the euro area will grow from 17 to 18.

What matters now is what we make of this progress. Do we talk it up, or talk it down? Do we draw confidence from it to pursue what we have started, or do we belittle the results of our efforts?

Honourable members,

I just came back from the G20 in Saint Petersburg. I can tell you: this year, contrary to recent years, we Europeans did not receive any lessons from other parts of the world on how to address the crisis. We received appreciation and encouragement.

Not because the crisis is over, because it is not over. The resilience of our Union will continue to be tested. But what we are doing creates the confidence that we are overcoming the crisis – provided we are not complacent.

We are tackling our challenges together.

We have to tackle them together.

In our world of geo-economic and geopolitical tectonic changes, I believe that only together, as the European Union, we can give our citizens what they aspire: that our values, our interests, our prosperity are protected and promoted in the age of globalisation.

So now is the time to rise above purely national issues and parochial interests and to have real progress for Europe. To bring a truly European perspective to the debate with national constituencies.

Now is the time for all those who care about Europe, whatever their political or ideological position, wherever they come from, to speak up for Europe.

If we ourselves don't do it, we cannot expect others to do it either.

Honourable Members,

We have come a long way since the start of the crisis.

In last year's State of the Union speech, I stated that 'despite all [our] efforts, our responses have not yet convinced citizens, markets or our international partners'.

One year on, the facts tell us that our efforts have started to convince. Overall spreads are coming down. The most vulnerable countries are paying less to borrow. Industrial output is increasing. Market trust is returning. Stock markets are performing well. The business outlook is steadily improving. Consumer confidence is sharply rising.

We see that the countries who are most vulnerable to the crisis and are now doing most to reform their economies, are starting to note positive results.

In Spain, as a signal of the very important reforms and increased competitiveness, exports of goods and services now make up 33\% of GDP, more than ever since the introduction of the euro. Ireland has been able to draw money from capital markets since the summer of 2012, the economy is expected to grow for a third consecutive year in 2013 and Irish manufacturing companies are re-hiring staff.

In Portugal, the external current account, which was structurally negative, is now expected to be broadly balanced, and growth is picking up after many quarters in the red. Greece has completed, just in 3 years, a truly remarkable fiscal adjustment, is regaining competitiveness and is nearing for the first time in decades a primary surplus. And Cyprus, that has started the programme later, is also implementing it as scheduled, which is the pre-condition for a return to growth.

For Europe, recovery is within sight.

Of course, we need to be vigilant. 'One swallow does not make a summer, nor one fine day'. Let us be realistic in the analysis. Let us not overestimate, but let's also not underestimate what has been done. Even one fine quarter doesn't mean we are out of the economic heavy weather. But it does prove we are on the right track. On the basis of the figures and evolutions as we now see them, we have good reason to be confident.

This should push us to keep up our efforts. We owe it to those for whom the recovery is not yet within reach, to those who do not yet profit from positive developments. We owe it to our 26 million unemployed. Especially to the young people who are looking to us to give them hope. Hope and confidence are also part of the economic equation.

Honourable members,

If we are where we are today, it is because we have shown the resolve to adapt both our politics and our policies to the lessons drawn from the crisis.

And when I say 'we', I really mean: 'we': it has really been a joint effort.

At each and every step, you, the European Parliament, you have played a decisive role through one of the most impressive records of legislative work ever. I personally believe this is not sufficiently known by the citizens of Europe, and you deserve more credit and recognition for this.

So let us continue to work together to reform our economies, for growth and jobs, and to adapt our institutional architecture. Only if we do so, we will leave this phase of the crisis behind us as well.

There is a lot we can still deliver together, in this Parliament's and this Commission's mandate.

What we can and must do, first and foremost, let's be concrete is delivering the banking union. It is the first and most urgent phase on the way to deepen our economic and monetary union, as mapped out in the Commission's Blueprint presented last autumn.

The legislative process on the Single Supervisory Mechanism is almost completed. The next step is the ECBs independent valuation of banks assets, before it takes up its supervisory role.

Our attention now must urgently turn to the Single Resolution Mechanism. The Commission's proposal is on the table since July and, together, we must do the necessary to have it adopted still during this term.

It is the way to ensure that taxpayers are no longer the ones in the front line for paying the price of bank failure. It is the way to make progress in decoupling bank from sovereign risk.

It is the way to remedy one of the most alarming and unacceptable results of the crisis: increased fragmentation of Europe's financial sector and credit markets - even an implicit re-nationalisation.

And it is also the way to help restoring normal lending to the economy, notably to SMEs. Because in spite of the accommodating monetary policy, credit is not yet sufficiently flowing to the economy across the euro area. This needs to be addressed resolutely.

Ultimately, this is about one thing: growth, which is necessary to remedy today's most pressing problem: unemployment. The current level of unemployment is economically unsustainable, politically untenable, socially unacceptable. So all of us here in the Commission – and I'm happy to have all my Commissioners today here with me - all of us want to work intensively with you, and with the member states, to deliver as much of our growth agenda as we possibly can, we are mobilizing all instruments, but of course we have to be honest, not all are at European level, some are at national level. I want to focus on implementation of the decisions on youth employment and financing of the real economy. We need to avoid a jobless recovery.

Europe therefore must speed up the pace of structural reforms. Our Country Specific Recommendations set out what the member states must do in this respect.

At EU level - because there is what can be done at national level and what can be done at European level -, the focus should be on what matters most for the real economy: exploiting the full potential of the single market comes first.

We have a well-functioning single market for goods, and we see the economic benefits of that. We need to extend the same formula to other areas: mobility, communications, energy, finance and e-commerce, to name but a few. We have to remove the obstacles that hold back dynamic companies and people. We have to complete connecting Europe.

I'd like to announce that, today, we will formally adopt a proposal that gives a push towards a single market for telecoms. Citizens know that Europe has dramatically brought down their costs for roaming. Our proposal will strengthen guarantees and lower prices for consumers, and present new opportunities for companies. We know that in the future, trade will be more and more digital. Isn't it a paradox that we have an internal market for goods but when it comes to digital market we have 28 national markets? How can we grab all the opportunities of the future that are opened by the digital economy if we don't conclude this internal market?

The same logic applies to the broader digital agenda: it solves real problems and improves daily life for citizens. The strength of Europe's future industrial base depends on how well people and businesses are interconnected. And by properly combining the digital agenda with data protection and the defence of privacy, our European model strengthens the trust of the citizens. Both with respect to internal and external developments, adopting the proposed legislation on data protection is of utmost importance to the European Commission.

The single market is a key lever for competitiveness and employment. Adopting all remaining proposals under the Single Market Act I and II, and implementing the Connecting Europe Facility in the next few months, we lay the foundations for prosperity in the years to come.

We are also adapting to a dynamic transformation on a global scale, so we must encourage this innovative dynamism at a European scale. That is why we must also invest more in innovation, in technology and the role of science. I have great faith in science, in the capacity of the human mind and a creative society to solve its problems. The world is changing dramatically. And I believe many of the solutions are going to come, in Europe and outside Europe, from new science studies, from new technologies. And I would like Europe to be leading that effort globally. This is why we - Parliament and Commission - have made such a priority of Horizon 2020 in the discussions on the EU budget.

That is why we use the EU budget to invest in skills, education and vocational training, dynamising and supporting talent. That is why we have pushed for Erasmus Plus.

And that is why, later this autumn, we will make further proposals for an industrial policy fit for the 21st century. Why we mobilize support for SMEs because we believe a strong dynamic industrial base is indispensable for a strong European economy.

And whilst fighting climate change, our 20-20-20 goals have set our economy on the path to green growth and resource efficiency, reducing costs and creating jobs.

By the end of this year, we will come out with concrete proposals for our energy and climate framework up to 2030. And we will continue to shape the international agenda by fleshing out a comprehensive, legally binding global climate agreement by 2015, with our partners. Europe alone cannot do all the fight for climate change. Frankly, we need the others also on board. At the same time, we will pursue our work on the impact of energy prices on competitiveness and on social cohesion.

All these drivers for growth are part of our 'Europe 2020' agenda, and fully and swiftly implementing it is more urgent than ever. In certain cases, we need to go beyond the 2020 agenda.

This means we must also pursue our active and assertive trade agenda. It is about linking us closer to growing third markets and guaranteeing our place in the global supply chain. Contrary to perception, where most of our citizens think we are losing in global trade, we have a significant and increasing trade surplus of more than 300 billion euro a year, goods, services, and agriculture. We need to build on that. This too will demand our full attention in the months to come, notably with the Transatlantic Trade and Investment Partnership with the US and the negotiations with Canada and Japan.

And last but not least, we need to step up our game in implementing the Multiannual Financial Framework, the European budget. The EU budget is the most concrete lever we have at hand to boost investments. In some of our regions, the European Union budget is the only way to get public investment because they don't have the sources at national level.

Both the European Parliament and the Commission wanted more resources. We have been in that fight together. But even so, one single year's EU budget represents more money - in today's prices - than the whole Marshall plan in its time! Let us now make sure that the programmes can start on the 1st of January 2014. That the results are being felt on the ground. And that we use the possibilities of innovative financing, from instruments that have already started, to EIB money, to project bonds.

We have to make good on the commitment we have made in July. From the Commission's side, we will deliver. We will, for example, present the second amending budget for 2013 still this month. There is no time to waste, so I warn against holding it up. In particular, I urge member states not to delay.

I cannot emphasise this enough: citizens will not be convinced with rhetoric and promises only, but only with a concrete set of common achievements. We have to show the many areas where Europe has solved problems for citizens. Europe is not the cause of problems, Europe is part of the solution.

I address what we have to do still more extensively in today's letter to the President of the European Parliament, which you will also have received. I will not go now in detail regarding the programme for next year.

My point today is clear: together, there is a lot still to achieve before the elections. It is not the time to thrown in the towel, it is time to roll up our sleeves.

Honourable Members,

None of this is easy. These are challenging times, a real stress test for the EU. The path of permanent and profound reform is as demanding as it is unavoidable. Let's make no mistake: there is no way back to business as usual. Some people believe that after this everything will come back as it was before. They are wrong, This crisis is different. This is not a cyclical crisis, but a structural one. We will not come back to the old normal. We have to shape a new normal. We are in a transformative period of history. We have to understand that, and not just say it. But we have to draw all the consequences from that, including in our state of mind, and how we react to the problems.

We see from the first results that it is possible.

And we all know from experience that it is necessary.

At this point in time, with a fragile recovery, the biggest downside risk I see is political: lack of stability and lack of determination. Over the last years we have seen that anything that casts doubt on governments' commitment to reform is instantly punished. On the positive side, strong and convincing decisions have an important and immediate impact.

In this phase of the crisis, governments' job is to provide the certainty and predictability that markets still lack.

Surely, you all know Justus Lipsius. Justus Lipsius is the name of the Council building in Brussels. Justus Lipsius was a very influential 16th century humanist scholar, who wrote a very important book called De Constantia.

He wrote, 'Constancy is a right and immovable strength of the mind, neither lifted up nor pressed down with external or casual accidents.' Only a 'strength of the mind', he argued, based on 'judgment and sound reason', can help you through confusing and alarming times.

I hope that in these times, these difficult times, all of us, including the governments' representatives that meet at the Justus Lipsius building, show that determination, that perseverance, when it comes to the implementation of the decisions taken. Because one of the issues that we have is to be coherent, not just take decisions, but afterwards be able to implement them on the ground.

Honourable members,

It is only natural that, over the last few years, our efforts to overcome the economic crisis have overshadowed everything else.

But our idea of Europe needs to go far beyond the economy. We are much more than a market. The European ideal touches the very foundations of European society. It is about values, and I underline this word: values. It is based on a firm belief in political, social and economic standards, grounded in our social market economy.

In today's world, the EU level is indispensable to protect these values and standards and promote citizens' rights: from consumer protection to labour rights, from women's rights to respect for minorities, from environmental standards to data protection and privacy.

Whether defending our interests in international trade, securing our energy provision, or restoring people's sense of fairness by fighting tax fraud and tax evasion: only by acting as a Union do we pull our weight at the world stage.

Whether seeking impact for the development and humanitarian aid we give to developing countries, managing our common external borders or seeking to develop in Europe a strong security and defense policy: only by integrating more can we really reach our objectives.

There is no doubt about it. Our internal coherence and international relevance are inextricably linked. Our economic attraction and political traction are fundamentally entwined.

Does anyone seriously believe that, if the euro had collapsed, we or our Member States would still have any credibility left internationally?

Does everyone still realise how enlargement has been a success in terms of healing history's deep scars, establishing democracies where no one had thought it possible? How neighbourhood policy was and still is the best way to provide security and prosperity in regions of vital importance for Europe? Where would we be without all of this?

Today, countries like Ukraine are more than ever seeking closer ties to the European Union, attracted by our economic and social model. We cannot turn our back on them. We cannot accept any attempts to limit these countries own sovereign choices. Free will and free consent need to be respected. These are also the principles that lie at the basis of our Eastern Partnership, which we want to take forward at our summit in Vilnius.

And does everyone still remember just how much Europe has suffered from its wars during the last century, and how European integration was the valid answer?

Next year, it will be one century after the start of the First World War. A war that tore Europe apart, from Sarajevo to the Somme. We must never take peace for granted. We need to recall that it is because of Europe that former enemies now sit around the same table and work together. It is only because they were offered a European perspective that now even Serbia and Kosovo come to an agreement, under mediation of the EU.

Last year's Nobel Peace Prize reminded us of that historic achievement: that Europe is a project of peace.

We should be more aware of it ourselves. Sometimes I think we should not be ashamed to be proud. Not arrogant. But more proud. We should look towards the future, but with a wisdom we gained from the past.

Let me say this to all those who rejoice in Europe's difficulties and who want to roll back our integration and go back to isolation: the pre-integrated Europe of the divisions, the war, the trenches, is not what people desire and deserve. The European continent has never in its history known such a long period of peace as since the creation of the European Community. It is our duty to preserve it and deepen it.

Honourable members,

It is precisely with our values that we address the unbearable situation in Syria, which has tested, over the last months, the world's conscience so severely. The European Union has led the international aid response by mobilising close to 1.5 billion euros, of which \euro850 million comes directly from the EU budget. The Commission will do its utmost to help the Syrian people and refugees in neighbouring countries.

We have recently witnessed events we thought had long been eradicated. The use of chemical weapons is a horrendous act that deserves a clear condemnation and a strong answer. The international community, with the UN at its centre, carries a collective responsibility to sanction these acts and to put an end to this conflict. The proposal to put Syria's chemical weapons beyond use is potentially a positive development. The Syrian regime must now demonstrate that it will implement this without any delay. In Europe, we believe that, ultimately, only a political solution stands a chance of delivering the lasting peace that the Syrian people deserve.

Honourable members,

There are those who claim that a weaker Europe would make their country stronger, that Europe is a burden; that they would be better off without it.

My reply is clear: we all need a Europe that is united, strong and open.

In the debate that is ongoing all across Europe, the bottom-line question is: Do we want to improve Europe, or give it up?

My answer is clear: let's engage!

If you don't like Europe as it is: improve it!

Find ways to make it stronger, internally and internationally, and you will have in me the firmest of supporters. Find ways that allow for diversity without creating discriminations, and I will be with you all the way.

But don't turn away from it.

I recognize: as any human endeavor, the EU is not perfect.

For example, controversies about the division of labour between the national and European levels will never be conclusively ended.

I value subsidiarity highly. For me, subsidiarity is not a technical concept. It is a fundamental democratic principle. An ever closer union among the citizens of Europe demands that decisions are taken as openly as possible and as closely to the people as possible.

Not everything needs a solution at European level. Europe must focus on where it can add most value. Where this is not the case, it should not meddle. The EU needs to be big on big things and smaller on smaller things - something we may occasionally have neglected in the past. The EU needs to show it has the capacity to set both positive and negative priorities. As all governments, we need to take extra care of the quality and quantity of our regulation knowing that, as Montesquieu said, 'les lois inutiles affaiblissent les lois nécessaires'. ['Useless laws weaken the necessary ones'.]

But there are, honourable members, areas of major importance where Europe must have more integration, more unity. Where only a strong Europe can deliver results.

I believe a political union needs to be our political horizon, as I stressed in last year's State of the Union. This is not just the demand of a passionate European. This is the indispensable way forward to consolidate our progress and ensure the future.

Ultimately, the solidity of our policies, namely of the economic and monetary union, depend on the credibility of the political and institutional construct that supports it.

So we have mapped out, in the Commission Blueprint for a deep and genuine Economic and Monetary Union, not only the economic and monetary features, but also the necessities, possibilities and limits of deepening our institutional set-up in the medium and long term. The Commission will continue to work for the implementation of its Blueprint, step by step, one phase after the other.

And I confirm, as announced last year, the intention to present, before the European elections, further ideas on the future of our Union and how best to consolidate and deepen the community method and community approach in the longer term. That way, they can be subject to a real European debate. They will set out the principles and orientations that are necessary for a true political union.

Honourable Members,

We can only meet the challenges of our time if we strengthen the consensus on fundamental objectives.

Politically, we must not be divided by differences between the euro area and those outside it, between the centre and the periphery, between the North and the South, between East and West. The European Union must remain a project for all members, a community of equals.

Economically, Europe has always been a way to close gaps between countries, regions and people. And that must remain so. We cannot do member states' work for them. The responsibility remains theirs. But we can and must complement it with European responsibility and European solidarity.

For that reason, strengthening the social dimension is a priority for the months to come, together with our social partners. The Commission will come with its communication on the social dimension of the economic and monetary union on the 2nd of October. Solidarity is a key element of what being part of Europe is all about, and something to take pride in.

Safeguarding its values, such as the rule of law, is what the European Union was made to do, from its inception to the latest chapters in enlargement.

In last year's State of the Union speech, at a moment of challenges to the rule of law in our own member states, I addressed the need to make a bridge between political persuasion and targeted infringement procedures on the one hand, and what I call the nuclear option of Article 7 of the Treaty, namely suspension of a member states' rights.

Experience has confirmed the usefulness of the Commission role as an independent and objective referee. We should consolidate this experience through a more general framework. It should be based on the principle of equality between member states, activated only in situations where there is a serious, systemic risk to the rule of law, and triggered by pre-defined benchmarks.

The Commission will come forward with a communication on this. I believe it is a debate that is key to our idea of Europe.

This does not mean that national sovereignty or democracy are constrained. But we do need a robust European mechanism to influence the equation when basic common principles are at stake.

There are certain non-negotiable values that the EU and its member states must and shall always defend.

Honourable Members,

The polarisation that resulted from the crisis poses a risk to us all, to the project, to the European project.

We, legitimate political representatives of the European Union, can turn the tide. You, the democratic representatives of Europe, directly elected, will be at the forefront of the political debate. The question I want to pose is: which picture of Europe will voters be presented with? The candid version, or the cartoon version? The myths or the facts? The honest, reasonable version, or the extremist, populist version? It's an important difference.

I know some people out there will say Europe is to blame for the crisis and the hardship.

But we can remind people that Europe was not at the origin of this crisis. It resulted from mismanagement of public finances by national governments and irresponsible behaviour in financial markets.

We can explain how Europe has worked to fix the crisis. What we would have lost if we hadn't succeeded in upholding the single market, because it was under threat, and the common currency, because some people predicted the end of the euro. If we hadn't coordinated recovery efforts and employment initiatives.

Some people will say that Europe is forcing governments to cut spending.

But we can remind voters that government debt got way out of hand even before the crisis, not because of but despite Europe. We can add that the most vulnerable in our societies, and our children, would end up paying the price if we don't persevere now. And the truth is that countries inside the euro or outside the euro, in Europe or outside Europe, they are making efforts to curb their very burdened public finances.

Some will campaign saying that we have given too much money to vulnerable countries. Others will say we have given too little money to vulnerable countries.

But every one of us can explain what we did and why: there is a direct link between one country's loans and another country's banks, between one country's investments and another country's businesses, between one country's workers and another country's companies. This kind of interdependence means only European solutions work.

What I tell people is: when you are in the same boat, one cannot say: 'your end of the boat is sinking.' We were in the same boat when things went well, and we are in it together when things are difficult.

Some people might campaign saying: Europe has grabbed too much power. Others will claim Europe always does too little, too late. The interesting things is that sometimes we have the same people saying that Europe is not doing enough and at the same time that's not giving more means to Europe to do what Europe has to do.

But we can explain that member states have entrusted Europe with tasks and competences. The European Union is not a foreign power. It is the result of democratic decisions by the European institutions and by member states.

At the same time we must acknowledge that, in some areas, Europe still lacks the power to do what is asked of it. A fact that is all too easily forgotten by those, and there are many out there, who always like to nationalise success and Europeanise failure. Ultimately, what we have, and what we don't have, is the result of democratic decision-making. And I think we should remind people of that.

Ladies and gentlemen,

Mr President,

Honourable members,

I hope the European Parliament will take up this challenge with all the idealism it holds, with as much realism and determination as the times demand of us.

The arguments are there.

The facts are there.

The agenda has been set out.

In 8 months' time, voters will decide.

Now, it's up to us to make the case for Europe.

We can do so by using the next 8 months to conclude as much as we can. We have a lot to do still.

Adopt and implement the European budget, the MFF. This is critical for investment in our regions all over Europe. This is indispensable for the first priority we have: to fight against unemployment, notably youth unemployment.

Advance and implement the banking union. This is critical to address the problem of financing for businesses and SMEs.

These are our clear priorities: employment and growth.

Our job is not finished. It is in its decisive phase.

Because, Honourable Members, the elections will not only be about the European Parliament, nor will they be about the European Commission or about the Council or about this or that personality.

They will be about Europe.

We will be judged together.

So let us work together - for Europe.

With passion and with determination.

Let us not forget: one hundred years ago –Europe was sleepwalking into the catastrophy of the war of 1914.

Next year, in 2014, I hope Europe will be walking out of the crisis towards a Europe that is more united, stronger and open.

Thank you for your attention.
 \newpage\section{Speech 5 - Juncker - 2015-09-11}
\url{https://ec.europa.eu/commission/presscorner/detail/en/SPEECH_15_5614}\\[3mm]
Mr President,

Honourable Members of the European Parliament,

Today is the first time during my mandate as President of the European Commission that I have the honour to address this House on the State of our European Union.

I would therefore like to recall the political importance of this very special institutional moment.

The State of the Union address is foreseen explicitly by the Framework Agreement that governs the relations between the European Parliament and the European Commission. This Agreement provides that “[e]ach year in the first part-session of September, a State of the Union debate will be held in which the President of the Commission shall deliver an address, taking stock of the current year and looking ahead to priorities for the following years. To that end, the President of the Commission will in parallel set out in writing to Parliament the main elements guiding the preparation of the Commission Work Programme for the following year.”

The State of the Union address requires the President of the Commission to take stock of the current situation of our European Union and to set priorities for the work ahead.

And it launches the interinstitutional process leading to a new Work Programme of the European Commission for the year ahead.

Together with Frans Timmermans, my First Vice-President, this morning I sent a letter to the Presidents of both branches of the European legislator: to President Martin Schulz, and to Luxembourg’s Prime Minister Xavier Bettel, who currently holds the rotating Presidency of the Council. This letter sets out in detail the numerous actions the Commission intends to take by means of legislation and other initiatives, from now until the end of 2016. We are proposing an ambitious, focused, and intense legislative agenda that will require Commission, Parliament and Council to work closely and effectively together.

I will not go into the details of our legislative agenda now. We will have a structured dialogue with the Parliament and the Council on this in the weeks to come.

But I feel that today is not the moment to speak about all this.

I am the first President of the Commission whose nomination and election is the direct result of the outcome of the European Parliament elections in May 2014.

Having campaigned as a lead candidate, as Spitzenkandidat, in the run up to the elections, I had the opportunity to be a more political President.

This political role is foreseen by the Treaties, by means of which the Member States made the Commission the promoter of the general interest of the Union. But the crisis years have diminished this understanding.

This is why I said last September before this House that I wanted to lead a political Commission. A very political Commission.

I said this not because I believe we can and should politicise everything.

I said it because I believe the immense challenges Europe is currently facing – both internally and externally – leave us no choice but to address them from a very political perspective, in a very political manner and having the political consequences of our decisions very much in mind.

Recent events have confirmed the urgent need for such a political approach in the European Union.

This is not the time for business as usual.

This is not the time for ticking off lists or checking whether this or that sectorial initiative has found its way into the State of the Union speech.

This is not the time to count how many times the word social, economic or sustainable appears in the State of the Union speech.

Instead, it is time for honesty.

It is time to speak frankly about the big issues facing the European Union.

 

Because our European Union is not in a good state.

 

There is not enough Europe in this Union.

And there is not enough Union in this Union.

 

We have to change this. And we have to change this now.

 

The Refugee Crisis: The Imperative to Act as a Union

Whatever work programmes or legislative agendas say: The first priority today is and must be addressing the refugee crisis.

Since the beginning of the year, nearly 500,000 people have made their way to Europe. The vast majority of them are fleeing from war in Syria, the terror of the Islamic State in Libya or dictatorship in Eritrea. The most affected Member States are Greece, with over 213,000 refugees, Hungary, with over 145,000, and Italy, with over 115,000.

The numbers are impressive. For some they are frightening.

But now is not the time to take fright. It is time for bold, determined and concerted action by the European Union, by its institutions and by all its Member States.

This is first of all a matter of humanity and of human dignity. And for Europe it is also a matter of historical fairness.

We Europeans should remember well that Europe is a continent where nearly everyone has at one time been a refugee. Our common history is marked by millions of Europeans fleeing from religious or political persecution, from war, dictatorship, or oppression.

Huguenots fleeing from France in the 17th century.

Jews, Sinti, Roma and many others fleeing from Germany during the Nazi horror of the 1930s and 1940s.

Spanish republicans fleeing to refugee camps in southern France at the end of the 1930s after their defeat in the Civil War.

Hungarian revolutionaries fleeing to Austria after their uprising against communist rule was oppressed by Soviet tanks in 1956.

Czech and Slovak citizens seeking exile in other European countries after the oppression of the Prague Spring in 1968.

Hundreds and thousands were forced to flee from their homes after the Yugoslav wars.

Have we forgotten that there is a reason there are more McDonalds living in the U.S. than there are in Scotland? That there is a reason the number of O'Neills and Murphys in the U.S. exceeds by far those living in Ireland?

Have we forgotten that 20 million people of Polish ancestry live outside Poland, as a result of political and economic emigration after the many border shifts, forced expulsions and resettlements during Poland’s often painful history?

Have we really forgotten that after the devastation of the Second World War, 60 million people were refugees in Europe? That as a result of this terrible European experience, a global protection regime – the 1951 Geneva Convention on the status of refugees – was established to grant refuge to those who jumped the walls in Europe to escape from war and totalitarian oppression?

We Europeans should know and should never forget why giving refuge and complying with the fundamental right to asylum is so important.

I have said in the past that we are too seldom proud of our European heritage and our European project.

Yet, in spite of our fragility, our self-perceived weaknesses, today it is Europe that is sought as a place of refuge and exile.

It is Europe today that represents a beacon of hope, a haven of stability in the eyes of women and men in the Middle East and in Africa.

That is something to be proud of and not something to fear.

Europe today, in spite of many differences amongst its Member States, is by far the wealthiest and most stable continent in the world.

We have the means to help those fleeing from war, terror and oppression.

I know that many now will want to say that this is all very well, but Europe cannot take everybody.

It is true that Europe cannot house all the misery of the world. But let us be honest and put things into perspective.

There is certainly an important and unprecedented number of refugees coming to Europe at the moment. However, they still represent just 0.11\% of the total EU population. In Lebanon, refugees represent 25\% of the population. And this in a country where people have only one fifth of the wealth we enjoy in the European Union.

Let us also be clear and honest with our often worried citizens: as long as there is war in Syria and terror in Libya, the refugee crisis will not simply go away.

We can build walls, we can build fences. But imagine for a second it were you, your child in your arms, the world you knew torn apart around you. There is no price you would not pay, there is no wall you would not climb, no sea you would not sail, no border you would not cross if it is war or the barbarism of the so-called Islamic State that you are fleeing.

So it is high time to act to manage the refugee crisis. There is no alternative to this.

There has been a lot finger pointing in the past weeks. Member States have accused each other of not doing enough or of doing the wrong thing. And more often than not fingers have been pointed from national capitals towards Brussels.

We could all be angry about this blame-game. But I wonder who that would serve. Being angry does not help anyone. And the attempt of blaming others is often just a sign that politicians are overwhelmed by unexpected events.

Instead, we should rather recall what has been agreed that can help in the current situation. It is time to look at what is on the table and move swiftly forwards.

We are not starting anew. Since the early 2000s, the Commission has persistently tabled legislation after legislation, to build a Common European Asylum System. And the Parliament and the Council have enacted this legislation, piece by piece. The last piece of legislation entered into force just in July 2015.

Across Europe we now have common standards for the way we receive asylum seekers, in respect of their dignity, for the way we process their asylum applications, and we have common criteria which our independent justice systems use to determine whether someone is entitled to international protection.

But these standards need to be implemented and respected in practice. And this is clearly not yet the case, we can see this every day on television. Before the summer, the Commission had to start a first series of 32 infringement proceedings to remind Member States of what they had previously agreed to do. And a second series will follow in the days to come. European laws must be applied by all Member States – this must be self-evident in a Union based on the rule of law.

Common asylum standards are important, but not enough to cope with the current refugee crisis. The Commission, the Parliament and the Council said this in spring. The Commission tabled a comprehensive European Agenda on Migration in May. And it would be dishonest to say that nothing has happened since then.

We tripled our presence at sea. Over 122,000 lives have been saved since then. Every life lost is one too many, but many more have been rescued that would have been lost otherwise – an increase of 250\%. 29 Member States and Schengen Associated countries are participating in the joint operations coordinated by Frontex in Italy, Greece and Hungary. 102 guest officers from 20 countries; 31 ships; 3 helicopters; 4 fixed wing aircrafts; 8 patrol cars, 6 thermo-vision vehicles and 4 transport vehicles – that is a first measure of European solidarity in action, even though more will have to be done.

We have redoubled our efforts to tackle smugglers and dismantle human trafficker groups. Cheap ships are now harder to come by, leading to less people putting their lives in peril in rickety, unseaworthy boats. As a result, the Central Mediterranean route has stabilised at around 115,000 arriving during the month of August, the same as last year. We now need to achieve a similar stabilisation of the Balkans route, which has clearly been neglected by all policy-makers.

The European Union is also the number one donor in the global efforts to alleviate the Syrian refugee crisis. Around \euro4 billion have been mobilised by the European Commission and Member States in humanitarian, development, economic and stabilisation assistance to Syrians in their country and to refugees and their host communities in neighbouring Lebanon, Jordan, Iraq, Turkey and Egypt. Indeed just today we launched two new projects to provide schooling and food security to 240,000 Syrian refugees in Turkey.

We have collectively committed to resettling over 22,000 people from outside of Europe over the next year, showing solidarity with our neighbours. Of course, this remains very modest in comparison to the Herculean efforts undertaken by Turkey, Jordan and Lebanon, who are hosting over 4 million Syrian refugees. I am encouraged that some Member States are showing their willingness to significantly step up our European resettlement efforts. This will allow us very soon to come forward with a structured system to pool European resettlement efforts more systematically.

Where Europe has clearly under-delivered, is on common solidarity with regard to the refugees who have arrived on our territory.

To me, it is clear that the Member States where most refugees first arrive – at the moment, these are Italy, Greece and Hungary – cannot be left alone to cope with this challenge.

This is why the Commission already proposed an emergency mechanism in May, to relocate initially 40,000 people seeking international protection from Italy and Greece.

And this is why today we are proposing a second emergency mechanism to relocate a further 120,000 from Italy, Greece and Hungary.

This requires a strong effort in European solidarity. Before the summer, we did not receive the backing from Member States I had hoped for. But I see that the mood is turning. And I believe it is high time for this.

I call on Member States to adopt the Commission proposals on the emergency relocation of altogether 160,000 refugees at the Extraordinary Council of Interior Ministers on 14 September. We now need immediate action. We cannot leave Italy, Greece and Hungary to fare alone. Just as we would not leave any other EU Member State alone. For if it is Syria and Libya people are fleeing from today, it could just as easily be Ukraine tomorrow.

Europe has made the mistake in the past of distinguishing between Jews, Christians, Muslims. There is no religion, no belief, no philosophy when it comes to refugees.

Do not underestimate the urgency. Do not underestimate our imperative to act. Winter is approaching – think of the families sleeping in parks and railway stations in Budapest, in tents in Traiskirchen, or on shores in Kos. What we will become of them on cold, winter nights?

Of course, relocation alone will not solve the issue. It is true that we also need to separate better those who are in clear need of international protection and are therefore very likely to apply for asylum successfully; and those who are leaving their country for other reasons which do not fall under the right of asylum. This is why today the Commission is proposing a common EU list of safe countries of origin. This list will enable Member States to fast track asylum procedures for nationals of countries that are presumed safe to live in. This presumption of safety must in our view certainly apply to all countries which the European Council unanimously decided meet the basic Copenhagen criteria for EU membership – notably as regards democracy, the rule of law, and fundamental rights. It should also apply to the other potential candidate countries on the Western Balkans, in view of their progress made towards candidate status.

I am of course aware that the list of safe countries is only a procedural simplification. It cannot take away the fundamental right of asylum for asylum seekers from Albania, Bosnia and Herzegovina, the former Yugoslav Republic of Macedonia, Kosovo, Montenegro, Serbia, and Turkey. But it allows national authorities to focus on those refugees which are much more likely to be granted asylum, notably those from Syria. And this focus is very much needed in the current situation.

I also believe that beyond the immediate action needed to address current emergencies, it is time we prepare a more fundamental change in the way we deal with asylum applications – and notably the Dublin system that requires that asylum applications be dealt with by the first country of entry.

We need more Europe in our asylum policy. We need more Union in our refugee policy.

A true European refugee and asylum policy requires solidarity to be permanently anchored in our policy approach and our rules. This is why, today, the Commission is also proposing a permanent relocation mechanism, which will allow us to deal with crisis situations more swiftly in the future.

A common refugee and asylum policy requires further approximation of asylum policies after refugee status is granted. Member States need to take a second look at their support, integration and inclusion policies. The Commission is ready to look into how EU Funds can support these efforts. And I am strongly in favour of allowing asylum seekers to work and earn their own money whilst their applications are being processed.

A united refugee and asylum policy also requires stronger joint efforts to secure our external borders. Fortunately, we have given up border controls between the Member States of the Schengen area, to guarantee free movement of people, a unique symbol of European integration. But the other side of the coin to free movement is that we must work together more closely to manage our external borders. This is what our citizens expect. The Commission said it back in May, and I said it during my election campaign: We need to strengthen Frontex significantly and develop it into a fully operational European border and coast guard system. It is certainly feasible. But it will cost money. The Commission believes this is money well invested. This is why we will propose ambitious steps towards a European Border and Coast Guard before the end of the year.

A truly united, European migration policy also means that we need to look into opening legal channels for migration. Let us be clear: this will not help in addressing the current refugee crisis. But if there are more, safe and controlled roads opened to Europe, we can manage migration better and make the illegal work of human traffickers less attractive. Let us not forget, we are an ageing continent in demographic decline. We will be needing talent. Over time, migration must change from a problem to be tackled to a well-managed resource. To this end, the Commission will come forward with a well-designed legal migration package in early 2016.

A lasting solution will only come if we address the root causes, the reasons why we are currently facing this important refugee crisis. Our European foreign policy must be more assertive. We can no longer afford to be ignorant or disunited with regard to war or instability right in our neighbourhood.

In Libya, the EU and our Member States need to do more to engage with regional partners to make sure a Government of National Accord is in place soon. We should be prepared to help, with all EU instruments available, such a government to deliver security and services to the population as soon as it is in place. Our EU development and humanitarian support will have to be immediate and comprehensive.

I would also like to point out that we are entering the fifth year of the Syrian crisis with no end in sight. So far, the international community has failed the Syrian people. Europe has failed the Syrian people.

Today I call for a European diplomatic offensive to address the crises in Syria and in Libya. We need a stronger Europe when it comes to foreign policy. And I am very glad that Federica Mogherini, our determined High Representative, has prepared the ground for such an initiative with her diplomatic success in the Iran nuclear talks. And that she stands ready to work closely together with our Member States towards peace and stability in Syria and Libya.

To facilitate Federica’s work, today the Commission is proposing to establish an emergency Trust Fund, starting with \euro1.8 billion from our common EU financial means to address the crises in the Sahel and Lake Chad regions, the Horn of Africa, and the North of Africa. We want to help create lasting stability, for instance by creating employment opportunities in local communities, and thereby address the root causes of destabilisation, forced displacement and illegal migration. I expect all EU Member States to pitch in and match our ambitions.

I do not want to create any illusions that the refugee crisis will be over any time soon. It will not. But pushing back boats from piers, setting fire to refugee camps, or turning a blind eye to poor and helpless people: that is not Europe.

Europe is the baker in Kos who gives away his bread to hungry and weary souls. Europe is the students in Munich and in Passau who bring clothes for the new arrivals at the train station. Europe is the policeman in Austria who welcomes exhausted refugees upon crossing the border. This is the Europe I want to live in.

The crisis is stark and the journey is still long. I am counting on you, in this House, and on all Member States to show European courage going forward, in line with our common values and our history.

 

A new start for Greece, for the euro area and for the European economy

Mr President, Honourable Members,

I said I want to talk about the big issues today. This is why this State of the Union speech needs to address the situation in Greece, as well as the broader lessons from the fifth year of Greek crisis the impact of which continues to be felt in the Eurozone and in the European economy and society as a whole.

Since the start of the year, the talks on Greece have tested all our patience. A lot of time and a lot of trust was lost. Bridges were burnt. Words were said that cannot easily be taken back.

We saw political posturing, bickering and insults carelessly bandied about.

Too often, we saw people thinking they can impose their views without a wayward thought for another's point of view.

We saw democracies in the Eurozone being played against each other. The recovery and creation of jobs witnessed last year in Greece vanished during these months.

Collectively, we looked into the abyss.

And it was once more only when we were at the brink that we were able to see the bigger picture and to live up to our responsibilities.

In the end, a deal was reached, commitments were adhered to and implemented. Trust has started to be regained, even though it remains very fragile.

I am not proud of every aspect of the results achieved. However, I am proud of the teams in the European Commission who worked day and night until late in August, relentlessly, to bridge the gap between far-flung positions and to bring about solutions in the interest of Europe and of the Greek people.

I know that not everybody was happy with what the Commission did.

Many Greek politicians were not happy that we insisted on reforms in Greece, notably as regards the unsustainable pension system and the unfair tax regime.

Many other European politicians could not understand why the Commission continued to negotiate. Some could not understand why we did not simply leave all the talks to the technicians of the International Monetary Fund. Why we sometimes also spoke about the social side of programme commitments and amended those to take account of the effects on the most vulnerable in society. Or that I personally dared to say again and again that the euro, and membership in the euro, is meant to be irreversible.

 

Mr President, Honourable Members,

The Commission’s mandate in negotiations with a programme country such as Greece has a very clear basis: it is the Treaty on European Union which calls on the Commission to promote the common interest of the Union and to uphold the law. The same law includes the Treaty clause, agreed by all Member States, that qualifies membership in the euro as irrevocable.

As long as Member States have not amended the Treaties, I believe the Commission and all other EU institutions have a clear mandate and duty to do everything possible to preserve the integrity of the euro area.

The Commission has also been explicitly entrusted by the European Stability Mechanism (ESM) Treaty, ratified by all euro area Member States, with conducting programme negotiations with a Member State. We have to do this in liaison with the European Central Bank and, where possible, together with the International Monetary Fund. But we have a clear mandate to do so.

Where the Treaties talk about the Commission, I read this as meaning the Commission as an institution that is politically led by the President and the College of Commissioners. This is why I did not leave the talks with Greece to the Commission bureaucracy alone, in spite of their great expertise and the hard work they are doing. But I spoke personally to our experts regularly, often several times per day, to orient them or to adjust their work. I also ensured that every week, the situation of the negotiations in Greece was discussed at length and very politically in the meetings of the College.

Because it is not a technical question whether you increase VAT not only on restaurants, but also on processed food. It is a political and social question.

It is not a technical question, but a deeply political question, whether you increase VAT on medicines in a country where 30\% of the population is no longer covered by the public health system as a result of the crisis. Or whether you cut military expenditure instead – in a country that continues to have one of the highest military expenditures in the EU.

It is certainly not a technical question whether you reduce the pensions of the poorest in society or the minimum wage; or if you instead levy a tax on Greek ship owners.

Of course, the figures in what is now the third Greek programme had to add up in the end. But we managed to do this with social fairness in mind. I read the Troika report of the European Parliament very thoroughly. I hope you can see that we have drawn the lessons from this, that we have made, for the first time, a social impact assessment of the programme. Even though I admit frankly that the Commission also had to compromise sometimes in these negotiations.

What matters to me, is that, in the end, a compromise was found which could be agreed by all 19 euro area Member States, including Greece.

After weeks of talks, small progress, repeated setbacks, many crisis moments, and often a good dose of drama, we managed to sign a new Stability Support Programme for Greece on 19 August.

Now that the new programme is in place, I want it to be a new start, for Greece and for the euro area as a whole.

Let us be very honest: We are only at the beginning of a new, long journey.

For Greece, the key now is to implement the deal which was agreed. There has to be broad political ownership for this.

I had the leaders of all the mainstream Greek political groups in my office before the final agreement was concluded. They all promised to support this agreement, and they gave first proof of their commitment when they voted for the new programme and for the first three waves of reforms in the Hellenic Parliament. I expect them to stand by their word and deliver on the agreement – whoever governs. Broad support and timely delivery of the reforms is what Greece needs, so that confidence can return both among the Greek people and to the Greek economy.

The programme is one thing, but it is not enough to put Greece on a path of sustainable growth. The Commission will stand by Greece to make sure the reforms take shape. And we will assist Greece in developing a growth strategy which is Greek owned and Greek led.

From the modernisation of the public administration and the independence of the tax authority, the Commission will provide tailor-made technical assistance, together with the help of European and international partners. This will be the main task of the new Structural Reform Support Service I established in July.

On 15 July, the Commission also put forward a proposal to limit national co-financing in Greece and to frontload funding for investment projects short of liquidity: a \euro35 billion package for growth. This is urgent for recovery after months of financial squeeze. This is money that will reach the Greek real economy, for businesses and authorities to invest and recruit.

The Commission worked day-in, day-out to put this on the table. National Parliaments met several times throughout the month of August. I therefore hope that the European Parliament will also play its part, in line with previous commitments. Our programme for growth in Greece has been on the table of this House for two months. If adopted, it will still take several weeks until the first euro will reach the real economy of Greece.

I call on you to follow the example of the Council, which will agree on this growth programme by the end of this month. The European Parliament should be at least as fast as the Council on this.

 

I said I wanted the new programme to be a new start not just for Greece but for the euro area as a whole, because there are important lessons we need to draw from the crisis that has haunted us for far too long.

The economic and social situation speaks for itself: over 23 million people are still unemployed today in the European Union, with more than half without a job for a year or more. In the euro area alone, more than 17.5 million people are without a job. Our recovery is hampered by global uncertainties. Government debt in the EU has reached more than 88\% of GDP on average, and stands at almost 93\% in the euro area.

The crisis is not over. It has just been put on pause.

This is not to say that nothing is happening. Unemployment figures are improving, GDP is rising at its highest rate for years, and the financing conditions of households and companies have recovered significantly. And several Member States once severely affected – like Latvia, Ireland, Spain and Portugal – which received European financial assistance are now steadily growing and consolidating their economies.

This is progress but recovery is too slow, too fragile and too dependent on our external partners.

More fundamentally, the crisis has left us with very wide differences across the euro area and the EU as a whole. It has damaged our growth potential. It has added to the long-term trend of rising inequalities. All this has fuelled doubts about social progress, the value of change and the merits of belonging together.

What we need is to recreate a process of convergence, both between Member States and within societies, with productivity, job creation and social fairness at its core.

We need more Union in our Europe.

For the European Union, and for my Commission in particular, this means two things: first, investing in Europe's sources of jobs and growth, notably in our Single Market; and secondly, completing our Economic and Monetary Union to creating the conditions for a lasting recovery. We are acting on both fronts.

Together with you and the Member States, we brought to life the \euro315 billion Investment Plan for Europe, with a new European Fund for Strategic Investments (EFSI).

Less than a year after I announced this plan, we are now at a point where some of the first projects are just taking off:

40,000 households all over France will get a lower energy bill and 6,000 jobs will be created, thanks to Investment Fund-financed improved energy efficiency in buildings.

In health clinics in Barcelona, better treatment will be available to patients through new plasma derived therapies, funded by the Investment Fund.

In Limerick and other locations in Ireland, families will have improved access to primary healthcare and social services through fourteen new primary care centres. This is just the beginning, with many more projects like these to follow.

At the same time as we deploy our Investment Plan, we are upgrading our Single Market to create more opportunities for people and business in all 28 Member States. Thanks to Commission projects such as the Digital Single Market, Capital Markets Union and the Energy Union, we are reducing obstacles to activities cross-border and using the scale of our continent to stimulate innovation, connecting talents and offering a wider choice of products and services.

But we will fail in our efforts to prosper if we do not learn a hard lesson: we have not yet convinced the people of Europe and the world that our Union is not just here to survive, but can also thrive and prosper.

Let us not fool ourselves: our collective inability to provide a swift and clear answer to the Greek crisis over the last months weakened us all. It damaged the trust in our single currency and the EU’s reputation in the world.

No wind favours he who has no destined port – we need to know where we are headed.

This is the essence of the report I presented in June with the other Presidents of the European institutions on the completion of our Economic and Monetary Union.

It was self-evident for me to include President Schulz in this important work. After all, the Parliament is the heart of democracy at EU level, just as national Parliaments are the heart of democracy at national level. The European Parliament is and must remain the Parliament of the euro area. And the European Parliament, in its role as co-legislator, will be in charge of deciding on the new initiatives the Commission will propose in the months to come to deepen our Economic and Monetary Union. I am therefore glad that for the first time, we have written not a ‘Four Presidents' Report’, but a ‘Five Presidents’ Report’.

Despite months of late-night discussions to find an agreement for Greece, we wrote this report in May and June to set out the course for a stronger future. The Five Presidents of the leading EU institutions have agreed a roadmap that should allow us to stabilise and consolidate the euro area by early 2017; and then, on the basis of a renewed convergence of our economies, to achieve more fundamental reform and move where we can from crisis resilience to new growth perspectives.

As we had expected, the Five Presidents’ Report has triggered a lively debate across Europe. Some say we need a government of the euro. Others say we need more discipline and respect of the rules. I agree with both: we need collective responsibility, a greater sense of the common good and full respect and implementation of what is collectively agreed. But I do not agree this should mean the multiplication of institutions or putting the euro on auto pilot, as if new institutions or magic rules could deliver more or better.

You cannot run a single currency on the basis of rules and statistics alone. It needs constant political assessment, as the basis of new economic, fiscal and social policy choices.

The Five Presidents' Report includes a full agenda of work for the years to come, and I want us to move swiftly on all fronts – economic, financial, fiscal and political Union. Some efforts will have to be focused on the euro area, while others should be open to all 28 Member States, in view of their close interaction with our Single Market.

Allow me to highlight five domains where the Commission will present ambitious proposals swiftly and where we will be expecting progress already this autumn.

First: the Five Presidents agreed that we need a common system to ensure that citizens' bank savings are always protected up to a limit of \euro100,000 per person and account. This is the missing part of our Banking Union.

Today, such protection schemes exist, but they are all national. What we need is a more European system, disconnected from government purses so that citizens can be absolutely sure that their savings are safe.

We all saw what happened in Greece during the summer: Citizens were – understandably – taking out their savings since they had little trust and confidence in the financial capability of the State to support its banking system. This must change.

A more common deposit guarantee system is urgently needed, and the Commission will present a legislative proposal on the first steps towards this before the end of the year.

I am of course fully aware there is no consensus on this yet. But I also know that many of you are as convinced as I am of the need to move ahead. I say to those who are more sceptical: the Commission is fully aware that there are differences in the starting positions of Member States. Some have developed and well-financed their national systems of deposit insurance. Others are still building up such systems. We need to take these differences into account. This is why the Five Presidents’ Report advocates not full mutualisation, but a new approach by means of a reinsurance system. We will present further details on this in the weeks to come.

Second: we need a stronger representation of the euro on the global scene. How is it possible that the euro area, which has the second largest currency in the world, can still not speak with one voice on economic matters in international financial institutions?

Imagine yourselves in the daily work of the International Monetary Fund for a moment. We know well how important the IMF is. Still, instead of speaking with one voice as the euro area, Belgium and Luxembourg have to agree their voting position with Armenia and Israel; and Spain sits in a joint constituency with Latin American countries.

How can it be that we – Europeans – are jointly major shareholders of global institutions such as the IMF and the World Bank and still end up acting as a minority?

How can it be that a strategically important new Infrastructure Investment Bank is created in Asia, and European governments, instead of coordinating their efforts, engage in a race who is first to become a member?

We need to grow up and put our common interests ahead of our national ones. For me, the President of the Eurogroup should be the natural spokesperson for the euro area in international financial institutions such as the IMF.

Third: we need a more effective and more democratic system of economic and fiscal surveillance. I want this Parliament, national Parliaments, as well as social partners at all levels, to be key actors in the process. I also want the interest of the euro area as a whole to be better reflected upfront in EU and national policies: the interest of the whole is not just the sum of its parts. This will be reflected in our proposals to streamline and strengthen the European Semester of economic policy coordination further.

In the future, I no longer want our recommendations for the economic orientation of the euro area as a whole to be empty words. I want them to provide real orientation, notably on the fiscal stance of the euro area.

Fourth: we need to enhance fairness in our taxation policies. This requires greater transparency and equity, for citizens and companies. We presented an Action Plan in June, the gist of which is the following: the country where a company generates its profits must also be the country of taxation.

One step towards this goal is our work on a Common Consolidated Corporate Tax Base. This simplification will make tax avoidance more difficult.

We are also working hard with the Council to conclude an agreement on the automatic exchange of information on tax rulings by the end of the year.

At the same time, we expect our investigations into the different national schemes to yield results very soon.

And we are fighting hard to get Member States to adopt the modalities of a Financial Transaction Tax by the end of the year.

We need more Europe, we need more Union, and we need more fairness in our taxation policy.

Fifth: We have to step up the work for a fair and truly pan-European labour market. Fairness in this context means promoting and safeguarding the free movement of citizens as a fundamental right of our Union, while avoiding cases of abuses and risks of social dumping.

Labour mobility is welcome and needed to make the euro area and the single market prosper. But labour mobility should be based on clear rules and principles. The key principle should be that we ensure the same pay for the same job at the same place.

As part of these efforts, I will want to develop a European pillar of social rights, which takes account of the changing realities of Europe's societies and the world of work. And which can serve as a compass for the renewed convergence within the euro area.

This European pillar of social rights should complement what we have already jointly achieved when it comes to the protection of workers in the EU. I will expect social partners to play a central role in this process. I believe we do well to start with this initiative within the euro area, while allowing other EU Member States to join in if they want to do so.

As said in the Five Presidents’ Report, we will also need to look ahead at more fundamental steps with regard to the euro area. The Commission will present a White Paper on this in spring 2017.

Yes, we will need to set up a Euro Area Treasury over time, which is accountable at European level. And I believe it should be built on the European Stability Mechanism we created during the crisis, which has, with a potential credit volume of \euro500 billion, a firepower that is as important as the one of the IMF. The ESM should progressively assume a broader macroeconomic stabilisation function to better deal with shocks that cannot be managed at the national level alone. We will prepare the ground for this to happen in the second half of this mandate.

The European Union is a dynamic project. A project to serve its people. There are no winners or losers. We all get back more than we put in. It is one, comprehensive project. This is also a message for our partners in the United Kingdom, which I have very much in my mind when thinking about the big political challenges of the months to come.

 

A fair deal for Britain

Since I took office, things have become clearer as regards the United Kingdom: before the end of 2017, there will be a referendum on whether Britain remains in the Union or not. This will of course be a decision for voters in the United Kingdom. But it would not be honest nor realistic to say that this decision will not be of strategic importance for the Union as a whole.

I have always said that I want the UK to stay in the European Union. And that I want to work together with the British government on a fair deal for Britain.

The British are asking fundamental questions to and of the EU. Whether the EU delivers prosperity for its citizens. Whether the action of the EU concentrates on areas where it can deliver results. Whether the EU is open to the rest of the world.

These are questions to which the EU has answers, and not just for the sake of the UK. All 28 EU Member States want the EU to be modern and focused for the benefit of all its citizens. We all agree that the EU must adapt and change in view of the major challenges and crisis we are facing at the moment.

This is why we are completing the Single Market, slashing red tape, improving the investment climate for small businesses.

This is why we are creating a Digital Single Market – to make it such that your location in the EU makes no difference to the price you pay when you book a car online. We are modernising the EU's copyright rules – to increase people's access to cultural content online while ensuring that authors get a fair remuneration. And just two months ago, the EU agreed to abolish roaming charges as of summer 2017, a move many tourists and travellers, notably from Britain, have been calling for, for years.

This is why we are negotiating trade agreements with leading nations such as the Transatlantic Trade and Investment Partnership. This is why we are opening markets and breaking down barriers for businesses and workers in all 28 EU Member States.

It is my very personal commitment to improve the way in which the Union works with national Parliaments. I have inscribed a duty to interact more closely with national Parliaments in the mission letters of all Members of my Commission. I am convinced that strengthening our relationship with national Parliaments will bring the Union closer to the people that it serves. This is an ambition that I know Prime Minister David Cameron also shares. I am confident that we will be able to find a common answer.

Over a year ago, when I campaigned to become President of the Commission, I made a vow that, as President, I would seek a fair deal for Britain. A deal that is fair for Britain. And that is also fair for the 27 other Member States.

I want to ensure we preserve the integrity of all four freedoms of the Single Market and at the same time find ways to allow the further integration of the Eurozone to strengthen the Economic and Monetary Union.

To be fair to the UK, part of this deal will be to recognise the reality that not all Member States participate in all areas of EU policy. Special Protocols define the position of the UK, for instance in relation to the euro and to Justice and Home Affairs. To be fair to the other Member States, the UK's choices must not prevent them from further integration where they see fit.

I will seek a fair deal for Britain. I will do this for one reason and one reason alone: because I believe that the EU is better with Britain in it and that Britain is better within the EU.

In key areas, we can achieve much more by acting collectively, than we could each on our own. This is in particular the case for the tremendous foreign policy challenges Europe is currently facing and which I will address in the next part of this speech.

 

United alongside Ukraine

Europe is a small part of the world. If we have something to offer, it is our knowledge and leadership.

Around a century ago, one in five of the world’s population were in Europe; today that figure is one in nine; in another century it will be one in twenty-five.

I believe we can, and should, play our part on the world stage; not for our own vanity, but because we have something to offer. We can show the world the strength that comes from uniting and the strategic interest in acting together. There has never been a more urgent and compelling time to do so.

We have more than 40 active conflicts in the world at the moment. While these conflicts rage, whilst families are broken and homes reduced to rubble, I cannot come to you, almost 60 years after the birth of the European Union and pitch you peace. For the world is not at peace.

If we want to promote a more peaceful world, we will need more Europe and more Union in our foreign policy. This is most urgent towards Ukraine.

The challenge of helping Ukraine to survive, to reform and to prosper is a European one. Ultimately, the Ukrainian dream, the dream of the Maidan is European: to live in a modern country, in a stable economy, in a sound and fair political system.

Over the past twelve months, I have got to know President Poroshenko well, at a Summit, over dinner at his home, during many meetings and countless phone calls. He has begun a transformation of his country. He is fighting for peace. He deserves our support.

We have already done a lot, lending \euro3.41 billion in three Macro-Financial Assistance programmes, helping to broker a deal that will secure Ukraine's winter gas supplies and advising on the reform of the judiciary. The EU and all its Member States must contribute if we are to succeed.

We will also need to maintain our unity.

We need unity when it comes to the security of our Eastern Member States, notably the Baltics. The security and the borders of EU Member States are untouchable. I want this to be understood very clearly in Moscow.

We need more unity when it comes to sanctions. The sanctions the EU has imposed on Russia have a cost for each of our economies, and repercussions on important sectors, like farming. But sanctions are a powerful tool in confronting aggression and violation of international law. They are a policy that needs to be kept in place until the Minsk Agreements are complied with in full. We will have to keep our nerve and our unity.

But we must also continue to look for solutions.

I spoke to President Putin in Brisbane at the G20, in a bilateral meeting that went on into the early hours of the morning. We recalled how long we have known each other, how different times had become. A spirit of cooperation between the EU and Russia has given way to suspicion and distrust.

The EU must show Russia the cost of confrontation but it must also make clear it is prepared to engage.

I do not want a Europe that stands on the sidelines of history. I want a Europe that leads. When the European Union stands united, we can change the world.

 

United in Leadership in Addressing Climate Change

One example of where Europe is already leading is in our action on climate change.

In Europe we all know that climate change is a major global challenge – and we have known for a while now.

The planet we share – its atmosphere and stable climate – cannot cope with the use mankind is making of it.

Some parts of the world have been living beyond their means, creating carbon debt and living on it. As we know from economics and crisis management, living beyond our means is not sustainable behaviour.

Nature will foot us the bill soon enough. In some parts of the world, climate change is changing the sources of conflict – the control over a dam or a lake can be more strategic than an oil refinery.

Climate change is even one the root causes of a new migration phenomenon. Climate refugees will become a new challenge – if we do not act swiftly.

The world will meet in Paris in 90 days to agree on action to meet the target of keeping the global temperature rise below 2 degrees Celsius. The EU is on track and made a clear pledge back in March: a binding, economy-wide emissions reduction target of at least 40\% by 2030, compared to 1990 levels. This is the most ambitious contribution presented to date.

Others are following, some only reluctantly.

Let me be very clear to our international partners: the EU will not sign just any deal. My priority, Europe's priority, is to adopt an ambitious, robust and binding global climate deal.

This is why my Commission and I have been spending part of this first year in drumming support for ambition in Paris. Last May I was in Tokyo where I challenged Prime Minister Abe to work with us in ensuring that Paris is a worthy successor of Kyoto.

In June at the G7 summit, leaders agreed to develop long-term low-carbon strategies and abandon fossil fuels by the end of the century.

Later I met Chinese Premier Li Keqiang to prepare Paris and to launch a partnership to ensure that cities of today are designed to meet the energy and climate needs of tomorrow.

And, in coordination with the High Representative, the members of the College have been engaged in climate diplomacy efforts. Today Commissioner Arias Cañete is in Papua New Guinea discussing the plans for Paris with the leaders of the Pacific Islands Forum. If corrective action is not taken to tackle climate change, the tide will rise and those islands will be the proverbial canary in the coalmine.

However, if Paris delivers, humanity will, for the first time, have an international regime to efficiently combat climate change.

Paris will be the next stop but not the last stop. There is a Road to Paris; but there is also a Road from Paris.

My Commission will work to ensure Europe keeps leading in the fight against climate change. We will practice what we preach.

We have no silver bullet to tackle climate change. But our laws, such as the EU Emissions Trading Scheme, and our actions have allowed us to decrease carbon emissions whilst keeping the economy growing.

Our forward-looking climate policy is also delivering on our much needed Energy Union goals: it is making us a world leader in the renewable energy sector, which today employs over one million people across the EU and generates \euro130 billion turnover, including \euro35 billion worth of exports. European companies today hold 40\% of all patents for renewable technologies and the pace of technological change increases the potential for new global trade in green technology.

This is why a strategic focus on innovation and on interconnecting our markets is being given in the implementation of the Energy Union.

This is what I promised you last year and this is what this Commission has delivered and will continue to deliver.

The fight against climate change will not be won or lost in diplomatic discussions in Brussels or in Paris. It will be won or lost on the ground and in the cities where most Europeans live, work and use about 80\% of all the energy produced in Europe.

That is why I have asked President Schulz to host the Covenant of the Mayors meeting in the Parliament next month, bringing together more than 5,000 European mayors. They have all pledged to meet the EU CO2 reduction objective. I hope that all members of this House will lend their support to the action that communities and localities across Europe are taking to making Paris and its follow up a success.

 

Conclusion

Mr President, Honourable Members,

There were many things I did not and could not mention today. For example, I would have liked to talk to you about Cyprus and my hope, my ambition and my wish to see the island united next year. After I met for a long talk with Presidents Nikos Anastasiades and Mustafa Akinci in the middle of the Green Line in July, I am confident that, with the necessary vision and political will from the two leaders, this is feasible under the current conditions and with continued good coordination between UN and EU efforts. I will offer all my support and assistance to help achieve this objective. Because I believe that walls and fences have no place in an EU Member State.

I have not spoken about Europe's farmers who were protesting this week in Brussels. I agree with them that there is something wrong in a market when the price of a litre of milk is less than the price of a litre of water. But I do not believe that we can or should start micromanaging the milk market from Brussels. We should compensate the farmers who are suffering from the effects of sanctions against Russia. And this is why the Commission is putting a \euro500 million solidarity package for farmers on the table. And European and national competition authorities should take a close look into the structure of the market. Something has turned sour in the milk market. My impression is that we need to break some retail oligopolies.

There is much more to be said but in touching upon the main issues, the main challenges confronting us today, for me there is one thing that becomes clear: whether it is the refugee crisis we are talking about, the economy or foreign policy: we can only succeed as a Union.

Who is the Union that represents Europe's 507 million citizens? The Union is not just Brussels or Strasbourg. The Union is the European Institutions. The Union is also the Member States. It is national governments and national Parliaments.

It is enough if just one of us fails to deliver for all of us to stumble.

Europe and our Union have to deliver. While I am a strong defender of the Community method in normal times, I am not a purist in crisis times – I do not mind how we cope with a crisis, be it by intergovernmental solutions or community-led processes. As long as we find a solution and get things done in the interest of Europe's citizens.

However, when we see the weaknesses of a method, we have to change our approach.

Look at the relocation mechanism for refugees we put on the table for Greece and Italy in May: the Commission proposed a binding, communitarian solidarity scheme. Member States opted instead for a voluntary approach. The result: the 40,000 figure was never reached. Not a single person in need of protection has been relocated yet and Italy and Greece continue to cope alone. This is simply not good enough.

Look at intergovernmental solutions like the 2011 Fiscal Compact to strengthen fiscal discipline or the 2014 Agreement setting up a common bank resolution fund. Today, we see that not a single Member State has completely implemented the Fiscal Compact. And only 4 out of 19 Member States have ratified the agreement on the bank resolution fund, which is meant to be launched on 1 January 2016.

This is simply not good enough if we want to cope with the present, immense challenges.

We have to change our way of working.

We have to be faster.

We have to be more European in our method.

Not because we want power at European level. But because we need urgently better and swifter results.

We need more Europe in our Union.

We need more Union in our Union.

All my life, I have believed in Europe. I have my reasons, many of which I know and am relieved are not relatable to generations today.

Upon taking office, I said I want to rebuild bridges that had started to crumble. Where solidarity had started to fray at the seams. Where old daemons sought to resurface.

We still have a long way to go.

But when, generations from now, people read about this moment in Europe's history books, let it read that we stood together in demonstrating compassion and opened our homes to those in need of our protection.

That we joined forces in addressing global challenges, protecting our values and resolving conflicts.

That we made sure taxpayers never again have to pay for the greed of financial speculators.

That hand in hand we secured growth and prosperity for our economies, for our businesses, and above all for our children.

Let it read that we forged a Union stronger than ever before.

Let it read that together we made European history. A story our grandchildren will tell with pride.
 \newpage\section{Speech 6 - Juncker - 2016-09-14}
\url{https://ec.europa.eu/commission/presscorner/detail/en/speech_16_3043}\\[3mm]
Mr President,

Honourable Members of the European Parliament,

I stood here a year ago and I told you that the State of our Union was not good. I told you that there is not enough Europe in this Union. And that there is not enough Union in this Union.

I am not going to stand here today and tell you that everything is now fine.

It is not.

Let us all be very honest in our diagnosis.

Our European Union is, at least in part, in an existential crisis.

Over the summer, I listened carefully to Members of this Parliament, to government representatives, to many national Parliamentarians and to the ordinary Europeans who shared their thoughts with me.

I have witnessed several decades of EU integration. There were many strong moments. Of course, there were many difficult times too, and times of crisis.

But never before have I seen such little common ground between our Member States. So few areas where they agree to work together.

Never before have I heard so many leaders speak only of their domestic problems, with Europe mentioned only in passing, if at all.

Never before have I seen representatives of the EU institutions setting very different priorities, sometimes in direct opposition to national governments and national Parliaments. It is as if there is almost no intersection between the EU and its national capitals anymore.

Never before have I seen national governments so weakened by the forces of populism and paralysed by the risk of defeat in the next elections.

Never before have I seen so much fragmentation, and so little commonality in our Union.

We now have a very important choice to make.

Do we give in to a very natural feeling of frustration? Do we allow ourselves to become collectively depressed? Do we want to let our Union unravel before our eyes?

Or do we say: Is this not the time to pull ourselves together? Is this not the time to roll up our sleeves and double, triple our efforts? Is this not the time when Europe needs more determined leadership than ever, rather than politicians abandoning ship?

Our reflections on the State of the Union must start with a sense of realism and with great honesty.

First of all, we should admit that we have many unresolved problems in Europe. There can be no doubt about this.

From high unemployment and social inequality, to mountains of public debt, to the huge challenge of integrating refugees, to the very real threats to our security at home and abroad – every one of Europe's Member States has been affected by the continuing crises of our times.

We are even faced with the unhappy prospect of a member leaving our ranks.

Secondly, we should be aware that the world is watching us.

I just came back from the G20 meeting in China. Europe occupies 7 chairs at the table of this important global gathering. Despite our big presence, there were more questions than we had common answers to.

Will Europe still be able to conclude trade deals and shape economic, social and environmental standards for the world?

Will Europe's economy finally recover or be stuck in low growth and low inflation for the next decade?

Will Europe still be a world leader when it comes to the fight for human rights and fundamental values?

Will Europe speak up, with one voice, when territorial integrity is under threat, in violation of international law?

Or will Europe disappear from the international scene and leave it to others to shape the world?

I know that you here in this House would be only too willing to give clear answers to these questions. But we need our words to be followed by joint action. Otherwise, they will be just that: words. And with words alone, you cannot shape international affairs.

Thirdly, we should recognise that we cannot solve all our problems with one more speech. Or with one more summit.

This is not the United States of America, where the President gives a State of the Union speech to both Houses of Congress, and millions of citizens follow his every word, live on television.

In comparison to this, our State of the Union moment here in Europe shows very visibly the incomplete nature of our Union. I am speaking today in front of the European Parliament. And separately, on Friday, I will meet with the national leaders in Bratislava.

So my speech can not only compete for your applause, ignoring what national leaders will say on Friday. I also cannot go to Bratislava with a different message than I have for you. I have to take into account both levels of democracy of our Union, which are both equally important.

We are not the United States of Europe. Our European Union is much more complex. And ignoring this complexity would be a mistake that would lead us to the wrong solutions.

Europe can only work if speeches supporting our common project are not only delivered in this honourable House, but also in the Parliaments of all our Member States.

Europe can only work if we all work for unity and commonality, and forget the rivalry between competences and institutions. Only then will Europe be more than the sum of its parts. And only then can Europe be stronger and better than it is today. Only then will leaders of the EU institutions and national governments be able to regain the trust of Europe's citizens in our common project.

Because Europeans are tired of the endless disputes, quarrels and bickering.

Europeans want concrete solutions to the very pertinent problem that our Union is facing. And they want more than promises, resolutions and summit conclusions. They have heard and seen these too often.

Europeans want common decisions followed by swift and efficient implementation.

Yes, we need a vision for the long term. And the Commission will set out such a vision for the future in a White Paper in March 2017, in time for the 60th anniversary of the Treaties of Rome. We will address how to strengthen and reform our Economic and Monetary Union. And we will also take into account the political and democratic challenges our Union of 27 will be facing in the future. And of course, the European Parliament will be closely involved in this process, as will national Parliaments.

But a vision alone will not suffice. What our citizens need much more is that someone governs. That someone responds to the challenges of our time.

Europe is a cord of many strands – it only works when we are all pulling in the same direction: EU institutions, national governments and national Parliaments alike. And we have to show again that this is possible, in a selected number of areas where common solutions are most urgent.

I am therefore proposing a positive agenda of concrete European actions for the next twelve months.

Because I believe the next twelve months are decisive if we want to reunite our Union. If we want to overcome the tragic divisions between East and West which have opened up in recent months. If we want to show that we can be fast and decisive on the things that really matter. If we want to show to the world that Europe is still a force capable of joint action.

We have to get to work.

I sent a letter with this message to President Schulz and Prime Minister Fico this morning.

The next twelve months are the crucial time to deliver a better Europe:

a Europe that protects;

a Europe that preserves the European way of life;

a Europe that empowers our citizens,

a Europe that defends at home and abroad; and

a Europe that takes responsibility.

 

A EUROPE THAT PRESERVES OUR WAY OF LIFE

I am convinced the European way of life is something worth preserving.

I have the impression that many seem to have forgotten what being European means.

What it means to be part of this Union of Europeans – what it is the farmer in Lithuania has in common with the single mother in Zagreb, the nurse in Valetta or the student in Maastricht.

To remember why Europe's nations chose to work together.

To remember why crowds celebrated solidarity in the streets of Warsaw on 1 May 2004.

To remember why the European flag waved proudly in Puerta del Sol on 1 January 1986.

To remember that Europe is a driving force that can help bring about the unification of Cyprus – something I am supporting the two leaders of Cyprus in.

Above all, Europe means peace. It is no coincidence that the longest period of peace in written history in Europe started with the formation of the European Communities.

70 years of lasting peace in Europe. In a world with 40 active armed conflicts, which claim the lives of 170,000 people every year.

Of course we still have our differences. Yes, we often have controversy. Sometimes we fight. But we fight with words. And we settle our conflicts around the table, not in trenches.

An integral part of our European way of life is our values.

The values of freedom, democracy, the rule of law. Values fought for on battlefields and soapboxes over centuries.

We Europeans can never accept Polish workers being harassed, beaten up or even murdered on the streets of Harlow. The free movement of workers is as much a common European value as our fight against discrimination and racism.

We Europeans stand firmly against the death penalty. Because we believe in and respect the value of human life.

We Europeans also believe in independent, effective justice systems. Independent courts keep governments, companies and people in check. Effective justice systems support economic growth and defend fundamental rights. That is why Europe promotes and defends the rule of law.

Being European also means being open and trading with our neighbours, instead of going to war with them. It means being the world's biggest trading bloc, with trade agreements in place or under negotiation with over 140 partners across the globe.

And trade means jobs – for every \euro1 billion we get in exports, 14,000 extra jobs are created across the EU. And more than 30 million jobs, 1 in 7 of all jobs in the EU, now depend on exports to the rest of the world.

That is why Europe is working to open up markets with Canada – one of our closest partners and one which shares our interests, our values, our respect for the rule of law and our understanding of cultural diversity. The EU-Canada trade agreement is the best and most progressive deal the EU has ever negotiated. And I will work with you and with all Member States to see this agreement ratified as soon as possible.

Being European means the right to have your personal data protected by strong, European laws. Because Europeans do not like drones overhead recording their every move, or companies stockpiling their every mouse click. This is why Parliament, Council and Commission agreed in May this year a common European Data Protection Regulation. This is a strong European law that applies to companies wherever they are based and whenever they are processing your data. Because in Europe, privacy matters. This is a question of human dignity.

Being European also means a fair playing field.

This means that workers should get the same pay for the same work in the same place. This is a question of social justice. And this is why the Commission stands behind our proposal on the Posting of Workers Directive. The internal market is not a place where Eastern European workers can be exploited or subjected to lower social standards. Europe is not the Wild West, but a social market economy.

A fair playing field also means that in Europe, consumers are protected against cartels and abuses by powerful companies. And that every company, no matter how big or small, has to pay its taxes where it makes its profits. This goes for giants like Apple too, even if their market value is higher than the GDP of 165 countries in the world. In Europe we do not accept powerful companies getting illegal backroom deals on their taxes.

The level of taxation in a country like Ireland is not our issue. Ireland has the sovereign right to set the tax level wherever it wants. But it is not right that one company can evade taxes that could have gone to Irish families and businesses, hospitals and schools. The Commission watches over this fairness. This is the social side of competition law. And this is what Europe stands for.

Being European also means a culture that protects our workers and our industries in an increasingly globalised world. Like the thousands who risk losing their jobs in Gosselies in Belgium – it is thanks to EU legislation that the company will now need to engage in a true social dialogue. And workers and local authorities can count on European solidarity and the help of EU funds.

Being European also means standing up for our steel industry. We already have 37 anti-dumping and anti-subsidy measures in place to protect our steel industry from unfair competition. But we need to do more, as overproduction in some parts of the world is putting European producers out of business. This is why I was in China twice this year to address the issue of overcapacity. This is also why the Commission has proposed to change the lesser duty rule. The United States imposes a 265\% import tariff on Chinese steel, but here in Europe, some governments have for years insisted we reduce tariffs on Chinese steel. I call on all Member States and on this Parliament to support the Commission in strengthening our trade defence instruments. We should not be naïve free traders, but be able to respond as forcefully to dumping as the United States.

A strong part of our European way of life that I want to preserve is our agricultural sector. The Commission will always stand by our farmers, particularly when they go through difficult moments as is the case today. Last year, the dairy sector was hit with a ban imposed by Russia. This is why the Commission mobilised \euro1 billion in support of milk farmers to help them get back on their feet. Because I will not accept that milk is cheaper than water.

Being European, for most of us, also means the euro. During the global financial crisis, the euro stayed strong and protected us from even worse instability. The euro is a leading world currency, and it brings huge, often invisible economic benefits. Euro area countries saved \euro50 billion this year in interest payments, thanks to the European Central Bank's monetary policy. \euro50 billion extra that our finance ministers can and should invest into the economy.

Mario Draghi is preserving the stability of our currency. And he is making a stronger contribution to jobs and growth than many of our Member States.

Yes, we Europeans suffered under a historic financial and debt crisis. But the truth is that while public deficits stood at 6.3\% on average in the euro area in 2009, today they are below 2\%.

Over the last three years, almost 8 million more people found a job. 1 million in Spain alone, a country which continues to show an impressive recovery from the crisis.

I wish all this was recalled more often – everywhere in Europe where elected politicians take the floor.

Because in our incomplete Union, there is no European leadership that can substitute national leadership.

European nations have to defend the rationale for unity. No one can do it for them.

They can.

We can be united even though we are diverse.

The great, democratic nations of Europe must not bend to the winds of populism.

Europe must not cower in the face of terrorism.

No – Member States must build a Europe that protects. And we, the European institutions, must help them deliver this promise.

 

A EUROPE THAT EMPOWERS

The European Union should not only preserve our European way of life but empower those living it.

We need to work for a Europe that empowers our citizens and our economy. And today, both have gone digital.

Digital technologies and digital communications are permeating every aspect of life.

All they require is access to high-speed internet. We need to be connected. Our economy needs it. People need it.

And we have to invest in that connectivity now.

That is why today, the Commission is proposing a reform for our European telecommunications markets. We want to create a new legal framework that attracts and enables investments in connectivity.

Businesses should be able to plan their investments in Europe for the next 20 years. Because if we invest in new networks and services, that is at least 1.3 million new jobs over the next decade.

Connectivity should benefit everyone.

That is why today the Commission is proposing to fully deploy 5G, the fifth generation of mobile communication systems, across the European Union by 2025. This has the potential to create a further two million jobs in the EU.

Everyone benefiting from connectivity means that it should not matter where you live or how much you earn.

So we propose today to equip every European village and every city with free wireless internet access around the main centres of public life by 2020.

As the world goes digital, we also have to empower our artists and creators and protect their works.Artists and creators are our crown jewels. The creation of content is not a hobby. It is a profession. And it is part of our European culture.

I want journalists, publishers and authors to be paid fairly for their work, whether it is made in studios or living rooms, whether it is disseminated offline or online, whether it is published via a copying machine or hyperlinked on the web.

The overhaul of Europe's copyright rules we are proposing today does exactly that.

Empowering our economy means investing not just in connectivity, but in job creation.

That is why Europe must invest strongly in its youth, in its jobseekers, in its start-ups.

The \euro315 billion Investment Plan for Europe, which we agreed together here in this House just twelve months ago, has already raised \euro116 billion in investments – from Latvia to Luxembourg – in its first year of operation.

Over 200,000 small firms and start-ups across Europe got loans. And over 100,000 people got new jobs. Thanks to the new European Fund for Strategic Investments I proposed, my Commission developed, and you here in the European Parliament supported and adopted in record time.

And now we will take it further. Today, we propose to double the duration of the Fund and double its financial capacity.

With your support, we will make sure that our European Investment Fund will provide a total of at least \euro500 billion – half a trillion – of investments by 2020. And we will work beyond that to reach \euro630 billion by 2022. Of course, with Member States contributing, we can get there even faster.

Alongside these efforts to attract private investment, we also need to create the right environment to invest in.

European banks are in much better shape than two years ago, thanks to our joint European efforts. Europe needs its banks. But an economy almost entirely dependent on bank credit is bad for financial stability. It is also bad for business, as we saw during the financial crisis. That is why it is now urgent we accelerate our work on the Capital Markets Union. The Commission is putting a concrete roadmap for this on your table today.

A Capital Markets Union will make our financial system more resilient. It will give companies easier and more diversified access to finance. Imagine a Finnish start-up that cannot get a bank loan. Right now, the options are very limited. The Capital Markets Union will offer alternative, vital sources of funding to help start-ups get started – business angels, venture capital, market financing.

To just mention one example – almost a year ago we put a proposal on the table that will make it easier for banks to provide loans. It has the potential of freeing up \euro100 billion of additional finance for EU businesses. So let us please speed up its adoption.

Our European Investment Plan worked better than anyone expected inside Europe, and now we are going to take it global. Something many of you and many Member States have called for.

Today we are launching an ambitious Investment Plan for Africa and the Neighbourhood which has the potential to raise \euro44 billion in investments. It can go up to \euro88 billion if Member States pitch in.

The logic is the same that worked well for the internal Investment Plan: we will be using public funding as a guarantee to attract public and private investment to create real jobs.

This will complement our development aid and help address one of the root causes of migration. With economic growth in developing countries at its lowest level since 2003, this is crucial. The new Plan will offer lifelines for those who would otherwise be pushed to take dangerous journeys in search of a better life.

As much as we invest in improving conditions abroad, we also need to invest in responding to humanitarian crises back home. And, more than anything, we need to invest in our young people.

I cannot and will not accept that Europe is and remains the continent of youth unemployment.

I cannot and will not accept that the millennials, Generation Y, might be the first generation in 70 years to be poorer than their parents.

Of course, this is mainly a task of national governments. But the European Union can support their efforts. We are doing this with the EU Youth Guarantee that was launched three years ago. My Commission enhanced the effectiveness and sped up delivery of the Youth Guarantee. More than 9 million young people have already benefitted from this programme. That is 9 million young people who got a job, traineeship or apprenticeship because of the EU. And we will continue to roll out the Youth Guarantee across Europe, improving the skillset of Europeans and reaching out to the regions and young people most in need.

The European Union can also contribute by helping create more opportunities for young people.

There are many young, socially-minded people in Europe willing to make a meaningful contribution to society and help show solidarity.

Solidarity is the glue that keeps our Union together.

The word solidarity appears 16 times in the Treaties which all our Member States agreed and ratified.

Our European budget is living proof of financial solidarity.

There is impressive solidarity when it comes to jointly applying European sanctions when Russia violates international law.

The euro is an expression of solidarity.

Our development policy is a strong external sign of solidarity.

And when it comes to managing the refugee crisis, we have started to see solidarity. I am convinced much more solidarity is needed. But I also know that solidarity must be given voluntarily. It must come from the heart. It cannot be forced.

We often show solidarity most readily when faced with emergencies.

When the Portuguese hills were burning, Italian planes doused the flames.

When floods cut off the power in Romania, Swedish generators turned the lights back on.

When thousands of refugees arrived on Greek shores, Slovakian tents provided shelter.

In the same spirit, the Commission is proposing today to set up a European Solidarity Corps. Young people across the EU will be able to volunteer their help where it is needed most, to respond to crisis situations, like the refugee crisis or the recent earthquakes in Italy.

I want this European Solidarity Corps up and running by the end of the year. And by 2020, to see the first 100,000 young Europeans taking part.

By voluntarily joining the European Solidarity Corps, these young people will be able to develop their skills and get not only work but also invaluable human experience.

 

A EUROPE THAT DEFENDS

A Europe that protects is a Europe that defends – at home and abroad.

We must defend ourselves against terrorism.

Since the Madrid bombing of 2004, there have been more than 30 terrorist attacks in Europe – 14 in the last year alone. More than 600 innocent people died in cities like Paris, Brussels, Nice, or Ansbach.

Just as we have stood shoulder to shoulder in grief, so must we stand united in our response.

The barbaric acts of the past year have shown us again what we are fighting for – the European way of life. In face of the worst of humanity we have to stay true to our values, to ourselves. And what we are is democratic societies, plural societies, open and tolerant.

But that tolerance cannot come at the price of our security.

That is why my Commission has prioritised security from day one – we criminalised terrorism and foreign fighters across the EU, we cracked down on the use of firearms and on terrorist financing, we worked with internet companies to get terrorist propaganda offline and we fought radicalisation in Europe's schools and prisons.

But there is more to be done.

We need to know who is crossing our borders.

That is why we will defend our borders with the new European Border and Coast Guard, which is now being formalised by Parliament and Council, just nine months after the Commission proposed it. Frontex already has over 600 agents on the ground at the borders with Turkey in Greece and over 100 in Bulgaria. Now, the EU institutions and the Member States should work very closely together to quickly help set up the new Agency. I want to see at least 200 extra border guards and 50 extra vehicles deployed at the Bulgarian external borders as of October.

We will defend our borders, as well, with strict controls, adopted by the end of the year, on everyone crossing them. Every time someone enters or exits the EU, there will be a record of when, where and why.

By November, we will propose a European Travel Information System – an automated system to determine who will be allowed to travel to Europe. This way we will know who is travelling to Europe before they even get here.

And we all need that information. How many times have we heard stories over the last months that the information existed in one database in one country, but it never found its way to the authority in another that could have made the difference?

Border security also means that information and intelligence exchange must be prioritised. For this, we will reinforce Europol – our European agency supporting national law enforcement – by giving it better access to databases and more resources. A counter terrorism unit that currently has a staff of 60 cannot provide the necessary 24/7 support.

A Europe that protects also defends our interests beyond our borders.

The facts are plain: The world is getting bigger. And we are getting smaller.

Today we Europeans make up 8\% of the world population – we will only represent 5\% in 2050. By then you would not see a single EU country among the top world economies. But the EU together? We would still be topping the charts.

Our enemies would like us to fragment.

Our competitors would benefit from our division.

Only together are we and will we remain a force to be reckoned with.

Still, even though Europe is proud to be a soft power of global importance, we must not be naïve. Soft power is not enough in our increasingly dangerous neighbourhood.

Take the brutal fight over Syria. Its consequences for Europe are immediate. Attacks in our cities by terrorists trained in Daesh camps. But where is the Union, where are its Member States, in negotiations towards a settlement?

Federica Mogherini, our High Representative and my Vice-President, is doing a fantastic job. But she needs to become our European Foreign Minister via whom all diplomatic services, of big and small countries alike, pool their forces to achieve leverage in international negotiations. This is why I call today for a European Strategy for Syria. Federica should have a seat at the table when the future of Syria is being discussed. So that Europe can help rebuild a peaceful Syrian nation and a pluralistic, tolerant civil society in Syria.

Europe needs to toughen up. Nowhere is this truer than in our defence policy.

Europe can no longer afford to piggy-back on the military might of others or let France alone defend its honour in Mali.

We have to take responsibility for protecting our interests and the European way of life.

Over the last decade, we have engaged in over 30 civilian and military EU missions from Africa to Afghanistan. But without a permanent structure we cannot act effectively. Urgent operations are delayed. We have separate headquarters for parallel missions, even when they happen in the same country or city. It is time we had a single headquarters for these operations.

We should also move towards common military assets, in some cases owned by the EU. And, of course, in full complementarity with NATO.

The business case is clear. The lack of cooperation in defence matters costs Europe between \euro25 billion and \euro100 billion per year, depending on the areas concerned. We could use that money for so much more.

It can be done. We are building a multinational fleet of air tankers. Let's replicate this example.

For European defence to be strong, the European defence industry needs to innovate. That is why we will propose before the end of the year a European Defence Fund, to turbo boost research and innovation.

The Lisbon Treaty enables those Member States who wish, to pool their defence capabilities in the form of a permanent structured cooperation. I think the time to make use of this possibility is now. And I hope that our meeting at 27 in Bratislava a few days from now will be the first, political step in that direction.

Because it is only by working together that Europe will be able to defend itself at home and abroad.

 

A EUROPE THAT TAKES RESPONSIBILITY

The last point I want to make is about responsibility. About taking responsibility for building this Europe that protects.

I call on all EU institutions and on all of our Member States to take responsibility.

We have to stop with the same old story that success is national, and failure European. Or our common project will not survive.

We need to remember the sense of purpose of our Union. I therefore call on each of the 27 leaders making their way to Bratislava to think of three reasons why we need the European Union. Three things they are willing to take responsibility for defending. And that they are willing to deliver swiftly afterwards.

Slow delivery on promises made is a phenomenon that more and more risks undermining the Union's credibility. Take the Paris agreement. We Europeans are the world leaders on climate action. It was Europe that brokered the first-ever legally binding, global climate deal. It was Europe that built the coalition of ambition that made agreement in Paris possible. But Europe is now struggling to show the way and be amongst the first to ratify our agreement. Only France, Austria and Hungary have ratified it so far.

I call on all Member States and on this Parliament to do your part in the next weeks, not months. We should be faster. Let's get the Paris agreement ratified now. It can be done. It is a question of political will. And it is about Europe's global influence.

The European institutions too, have to take responsibility.

I have asked each of my Commissioners to be ready to discuss, in the next two weeks, the State of our Union in the national Parliaments of the countries they each know best. Since the beginning of my mandate, my Commissioners have made over 350 visits to national Parliaments. And I want them to do this even more now. Because Europe can only be built with the Member States, never against them.

We also have to take responsibility in recognising when some decisions are not for us to take. It is not right that when EU countries cannot decide among themselves whether or not to ban the use of glyphosate in herbicides, the Commission is forced by Parliament and Council to take a decision.

So we will change those rules – because that is not democracy.

The Commission has to take responsibility by being political, and not technocratic.

A political Commission is one that listens to the European Parliament, listens to all Member States, and listens to the people.

And it is us listening that motivated my Commission to withdraw 100 proposals in our first two years of office, to present 80\% fewer initiatives than over the past 5 years and to launch a thorough review of all existing legislation. Because only by focusing on where Europe can provide real added value and deliver results, we will be able to make Europe a better, more trusted place.

Being political also means correcting technocratic mistakes immediately when they happen. The Commission, the Parliament and the Council have jointly decided to abolish mobile roaming charges. This is a promise we will deliver. Not just for business travellers who go abroad for two days. Not only for the holiday maker who spends two weeks in the sun. But for our cross-border workers. And for the millions of Erasmus students who spend their studies abroad for one or two semesters. I have therefore withdrawn a draft that a well-meaning official designed over the summer. The draft was not technically wrong. But it missed the point of what was promised. And you will see a new, better draft as of next week. When you roam, it should be like at home.

Being political is also what allows us to implement the Stability and Growth Pact with common sense. The Pact's creation was influenced by theory. Its application has become a doctrine for many. And today, the Pact is a dogma for some. In theory, a single decimal point over 60 percent in a country's debt should be punished. But in reality, you have to look at the reasons for debt. We should try to support and not punish ongoing reform efforts. For this we need responsible politicians. And we will continue to apply the Pact not in a dogmatic manner, but with common sense and with the flexibility that we wisely built into the rules.

Finally, taking responsibility also means holding ourselves accountable to voters. That is why we will propose to change the absurd rule that Commissioners have to step down from their functions when they want to run in European elections. The German Chancellor, the Czech, Danish or Estonian, Prime Minister do not stop doing their jobs when they run for re-election. Neither should Commissioners. If we want a Commission that responds to the needs of the real world, we should encourage Commissioners to seek the necessary rendez-vous with democracy. And not prevent this.

 

CONCLUSION

Honourable Members,

I am as young as the European project that turns 60 next years in March 2017.

I have lived it, worked for it, my whole life.

My father believed in Europe because he believed in stability, workers' rights and social progress.

Because he understood all too well that peace in Europe was precious – and fragile.

I believe in Europe because my father taught me those same values.

But what are we teaching our children now? What will they inherit from us? A Union that unravels in disunity? A Union that has forgotten its past and has no vision for the future?

Our children deserve better.

They deserve a Europe that preserves their way of life.

They deserve a Europe that empowers and defends them.

They deserve a Europe that protects.

It is time we – the institutions, the governments, the citizens – all took responsibility for building that Europe. Together.
 \newpage\section{Speech 7 - Juncker - 2017-09-13}
\url{https://ec.europa.eu/commission/presscorner/detail/en/SPEECH_17_3165}\\[3mm]
Mr President, Honourable Members of the European Parliament,

When I stood before you this time last year, I had a somewhat easier speech to give.

It was plain for all to see that our Union was not in a good state.

Europe was battered and bruised by a year that shook our very foundation.

We only had two choices. Either come together around a positive European agenda or each retreat into our own corners.

Faced with this choice, I argued for unity.

I proposed a positive agenda to help create – as I said last year – a Europe that protects, a Europe that empowers, a Europe that defends.

Over the past twelve months, the European Parliament has helped bring this agenda to life. We continue to make progress with each passing day. Just last night you worked to find an agreement on trade defence instruments and on doubling our European investment capacity. And you succeeded. Thank you for that.

I also want to thank the 27 leaders of our Member States. Days after my speech last year, they welcomed my agenda at their summit in Bratislava. In doing so they chose unity. They chose to rally around our common ground.

Together, we showed that Europe can deliver for its citizens when and where it matters.

Ever since, we have been slowly but surely gathering momentum.

It helped that the economic outlook swung in our favour.

We are now in the fifth year of an economic recovery that really reaches every single Member State.

Growth in the European Union has outstripped that of the United States over the last two years. It now stands above 2\% for the Union as a whole and at 2.2\% for the monetary area.

Unemployment is at a nine year low. Almost 8 million jobs have been created during this mandate so far. With 235 million people at work, more people are in employment in the European Union than ever before.

The European Commission cannot take the credit for this alone. Though I am sure that had 8 million jobs been lost, we would have taken the blame.

But Europe's institutions played their part in helping the wind change.

We can take credit for our European Investment Plan which has triggered \euro225 billion worth of investment so far. It has granted loans to 450,000 small firms and more than 270 infrastructure projects.

We can take credit for the fact that, thanks to determined action, European banks once again have the capital firepower to lend to companies so that they can grow and create jobs.

And we can take credit for having brought public deficits down from 6.6\% to 1.6\%. This is thanks to an intelligent application of the Stability and Growth Pact. We ask for fiscal discipline but are careful not to kill growth. This is in fact working very well across the Union – despite the criticism.

Ten years since crisis struck, Europe's economy is finally bouncing back.

And with it, our confidence.

Our 27 leaders, the Parliament and the Commission are putting the Europe back in our Union. And together we are putting the Union back in our Union.

In the last year, we saw all 27 leaders walk up the Capitoline Hill in Rome, one by one, to renew their vows to each other and to our Union.

All of this leads me to believe: the wind is back in Europe's sails.

We now have a window of opportunity but it will not stay open forever.

Let us make the most of the momentum, catch the wind in our sails.

For this we must do two things:

First, we should stay the course set out last year. We still have 16 months in which real progress can be made by Parliament, Council and Commission. We must use this time to finish what we started in Bratislava and deliver on our own positive agenda.

Secondly, we should chart the direction for the future. As Mark Twain wrote – I am quoting – years from now we will be more disappointed by the things we did not do, than by those we did. Now is the time to build a more united, a stronger, a more democratic Europe for 2025.

 


 

STAYING COURSE

Mr President, Honourable Members,

As we look to the future, we cannot let ourselves be blown off course.

We set out to complete an Energy Union, a Security Union, a Capital Markets Union, a Banking Union and a Digital Single Market. Together, we have already come a long way.

As the Parliament testified, 80\% of the proposals promised at the start of the mandate have already been put forward by the Commission. We must now work together to turn proposals into law, and law into practice.

As ever, there will be a degree of give and take. The Commission's proposals to reform our Common Asylum System and strengthen rules on the Posting of Workers have caused controversy, I know. Achieving a good result will need all sides to do their part so they can move towards each other. I want to say today: as long as the outcome is the right one for our Union and is fair to all its Member States, the Commission will be open to compromise

We are now ready to put the remaining 20\% of initiatives on the table by May 2018.

This morning, I sent a Letter of Intent to the President of the European Parliament and to the Prime Minister of Estonia – whose strong work for Europe I would like to praise – outlining the priorities for the year ahead.

I will not and I cannot list all these proposals here, but let me mention five which are particularly important.

Firstly, I want us to strengthen our European trade agenda.

Yes, Europe is open for business. But there must be reciprocity. We have to get what we give.

Trade is not something abstract. Trade is about jobs, creating new opportunities for Europe's businesses big and small. Every additional \euro1 billion in exports supports 14,000 extra jobs in Europe.

Trade is about exporting our standards, be they social or environmental standards, data protection or food safety requirements.

Europe has always been an attractive place to do business.

But over the last year, partners across the globe are lining up at our door to conclude trade agreements with us.

With the help of this Parliament, we have just secured a trade agreement with Canada that will provisionally apply as of next week. We have a political agreement with Japan on a future economic partnership. And by the end of the year, we have a good chance of doing the same with Mexico and South American countries.

Today, we are proposing to open trade negotiations with Australia and New Zealand.

I want all of these agreements to be finalised by the end of this mandate. And I want them negotiated in the greatest transparency.

Open trade must go hand in hand with open policy making.

The European Parliament will have the final say on all trade agreements. So its Members, like members of national and regional parliaments, must be kept fully informed from day one of the negotiations. The Commission will make sure of this.

From now on, the Commission will publish in full all draft negotiating mandates we propose to the Council.

Citizens have the right to know what the Commission is proposing. Gone are the days of no transparency. Gone are the days of rumours, of incessantly questioning the Commission's motives.

I call on the Council to do the same when it adopts the final negotiating mandates.

Let me say once and for all: we are not naïve free traders.

Europe must always defend its strategic interests.

This is why today we are proposing a new EU framework for investment screening. If a foreign, state-owned, company wants to purchase a European harbour, part of our energy infrastructure or a defence technology firm, this should only happen in transparency, with scrutiny and debate. It is a political responsibility to know what is going on in our own backyard so that we can protect our collective security if needed.

 

Secondly, the Commission wants to make our industry stronger and more competitive.

This is particularly true for our manufacturing base and the 32 million workers that form its backbone. They make the world-class products that give us our edge, like our cars.

I am proud of our car industry. But I am shocked when consumers are knowingly and deliberately misled. I call on the car industry to come clean and make it right. Instead of looking for loopholes, they should be investing in the clean cars of tomorrow

Honourable Members, the new Industrial Policy Strategy we are presenting today will help our industries stay, or become, the number one in innovation, digitisation and decarbonisation.

 

Third: I want Europe to be the leader when it comes to the fight against climate change.

Last year, we set the global rules of the game with the Paris Agreement ratified here, in this very House. Set against the collapse of ambition in the United States, Europe must ensure we make our planet great again. It is the shared heritage of all of humanity.

The Commission will shortly present proposals to reduce the carbon emissions of our transport sector.

 

Fourth priority for the year ahead: I want us to better protect Europeans in the digital age.

Over the past years, we have made marked progress in keeping Europeans safe online. New rules, put forward by the Commission, will protect our intellectual property, our cultural diversity and our personal data. We have stepped up the fight against terrorist propaganda and radicalisation online. But Europe is still not well equipped when it comes to cyber-attacks.

Cyber-attacks can be more dangerous to the stability of democracies and economies than guns and tanks. Last year alone there were more than 4,000 ransomware attacks per day and 80\% of European companies experienced at least one cyber-security incident.

Cyber-attacks know no borders and no one is immune. This is why, today, the Commission is proposing new tools, including a European Cybersecurity Agency, to help defend us against such attacks.

Fifth: migration must stay on our radar.

In spite of the debate and controversy around this topic, we have managed to make solid progress – though admittedly insufficient in many areas.

We are now protecting Europe's external borders more effectively. Over 1,700 officers from the new European Border and Coast Guard are now helping Member States' 100,000 national border guards patrol in places like Greece, Italy, Bulgaria and Spain. We have common borders but Member States that by geography are the first in line cannot be left alone to protect them. Common borders and common protection must go hand in hand.

We have managed to stem irregular flows of migrants, which were a cause of great anxiety for many. We have reduced irregular arrivals in the Eastern Mediterranean by 97\% thanks our agreement with Turkey. And this summer, we managed to get more control over the Central Mediterranean route with arrivals in August down by 81\% compared to the same month last year.

In doing so, we have drastically reduced the loss of life in the Mediterranean.

I cannot talk about migration without paying strong tribute to Italy for their tireless and noble work. Over the summer months, the Commission worked in perfect harmony with the Prime Minister of Italy, my friend Paolo Gentiloni, and his government to improve the situation. We did so - and we will continue to do so - because Italy is saving Europe's honour in the Mediterranean.

We must also urgently improve migrants' living conditions in Libya. I am appalled by the inhumane conditions in detention or reception centres. Europe has a responsibility – a collective responsibility – and the Commission will work in concert with the United Nations to put an end to this scandalous situation that cannot be made to last.

Even if it saddens me to see that solidarity is not yet equally shared across all our Member States, Europe as a whole has continued to show solidarity. Last year alone, our Member States resettled or granted asylum to over 720,000 refugees – three times as much as the United States, Canada and Australia combined. Europe, contrary to what some say, is not a fortress and must never become one. Europe is and must remain the continent of solidarity where those fleeing persecution can find refuge.

I am particularly proud of the young Europeans volunteering to give language courses to Syrian refugees or the thousands more young people who are serving in our new European Solidarity Corps. These young people are bringing life and colour to European solidarity.

But we now need to redouble our efforts. At the end of the month, the Commission will present a new set of proposals with an emphasis on returns, solidarity with Africa and opening legal pathways.

When it comes to returns, I would like to repeat that people who have no right to stay in Europe must be returned to their countries of origin. When only 36\% of irregular migrants are returned, it is clear we need to significantly step up our work. This is the only way Europe will be able to show solidarity with refugees in real need of protection.

Solidarity cannot be an exclusively intra-European affair. We must also showsolidarity withAfrica. Africa is a noble continent, a young continent, the cradle of humanity. Our \euro2.7 billion EU-Africa Trust Fund is creating employment opportunities across the continent. The EU budget fronted the bulk of the money, but all our Member States combined have still only contributed \euro150 million. The Fund is currently reaching its limits. We know – or we should know – the dangers of a lack of funding – in 2015 many migrants headed towards Europe when the UN's World Food Programme ran out of funds. I call on all Member States to now match their actions with their words and ensure the Africa Trust Fund does not meet the same fate. The risk is high.

We will also work on opening up legal pathways. Irregular migration will only stop if there is a real alternative to perilous journeys. We are close to having resettled 22,000 refugees from Turkey, Jordan and Lebanon and I support UN High Commissioner for Refugees' call to resettle a further 40,000 refugees from Libya and the surrounding countries.

At the same time, legal migration is an absolute necessity for Europe as an ageing continent. This is why the Commission made proposals to make it easier for skilled migrants to reach Europe with a Blue Card. I would like to thank the Parliament for its support on this.

 

 

 

 

SETTING SAIL

Dear Mr President, Ladies and Gentlemen,

Honourable Members,

I have mentioned just a few of the initiatives we want and must deliver over the next 16 months. But this alone will not be enough to regain the hearts and minds of Europeans.

Now is the time to chart the direction for the future.

In March, the Commission presented our White Paper on the future of Europe, with five scenarios for what Europe could look like by 2025. These scenarios have been discussed, sometimes superficially, sometimes violently. They have been scrutinised and partly ripped apart. That is good – they were conceived for exactly this purpose. I wanted to launch a process in which Europeans determined their own path and their own future.

The future of Europe cannot be decided by decree. It has to be the result of democratic debate and, ultimately, broad consensus. This House contributed actively, through the three ambitious resolutions on Europe's future which I would like to particularly thank the rapporteurs for. And I want to thank all the colleagues that participated in the more than 2,000 public events across Europe that the Commission organised since March.

Now is the time to draw first conclusions from this debate. Time to move from reflection to action. From debate to decision.

Today I would like to present you my view: my own 'sixth scenario', if you will.

This scenario is rooted in decades of first-hand experience. I have lived, fought and worked for the European project my entire life. I have seen and lived through good times and bad.

I have sat on many different sides of the table: as a Minister, as Prime Minister, as President of the Eurogroup, and now as President of the Commission. I was there in Maastricht, Amsterdam, Nice and Lisbon as our Union evolved and enlarged.

I have always fought for Europe. At times I have suffered because of Europe. And even despaired for Europe.

Through thick and thin, I have never lost my love of Europe.

But there is, as we know, rarely love without pain.

Love for Europe because Europe and the European Union have achieved something unique in this fraying world: peace within and peace outside of Europe. Prosperity for many if not yet for all.

This is something we have to remember during the European Year of Cultural Heritage. 2018 must be a celebration of cultural diversity.

 

 

A UNION OF VALUES

Our values are our compass.

For me, Europe is more than just a single market. More than money, more than a currency, more than the euro. It was always about values.

That is why, in my sixth scenario, there are three fundamentals, three unshakeable principles: freedom, equality and the rule of law.

 

Europe is first of all a Union of freedom. Freedom from the kind of oppression and dictatorship our continent knows all too well – sadly none more than central and Eastern European countries. Freedom to voice your opinion, as a citizen and as a journalist – a freedom we too often take for granted. It was on these freedoms that our Union was built. But freedom does not fall from the sky. It must be fought for. In Europe and throughout the world.

 

Second, Europe must be a Union of equality and a Union of equals.

Equality between its Members, big or small, East or West, North or South.

Make no mistake, Europe extends from Vigo to Varna. From Spain to Bulgaria.

East to West: Europe must breathe with both lungs. Otherwise our continent will struggle for air.

In a Union of equals, there can be no second class citizens. It is unacceptable that in 2017 there are still children dying of diseases that should long have been eradicated in Europe. Children in Romania or Italy must have the same access to measles vaccines as children in other European countries. No ifs, no buts. This is why we are working with all Member States to support national vaccination efforts. Avoidable deaths must not occur in Europe.

In a Union of equals, there can be no second class workers. Workers should earn the same pay for the same work in the same place. This is why the Commission proposed new rules on posting of workers. We should make sure that all EU rules on labour mobility are enforced in a fair, simple and effective way by a new European inspection and enforcement body. It is absurd to have a Banking Authority to police banking standards, but no common Labour Authority for ensuring fairness in our single market. We will create such an Authority.

In a Union of equals, there can be no second class consumers either. I cannot accept that in some parts of Europe, in Central and Eastern Europe,people are sold food of lower quality than in other countries, despite the packaging and branding being identical. Slovaks do not deserve less fish in their fish fingers. Hungarians less meat in their meals. Czechs less cacao in their chocolate. EU law outlaws such practices already. And we must now equip national authorities with stronger powers to cut out these illegal practices wherever they exist.

 

Third, in Europe the strength of the law replaced the law of the strong.

The rule of law means that law and justice are upheld by an independent judiciary.

Accepting and respecting a final judgement is what it means to be part of a Union based on the rule of law. Our Member States gave final jurisdiction to the European Court of Justice. The judgements of the Court have to be respected by all. To undermine them, or to undermine the independence of national courts, is to strip citizens of their fundamental rights.

The rule of law is not optional in the European Union. It is a must.

Our Union is not a State but it must be a community of law.

A MORE UNITED UNION

These three principles – freedom, equality and the rule of law – must remain the foundations on which we build a more united, stronger and more democratic Union.

When we talk about the future, experience tells me new Treaties and new institutions are not the answer people are looking for. They are merely a means to an end, nothing more, nothing less. They might mean something to us here in Strasbourg or in Brussels. They do not mean a lot to anyone else.

I am only interested in institutional reforms if they lead to more efficiency in our European Union.

Instead of hiding behind calls for Treaty change – which is in any case inevitable – we must first change the mind-set that for some to win others must lose.

Democracy is about compromise. And the right compromise makes winners out of everyone in the long run. A more united Union should see compromise, not as something negative, but as the art of bridging differences. Democracy cannot function without compromise. Europe cannot function without compromise.

A more united Union also needs to become more inclusive.

If we want to protect our external borders and rightly so strengthen them even more, then we need to open the Schengen area of free movement to Bulgaria and Romania immediately. We should also allow Croatia to become a full Schengen member once all the criteria are met.

If we want the euro to unite rather than divide our continent, then it should be more than the currency of a select group of countries. The euro is meant to be the single currency of the European Union as a whole. All but two of our Member States are required and entitled to join the euro once they fulfil the conditions.

Member States that want to join the euro must be able to do so. This is why I am proposing to create a Euro-accession Instrument, offering technical and even financial assistance.

If we want banks to operate under the same rules and under the same supervision across our continent, then we should encourage all Member States to join the Banking Union. We need to reduce the remaining risks in the banking systems of some of our Member States. Banking Union can only function if risk-reduction and risk-sharing go hand in hand. As everyone well knows, this can only be achieved if the conditions, as proposed by the Commission in November 2015, are met. There can only be a common deposit insurance scheme once everyone will have done their national homework.

And if we want to avoid social fragmentation and social dumping in Europe, then Member States should agree on the European Pillar of Social Rights as soon as possible and at the latest at the Gothenburg summit in November. National social systems will still remain diverse and separate for a long time. But at the very least, we should agree on a European Social Standards Union in which we have a common understanding of what is socially fair in our single market.

I remain convinced: Europe cannot work if it shuns workers.

Ladies and Gentlemen, if we want more stability in our neighbourhood, then we must also maintain a credible enlargement perspective for the Western Balkans.

It is clear that there will be no further enlargement during the mandate of this Commission and this Parliament. No candidate is ready. But thereafter the European Union will be greater than 27 in number. Accession candidates must give the rule of law, justice and fundamental rights utmost priority in the negotiations.

This rules out EU membership for Turkey for the foreseeable future.

Turkey has been taking giant strides away from the European Union for some time.

Journalists belong in newsrooms not in prisons. They belong where freedom of expression reigns.

The call I make to those in power in Turkey is this: Let our journalists go. And not only ours. Stop insulting our Member States by comparing their leaders to fascists and Nazis. Europe is a continent of mature democracies. But deliberate insults create roadblocks. Sometimes I get the feeling Turkey is deliberately placing these roadblocks so that it can blame Europe for any breakdown in accession talks.

As for us, we will always keep our hands stretched out towards the great Turkish people and all those who are ready to work with us on the basis of our values.

 

 

A STRONGER UNION

Ladies and Gentlemen,

I want our Union to be stronger and for this we need a stronger single market.

When it comes to important single market questions, I want decisions in the Council to be taken more often and more easily by qualified majority – with the equal involvement of the European Parliament. We do not need to change the Treaties for this. There are so-called “passerelle clauses” in the current Treaties which allow us to move from unanimity to qualified majority voting in certain cases – provided the European Council decides unanimously to do so.

I am also strongly in favour of moving to qualified majority voting for decisions on the common consolidated corporate tax base, on VAT, on fair taxes for the digital industry and on the financial transaction tax.

 

Europe has to be able to act quicker and more decisively, and this also applies to theEconomic and Monetary Union.

The euro area is more resilient now than in years past. We now have the European Stabilisation Mechanism (ESM). I believe the ESM should now progressively graduate into a European Monetary Fund which, however, must be firmly anchored in the European Union's rules and competences. The Commission will make concrete proposals for this in December.

We need a European Minister of Economy and Finance: a European Minister that promotes and supports structural reforms in our Member States. He or she can build on the work the Commission has been doing since 2015 with our Structural Reform Support Service. The new Minister should coordinate all EU financial instruments that can be deployed if a Member State is in a recession or hit by a fundamental crisis.

I am not calling for a new position just for the sake of it. I am calling for efficiency. The Commissioner for economic and financial affairs – ideally also a Vice-President – should assume the role of Economy and Finance Minister. He or she should also preside the Eurogroup.

The European Economy and Finance Minister must be accountable to the European Parliament.

We do not need parallel structures. We do not need a budget for the Euro area but a strong Euro area budget line within the EU budget.

I am also not fond of the idea of having a separate euro area parliament.

The Parliament of the euro area is this European Parliament.

 

The European Union must also be stronger in fighting terrorism. In the past three years, we have made real progress. But we still lack the means to act quickly in case of cross-border terrorist threats.

This is why I call for setting up a European intelligence unit that ensures data concerning terrorists and foreign fighters are automatically shared among intelligence services and with the police.

I also see a strong case for tasking the new European Public Prosecutor with prosecuting cross-border terrorist crimes.

 

I want our Union to become a stronger global actor. In order to have more weight in the world, we must be able to take foreign policy decisions quicker. This is why I want Member States to look at which foreign policy decisions could be moved from unanimity to qualified majority voting. The Treaty already provides for this, if all Member States agree to do it. We need qualified majority decisions in foreign policy if we are to work efficiently.

And I want us to dedicate further efforts to defence matters. A new European Defence Fund is in the offing. As is a Permanent Structured Cooperation in the area of defence. By 2025 we need a fully-fledged European Defence Union. We need it. And NATO wants it.

 

Last but not least, I want our Union to have a stronger focus on things that matter, building on the work this Commission has already undertaken. We should not meddle in the everyday lives of European citizens by regulating every aspect. We should be big on the big things. We should not march in with a stream of new initiatives or seek ever growing competences. We should give back competences to Member States where it makes sense.

This is why this Commission has sought to be big on big issues and small on the small ones and has done so, putting forward less than 25 new initiatives a year where previous Commissions proposed well over 100.

To finish the work we started, I am setting up a Subsidiarity and Proportionality Task Force as of this month to take a very critical look at all policy areas to make sure we are only acting where the EU adds value. The First Vice-President, my friend, Frans Timmermans, who has a proven track record on better regulation, will head this Task Force. The Timmermans Task Force should include Members of this Parliament as well as Members of national Parliaments. It should report back in a years' time.

 

A MORE DEMOCRATIC UNION

Honourable Members,

Mr President,

Our Union needs to take a democratic leap forward.

I would like to see European political parties start campaigning for the next European elections much earlier than in the past. Too often Europe-wide elections have been reduced to nothing more than the sum of national campaigns. European democracy deserves better.

Today, the Commission is proposing new rules on the financing of political parties and foundations. We should not be filling the coffers of anti-European extremists. We should be giving European parties the means to better organise themselves.

I also have sympathy for the idea of having transnational lists in European elections – though I am aware this is an idea more than a few of you disagree with. I will seek to convince the President of my parliamentary Group to follow me in this ambition which will bring Europe democracy and clarity.

I also believe that, over the months to come, we should involve national Parliaments and civil society at national, regional and local level more in the work on the future of Europe. Over the last three years, as we promised, Members of the Commission have visited national Parliaments more than 650 times. They also debated in more than 300 interactive Citizens' Dialogues in more than 80 cities and towns across 27 Member States. This is why I support President Macron's idea of organising democratic conventions across Europe in 2018.

As the debate gathers pace, I will personally pay particular attention to Estonia, to Latvia, to Lithuania and to Romania in 2018. This is the year they will celebrate their 100th anniversary. Those who want to shape the future of our continent should well understand and honour our common history. This includes these four countries – the European Union would not be whole without them.

The need to strengthen democracy and transparency also has implications for the European Commission. Today, I am sending the European Parliament a new Code of Conduct for Commissioners. The new Code first of all makes clear that Commissioners can be candidates in European Parliament elections under the same conditions as everyone else. The new Code will of course strengthen the integrity requirements for Commissioners both during and after their mandate.

If you want to strengthen European democracy, then you cannot reverse the small democratic progress seen with the creation of lead candidates – 'Spitzenkandidaten'. I would like the experience to be repeated.

More democracy means more efficiency. Europe would function better if we were to merge the Presidents of the European Council and the European Commission.

This is nothing against my good friend Donald, with whom I have worked intimately and seamlessly together since the beginning of my mandate. This is nothing against Donald or against me.

Europe would be easier to understand if one captain was steering the ship.

Having a single President would simply better reflect the true nature of our European Union as both a Union of States and a Union of citizens.

 

 

OUR ROADMAP

My dear colleagues,

The vision of a more united, stronger and more democratic Europe I am outlining today combines elements from all of the scenarios I set out to you in March.

But our future cannot remain a simple scenario, a sketch, an idea amongst others.

We have to prepare the Union of tomorrow, today.

This morning I sent a Roadmap to President Tajani, President Tusk as well as to the holders of the rotating Presidencies of the Council between now and March 2019, outlining where we should go from here.

An important element will be the budgetary plans the Commission will present in May 2018. Here again we have a choice: either we pursue the European Union's ambitions in the strict framework of the existing budget, or we increase the European Union's budgetary capacity so that it might better reach its ambitions. I am for the second option.

On 29 March 2019, the United Kingdom will leave the European Union. This will be both a sad and tragic moment. We will always regret it. But we have to respect the will of the British people. We will advance, we must advance because Brexit is not everything. Because Brexit is not the future of Europe.

On 30 March 2019, we will be a Union of 27. I suggest that we prepare for this moment well, amongst the 27 and within the EU institutions.

European Parliament elections will take place just a few weeks later, in May 2019. Europeans have a date with democracy. They need to go to the polls with a clear understanding of how the European Union will develop over the years to come.

This is why I call on President Tusk and Romania, the country holding the Presidency in the first half of 2019, to organise a Special Summit in Romania on 30 March 2019. My wish is that this summit be held in the beautiful city of Sibiu, also known as Hermannstadt. This should be the moment we come together to take the decisions needed for a more united, stronger and democratic Europe.

My hope is that on 30 March 2019, Europeans will wake up to a Union where we stand by all our values. Where all Member States respect the rule of law without exception. Where being a full member of the euro area, the Banking Union and the Schengen area has become the norm for all.

Where we have shored up the foundations of our Economic and Monetary Union so that we can defend our single currency in good times and bad, without having to call on external help. Where our single market will be fairer towards workers from the East and from the West.

I want Europeans to wake up to a Europe where we have managed to agree on a strong pillar of social standards. Where profits will be taxed where they were made. Where terrorists have no loopholes to exploit. Where we have agreed on a proper European Defence Union. Where eventually a single President leads the work of the Commission and the European Council, having been elected after a democratic Europe-wide election campaign.

Mr President, if our citizens wake up to this Union on 30 March 2019, then the European Union will be a Union able to meet their legitimate expectations.

 

 

CONCLUSION

Honourable Members,

Europe was not made to stand still. It must never do so.

Helmut Kohl and Jacques Delors, whom I had the honour to know, taught me that Europe only moves forward when it is bold. The single market, Schengen and the single currency: these were all ideas that were written off as pipe dreams before they happened. And yet these three ambitious projects are now a part of our daily reality.

Now that Europe is doing better, people tell me I should not rock the boat.

But now is not the time to err on the side of caution.

We started to fix the European roof. But today and tomorrow we must patiently, floor by floor, moment by moment, inspiration by inspiration, continue to add new floors to the European House.

We must complete the European House now that the sun is shining and whilst it still is.

Because when the next clouds appear on the horizon – and they will appear one day – it will be too late.

So let's throw off the bowlines.

Sail away from the harbour.

And catch the trade winds in our sails.
 \newpage\section{Speech 8 - Juncker - 2018-09-12}
\url{https://ec.europa.eu/commission/presscorner/detail/en/speech_18_5808}\\[3mm]
INTRODUCTION: A PERPETUAL RESPONSIBILITY

 

Mr President,

Madam President of the Council,

Honourable Members of the European Parliament,

At times, history moves forward only haltingly but it is always quick to pass us by.

Such is the fate of a Commission with just a five-year mandate to make a real difference. In such a short space of time, it is not possible to change the course of things definitively.

This Commission – like those that have gone before – is merely a chapter, a brief moment in the long history of the European Union. But the time has not yet come to pass judgement on the Commission over which I preside.

This is why I will not today present you with an overview of the last four years' achievements.

Instead, I say to you that our efforts will continue unabated over the coming months to render this imperfect Union that little bit more perfect with each passing day.

There is much still to be done. And this is what I want to talk to you about this morning.

No self-congratulating, no boasting. Modesty and hard work: this is the attitude the Commission will continue to adopt. This is what is on our agenda for the months to come.

History - in the true sense of the term - can also show up, unannounced, in the life of nations and be slow to leave.

Such was the fate of Europe's nations during the Great War starting in 1914. A war which took the sunny, optimistic and peaceful continent of the time by surprise.

In 1913, Europeans expected to live a lasting peace. And yet, just a year later, a brutal war broke out amongst brothers, engulfing the continent.

I speak of these times not because I believe we are on the brink of another catastrophe in Europe. But because Europe is the guardian of peace.

We should be thankful we live on a peaceful continent, made possible by the European Union.

So let us show the European Union a bit more respect. Let us stop dragging its name through the mud and start defending our communal way of life more.

We should embrace the kind of patriotism that is used for good, and never against others. We should reject the kind of exaggerated nationalism that projects hate and destroys all in its path. The kind of nationalism that points the finger at others instead of searching for ways to better live together.

Living up to Europe's rallying cry – never again war – is our eternal duty, our perpetual responsibility. We must all remain vigilant.

 

THE STATE OF OUR UNION IN 2018

EFFORTS THAT ARE BEARING FRUIT

Honourable Members of the European Parliament,

What is the State of the Union today, in 2018?

Ten years after Lehman Brothers, Europe has largely turned the page on an economic and financial crisis which came from outside but which cut deep at home.

Europe's economy has now grown for 21 consecutive quarters.

Jobs have returned, with almost 12 million new jobs created since 2014. 12 million – that is more jobs than there are people in Belgium.

Never have so many men and women – 239 million people – been in work in Europe.

Youth unemployment is at 14.8 \%. This is still too high a figure but is the lowest it has been since the year 2000.

Investment is back, thanks notably to our European Fund for Strategic Investments, which some – less and less – still call the ‘Juncker Fund'. A Fund that has triggered 335 billion euro worth of public and private investment. We are closing in on 400 billion.

And then there is Greece. After what can only be described as some very painful years, marked by unprecedented social hardship, Greece successfully exited its programme and is now back on its own two feet. I once again applaud the people of Greece for their Herculean efforts. Efforts which other Europeans continue to underestimate. As you know, I have always fought for Greece, its dignity, its role in Europe, and its place inside the euro area. Of this I am proud.

Europe has also reaffirmed its position as a trade power. Our global trading position is the living proof of the need to share sovereignty. The European Union now has trade agreements with 70 countries around the world, covering 40 \% of the world's GDP. These agreements – so often contested but often unjustly – help us export Europe's high standards for food safety, workers' rights, the environment and consumer rights far beyond our borders.

When, amidst dangerous global tensions, I went to Beijing, Tokyo and Washington in the space of one week last July, I was able to speak, as President of the European Commission, on behalf of the world's biggest single market. On behalf of a Union accounting for a fifth of the world's economy. On behalf of a Union willing to stand up for its values and interests. I showed Europe to be an open continent. But not a naïve one.

The strength of a united Europe, both in principle and in practice, gave me the clout I needed to get tangible results for citizens and businesses alike.

United, as a Union, Europe is a force to be reckoned with. In Washington, I spoke in Europe's name. For some, the agreement I struck with President Trump came as a surprise. But it was no surprise – as whenever Europe speaks with one voice, there is never any surprise. When Europe speaks with one voice, it can prevail.

When needed, Europe must act as one.

 

A GLOBAL RESPONSIBILITY

We proved this when relentlessly defending the Paris Agreement on climate change. We did this because, as Europeans, we want to leave a healthier planet behind for those that follow. I share our Energy Commissioner's conclusions when it comes to our targets for reducing CO2 emissions by 2030. They are both scientifically accurate and politically indispensable.

This summer's droughts are a stark reminder – not only for farmers – of just how important that work is to safeguard the future for generations of Europeans. We obviously cannot turn a blind eye to the climate challenge. We – Commission and Parliament – must look to the future.

 

Ladies and Gentlemen,

The world has not stopped turning. It is more volatile than ever. The external challenges facing our continent are multiplying by the day.

There can therefore be not a moment's respite in our efforts to build a stronger and more united Europe.

Europe can export stability, as we have done with the successive enlargements of our Union. For me, these are and will remain success stories – for we were able to reconcile Europe's history and geography. But there is more to be done. We must find unity when it comes to the Western Balkans and their future membership. Should we not, our immediate neighbourhood will be shaped by others.

Take a look around. What is happening in Syria, in Idlib, now must be of deep and direct concern to us all. We cannot remain silent in face of this impending humanitarian disaster – which appears now all but inevitable.

The conflict in Syria is a case in point for how the international order that served Europeans so well after the Second World War is being increasingly called into question.

In today's world, Europe can no longer be certain that words given yesterday can still be counted on today.

That old alliances may not look the same tomorrow.

THE HOUR OF EUROPEAN SOVEREIGNTY

As I have said, the world today needs a strong and united Europe.

A Europe that works for peace, trade agreements and stable currency relations, even as others become rather too prone to trade and currency wars. I am not in favour of a selfish unilateralism that defies expectations and dashes hopes. I will always firmly champion multilateralism.

If Europe were better aware of the political, economic and military might of its nations, its role in the world could be strengthened. We will always be a global payer but it is time we started being a global player too.

This is why – despite great resistance at the time – I reignited the Europe of Defence as early as 2014. And this is why I will continue to work day and night over the next months to see the European Defence Fund and Permanent Structured Cooperation in Defence become fully operational. Allow me to clarify one important point: we will not militarise the European Union. What we want is to become more autonomous and live up to our global responsibilities.

Only a strong and united Europe can protect our citizens against threats internal and external – from terrorism to climate change.

Only a strong and united Europe can protect jobs in an open, interconnected world.

Only a strong and united Europe can master the challenges of global digitisation.

It is because of our single market – the largest in the world – that we can set standards for big data, artificial intelligence, and automation. And that we are able to uphold Europeans' values, rights and identities in doing so. But we can only do so if we stand united.

A strong and united Europe is what allows its Member States to reach for the stars. It is our Galileo programme that is today keeping Europe in the space race. No single Member State – none – could have put 26 satellites in orbit, for the benefit of 400 million users worldwide. No single Member State could have done this alone. Galileo is a success – in great part, if not entirely, thanks to Europe. No Europe, no Galileo. We should be proud.

 

Mr President,

The geopolitical situation makes this Europe's hour: the time for European sovereignty has come.

It is time Europe took its destiny into its own hands. It is time Europe developed what I coined ‘Weltpolitikfähigkeit' – the capacity to play our role in shaping global affairs. Europe has to become a more sovereign actor in international relations.

European sovereignty is born of Member States' national sovereignty and does not replace it. Sharing sovereignty – when and where needed – makes each of our States and nations stronger.

This belief that ‘united we stand taller' is the very essence of what it means to be part of the European Union.

European sovereignty can never be to the detriment of others. Europe is a continent of openness and tolerance. And it will remain so.

Europe will never be a fortress, turning its back on the world or those suffering within it. Europe is not an island. It must and will champion multilateralism. Because the world we live in belongs to all and not a select few.

This is what is at stake when Europeans take to the polls in May next year. We will use the 250 days before the European elections to prove to citizens that, acting as one, this Union is capable of delivering on expectations and on what we promised to achieve at the start of this mandate.

By the elections, we must show that Europe can overcome differences between North and South, East and West. The European Union – Europe – is too small to let itself be divided in halves or quarters.

We must show that together – East, West, South and North – we can plant the seeds of a more sovereign Europe.

  

DELIVERING ON OUR PROMISES

Mr President,

Ladies and Gentlemen,

Europeans taking to the polls in May 2019 will not care what the Commission has proposed. But they will care strongly about internet giants paying taxes where they make their profits. What the voters want – and a good many of you too; not all, I know – is for the Commission's proposal to become law quickly. And they are quite right.

Europeans taking to the polls in May 2019 will not care about the Commission's good intention to crack down on single-use plastics to protect our oceans against marine litter. To get Europeans on our side and convince them that what we are doing is right, we need a European law that bans these plastics to be actually in force.

We all say – usually in soap-box speeches – that we want to be big on big things and small on small things. But there will be no applause in May 2019 when EU law dictates that Europeans have to continue to change their clocks twice a year. Clock-changing must stop. Member States should themselves decide whether their citizens live in summer or winter time. It is a question of subsidiarity. I expect the Parliament and Council will share this view and will ensure that regional solutions are found that work for the internal market. We are out of time.

This is why I am today calling on all to work closely together over the next months, so that we can jointly deliver on what we promised in 2014 – before the European Parliament elections.

At the beginning of this mandate, we all collectively promised to deliver a more innovative Digital Single Market, a deeper Economic and Monetary Union, a Banking Union, a Capital Markets Union, a fairer Single Market, an Energy Union with a forward-looking climate policy, a comprehensive Migration Agenda, and a Security Union. And we – or at least most of us – resolved that the social dimension of Europe should no longer be given Cinderella treatment, but should be geared towards the future.

The Commission has put all the proposals and initiatives we announced in 2014 on the table. Half of them have been adopted by Parliament and the Council. 20 \% are well on the way. And 30 \% are still under discussion – in some cases difficult discussion.

Ladies and Gentlemen,

I cannot accept that the blame for every failure – and of course there have been a few – is laid solely at the Commission's door. Our proposals are there for all to see, in particular as regards refugees and migration. They need to be adopted and implemented. Whatever happens, I will continue to resist all attempts to blame the Commission alone – and I already know this will happen. There are scapegoats to be found in all the institutions – with the fewest in Commission and Parliament.

Leadership on the broadest front is what is needed now. This is notably the case when it comes to completing our Security Union. Europeans rightly expect their Union to keep them safe. This is why the Commission is today proposing new rules to get terrorist content off the web within one hour – the critical window in which the greatest damage can be done. And we are proposing to extend the tasks of the newly established European Public Prosecutor's Office to include the fight against terrorist offences. We need to be able to prosecute terrorists in a more coordinated way, across our Union. Terrorists know no borders. We cannot allow ourselves to become unwitting accomplices because of our inability to cooperate.

In the same vein, we have also today proposed new measures to allow us to fight money laundering more effectively across our borders.

We must be equally committed to ensuring free and fair elections in Europe. This is why the Commission is proposing new rules to better protect our democratic processes from manipulation by third countries and, as we have already seen, powerful private interests.

Leadership and a spirit of compromise are of course very much needed when it comes to migration. We have made more progress on this than is often acknowledged. Five of the seven Commission proposals to reform our Common European Asylum System have been agreed. Our efforts to manage migration have borne fruit, with migrant numbers down by 97 \% in the Eastern Mediterranean and by 80 \% along the Central Mediterranean route. EU operations have helped rescue over 690,000 people at sea since 2015.

However, Member States have not yet found the right balance between the responsibility each must assume on its own territory and the solidarity all must show if we are to preserve a Schengen area without internal borders. I am and will remain strictly opposed to internal borders. Where borders have been reinstated, they must be removed. Failure to do so would amount to an unacceptable step back for the Europe of today and tomorrow.

The Commission and several Council presidencies have put numerous migration related proposals on the table. I call on the Austrian presidency to now take the decisive step to broker a sustainable solution on a balanced migration reform. I know that it will do so. We cannot continue to squabble to find ad hoc solutions each time a new ship arrives. Temporary solidarity is not good enough. We need more, better organised, lasting solidarity – today and forever more.

We need more solidarity not for solidarity's sake but for the sake of efficiency. This is true in the case of our civil protection mechanism. When fires rage in one European country, all of Europe burns. The most striking images from this summer were not only those of the formidable fires, but also those of the Swedish people in danger applauding Polish firefighters coming to their aid – Europe at its best.

Turning back to migration: the Commission is today proposing to further strengthen the European Border and Coast Guard to better protect our external borders with an additional 10,000 European border guards funded by the EU budget by 2020.

We are also proposing to further develop the European Asylum Agency to make sure that Member States get more European support in processing asylum seekers in line with the Geneva Convention.

And we are proposing to accelerate the return of irregular migrants.

I would also like to remind Member States of the need to open legal pathways to the Union. I renew my call. We need skilled migrants. Commission proposals addressing this issue have been on the table for some time and must now be taken up.

 

Mr President,

Honourable Members of the European Parliament,

To speak of the future, one must speak of Africa – Europe's twin continent. 

By 2050, Africa's population will number 2.5 billion. One in four people on earth will be African.

We need to invest more in our relationship with the nations of this great and noble continent. And we have to stop seeing this relationship through the sole prism of development aid. Such an approach is beyond inadequate, in fact it is humiliating.

Africa does not need charity, it needs true and fair partnerships. And we in Europe need this partnership just as much.

In preparing my speech today, I spoke to my African friends, notably Paul Kagame, the Chairperson of the African Union. We agreed that reciprocal commitments are the way forward.

We want to build a new partnership with Africa. Today, we are proposing a new Alliance for Sustainable Investment and Jobs between Europe and Africa. This Alliance – as we envision it – would help create up to 10 million jobs in Africa in the next 5 years alone.

We want to create a framework that brings more private investment to Africa.

We are not starting from scratch: our External Investment Plan, launched here two years ago, will mobilise over \euro44 billion in both the public and private investment. Alone the projects already that are in the pipeline or have already begun will unlock \euro24 billion.

We want to focus our investment where it matters the most. By 2020, the EU will have supported 35,000 African students and researchers with our strengthened Erasmus programme. By 2027, this figure will reach 105,000.

Trade between Europe and Africa is not insignificant. 36 \% of Africa's trade is already with the European Union. But this is not enough. I believe we should develop the numerous trade agreements between Europe and the countries of Africa into a continent-to-continent free trade agreement, as an economic partnership between equals.

 

Mr President,

Ladies and Gentlemen,

Another issue where I see a strong need for the Union for leadership is Brexit. I will not enter into the details of the negotiations, which are being masterfully handled by my friend Michel Barnier. He works on the basis of a unanimous position confirmed time and again by the 27 Member States. However, allow me to recall three principles which should guide our work on Brexit in the months to come.

First of all, we of course respect the British decision to leave our Union, but we regret it deeply. But we also ask the British government to understand that someone who leaves the Union cannot be in the same privileged position as a Member State. If you leave the Union, you are of course no longer part of our Single Market, and certainly not only in parts of it.

Secondly, the European Commission, this Parliament and all other Member States will always show loyalty and solidarity with Ireland when it comes to the Irish border. This is why we want to find a creative solution that prevents a hard border in Northern Ireland. And we will defend all the elements of the Good Friday Agreement. It is Brexit that risks making the border more visible in Northern Ireland. It is not the European Union.

Thirdly, after 29 March 2019, the United Kingdom will never be an ordinary third country for us. The United Kingdom will always be a very close neighbour and partner, in political, economic and security terms.

In the past months, when we needed unity in the Union, Britain was at our side, driven by the same values and principles of all other Europeans. This is why I welcome Prime Minister May's proposal to develop an ambitious new partnership for the future, after Brexit. We agree with the statement made in Chequers that the starting point for such a partnership should be a free trade area between the United Kingdom and the European Union.

On the basis of these principles, the Commission's negotiators, mainly my good friend Michel Barnier, stand ready to work day and night to reach a deal. We owe it to our citizens and our businesses to ensure the United Kingdom's withdrawal is orderly and that there is stability afterwards. It will not be the Commission that will stand in the way of this.

 

A STRONG PERSPECTIVE FOR THE FUTURE

Honourable Members of the European Parliament,

There is much work to be done before the European elections and before Europe's Leaders meet in Romania on 9 May 2019.

In Sibiu, we will need to convince our fellow citizens that, on the essential points, we share the same ideas on the purpose of our Union. Europeans deserve better than uncertainty and confused objectives. They deserve clarity of intent, not approximations or half-measures. So let us not give European citizens half-measures.

This is what is at stake on the road to Sibiu.

By then we must have ratified the EU-Japan partnership agreement – for both economic and geopolitical reasons.

By then, we should also have brokered an agreement in principle on the EU budget after 2020.

If we want to give young Europeans the opportunity to make the most of our Erasmus programme – which we must – then we must decide on this, as well as other aspects of the budget.

If we want to give our researchers and start-ups more opportunities, and prevent funding gaps costing research jobs, we have to decide before the elections.

If we want – without militarising the European Union – to increase defence spending by a factor of 20, we will need to decide upon this before the European elections.

If we want to increase our investment in Africa by 23 \%, we must decide quickly.

I am always being told – by Heads of State, ministers, MEPs and national MPs – that decisions cannot be taken before elections, as if elections were some sort of crisis of democracy. No. Elections are a natural part of democracy and it is natural – even urgent – to take decisions before elections to show the way to those voting for us.

By next year, we should also address the international role of the euro. The euro is 20 years young and has already come a long way – despite its critics.

It is now the world's second currency, after the US dollar, with 60 countries linking their currencies to the euro in one way or another. But we must do more to allow our single currency to play its full role on the international scene.

Recent events have brought into sharp focus the need to deepen our Economic and Monetary Union and build deep and liquid capital markets. The Commission has made a series of proposals to do just that – most of which now await adoption.

We can and must go further. It is absurd that Europe pays for 80 \% of its energy import bill – worth 300 billion euro a year – in US dollars when only roughly 2 \% of our energy imports come from the United States. It is absurd, ridiculous that European companies buy European planes in dollars instead of euro. This all needs to be changed.

This is why, before the end of the year, the Commission will present initiatives to strengthen the international role of the euro. The euro must become the face and the instrument of a new, more sovereign Europe. For this, we must first put our own house in order by strengthening and deepening our Economic and Monetary Union, as we have already started to do. Without this, we will lack the means to strengthen the international of role of the euro. We must complete our Economic and Monetary Union to make Europe and the euro stronger.

Last but not least, by Sibiu (Hermannstadt) I want to make visible progress in strengthening our foreign policy. We must improve our ability to speak with one voice when it comes to our foreign policy. It is not right that Europe silenced itself at the United Nations Human Rights Council when it came to condemning human rights abuses by China because one Member State opposed it. I give this one example – I could give many others.

This is why today the Commission is proposing to move to qualified majority voting in specific areas of our external relations. I repeat the same message I gave last year. We shall be making proposals to handle certain areas of foreign policy – not all – using qualified majority.

This is possible on the basis of the current Treaties and I believe the time has come to make use of this ‘lost treasure' of the Lisbon Treaty.

I also think, for that matter, that we should be able to decide on certain tax matters by qualified majority.

I would like to say a few words, Mr President, about the increasingly worrying way in which we air our disagreements. Heated exchanges amongst governments and sometimes institutions are becoming more and more common. But harsh or hurtful words will not get Europe anywhere.

What worries me is not only the tone used in political discourse. It is also true of the way some seek to shut down debate altogether by targeting media and journalists. Europe must always be a place where freedom of the press is sacrosanct. Too many of our journalists are intimidated, attacked, murdered. There is no democracy without a free press.

In general, ladies and gentlemen, we must do more to revive the lost art of compromise. Compromise is not a weakness. It does not mean sacrificing our convictions or stifling free debate that respects others' points of view.

The Commission will resist all attacks on the rule of law.We continue to be very concerned by the developments in some of our Member States. Article 7 must be applied whenever the rule of law is threatened.

In this regard, I must say that First Vice-President Timmermans is doing a remarkable but often lonely – too often lonely – job of defending the rule of law. The whole Commission, and I personally, support him fully.

We must be very clear on one point: judgements from the Court of Justice must be respected and implemented. The European Union is a community of law. Respecting the rule of law and abiding by Court decisions are not optional.

 

CONCLUSION

Mr President,

Ladies and Gentlemen, and for many of you: dear friends,

I started this speech – my last State of the Union though surely not my last speech – by talking about history. I spoke of both the events that have marked this Commission's time in office and of history writ large, the History of Europe.

We are all responsible for the Europe of today. And we must all take responsibility for the Europe of tomorrow.

Such is history: parliaments and Commissions come and go, Europe is here to stay. But for Europe to become what it must, we must keep several aspects in mind.

I want Europe to get off the side-lines of world affairs. Europe can no longer be a spectator or a mere commentator of international events. Europe must be an active player, an architect of tomorrow's world.

There is strong demand for Europe throughout the world. To meet such high demand, Europe will have to speak with one voice on the world stage. In the concert of nations, Europe's voice must ring clear in order to be heard. Federica Mogherini has made Europe's diplomacy more coherent than it has ever been before. But let us never slide back into the incoherence of competing and parallel national diplomacies. Europe diplomacy must be conducted in the singular. And our solidarity must be all-embracing.

I want us to do more to bring together the East and West of Europe. It is time we put an end to the sorry spectacle of a divided Europe. Our continent and those who brought an end to the Cold War deserve better.

I would like the European Union to take better care of its social dimension. Those that ignore the legitimate concerns of workers and small businesses undermine European unity. It is time we turned the good intentions that we proclaimed at the Gothenburg Social Summit into law.

I would like next year's elections to be a landmark for European democracy. I would like to see the Spitzenkandidaten (lead candidate) process – that small step forward for European democracy – repeated. For me, this process would be made all the more credible if we were to have transnational lists. I think that this should be done by 2024.

But above all, I would like us to reject unhealthy nationalism and embrace enlightened patriotism. We should never forget that the patriotism of the 21st Century is two-fold: both European and national, with one not excluding the other.

As the French philosophe Blaise Pascal said: I like things that go together. In order to stand on its own two feet, Europe must move forward as one. To love Europe, is it love its nations. To love your country is to love Europe. Patriotism is a virtue. Unchecked nationalism is riddled with both poison and deceit.

In short, we must remain true to ourselves.

The trees we plant today must provide shade for our great grand-children whether they hail from North or South, from East or West. To give them all they need to grow and breathe easily.

A few years ago, standing in this very same spot, I told you that Europe was the love of my life. I love Europe still and shall do so forever more.

Thank for your attention.
 \newpage\section{Speech 9 - von der Leyen - 2020-09-16}
\url{https://ec.europa.eu/commission/presscorner/detail/en/SPEECH_20_1655}\\[3mm]
Building the world we want to live in: A Union of vitality in a world of fragility

 

Dear President,

Honourable Members,

One of the most courageous minds of our times, Andrei Sakharov – a man so admired by this House - always spoke of his unshakeable belief in the hidden strength of the human spirit.

In these last six months, Europeans have shown how strong that human spirit really is. 

We saw it in the care workers who moved into nursing homes to look after the ill and the elderly.

In the doctors and nurses who became family members for those in their last breath.

In the front line workers who worked day after night, week after week, who took risks so most of us didn't have to.

We are inspired by their empathy, bravery and sense of duty – and I want to start this speech by paying tribute to them all.

Their stories also reveal a lot about the state of our world and the state of our Union.

They show the power of humanity and the sense of mourning which will live long in our society.

And they expose to us the fragility all around us.

A virus a thousand times smaller than a grain of sand exposed how delicate life can be.

It laid bare the strains on our health systems and the limits of a model that values wealth above wellbeing.

It brought into sharper focus the planetary fragility that we see every day through melting glaciers, burning forests and now through global pandemics.

It changed the very way we behave and communicate – keeping our arms at length, our faces behind masks.

It showed us just how fragile our community of values really is – and how quickly it can be called into question around the world and even here in our Union.

But people want to move out of this corona world, out of this fragility, out of uncertainty.  They are ready for change and they are ready to move on.

And this is the moment for Europe.

The moment for Europe to lead the way from this fragility towards a new vitality.And this is what I want to talk about today.

 

Honourable Members,

I say this because in the last months we have rediscovered the value of what we hold in common.

As individuals, we have all sacrificed a piece of our personal liberty for the safety of others.

And as a Union, we all shared a part of our sovereignty for the common good.

We turned fear and division between Member States into confidence in our Union.

We showed what is possible when we trust each other and trust our European institutions.

And with all of that, we choose to not only repair and recover for the here and now, but to shape a better way of living for the world of tomorrow.

This is NextGenerationEU. 

This is our opportunity to make change happen by design – not by disaster or by diktat from others in the world.

To emerge stronger by creating opportunities for the world of tomorrow and not just building contingencies for the world of yesterday.

We have everything we need to make this happen. We have shaken off the old excuses and home comforts that have always held us back. We have the vision, we have the plan, we have the investment.

It is now time to get to work.

This morning, I have sent a Letter of Intent to President Sassoli and Chancellor Merkel – on behalf of the German Presidency - outlining the Commission's plans for the year ahead.

I will not present every initiative today but I want to touch on what our Union must focus on in the next twelve months.

 

PULLING THROUGH TOGETHER: MAKING GOOD ON EUROPE'S PROMISE

Honourable members,

The people of Europe are still suffering.

It is a period of profound anxiety for millions who are concerned about the health of their families, the future of their jobs or simply just getting through until the end of the month.

The pandemic – and the uncertainty that goes with it – is not over. And the recovery is still in its early stage.

So our first priority is to pull each other through this. To be there for those that need it.

And thanks to our unique social market economy, Europe can do just that.

It is above all a human economy that protects us against the great risks of life - illness, ill-fortune, unemployment or poverty. It offers stability and helps us better absorb shocks. It creates opportunity and prosperity by promoting innovation, growth and fair competition.

Never before has that enduring promise of protection, stability and opportunity been more important than it is today.

Allow me to explain why.

First, Europe must continue to protect lives and livelihoods.

This is all the more important in the middle of a pandemic that shows no signs of running out of steam or intensity.

We know how quickly numbers can spiral out of control. So we must continue to handle this pandemic with extreme care, responsibility and unity.

In the last six months, our health systems and workers have produced miracles.

Every country has worked to do its best for its citizens.

And Europe has done more together than ever before.

When Member States closed borders, we created green lanes for goods.

When more than 600,000 European citizens were stranded all over the world, the EU brought them home.

When some countries introduced export bans for critical medical goods, we stopped that and ensured that critical medical supply could go where it was needed.

We worked with European industry to increase the production of masks, gloves, tests and ventilators.

Our Civil Protection Mechanism ensured that doctors from Romania could treat patients in Italy or that Latvia could send masks to its Baltic neighbours. 

And we achieved this without having full competences.

For me, it is crystal clear – we need to build a stronger European Health Union.

And to start making this a reality, we must now draw the first lessons from the health crisis.

We need to make our new EU4Health programme future proof.  This is why I had proposed to increase funding and I am grateful that this Parliament is ready to fight for more funding and remedy the cuts made by the European Council.

And we need to strengthen our crisis preparedness and management of cross-border health threats.

As a first step, we will propose to reinforce and empowerthe European Medicines Agencyand ECDC – our centre for disease prevention and control.

As a second step, we will build a European BARDA – an agency for biomedical advanced research and development. This new agency will support our capacity and readiness to respond to cross-border threats and emergencies – whether of natural or deliberate origin. We need strategic stockpiling to address supply chain dependencies, notably for pharmaceutical products.

As a third step, it is clearer than ever that we must discuss the question of health competences. And I think this is a noble and urgent task for the Conference on the Future of Europe.

And because this was a global crisis we need to learn the global lessons. This is why, along with Prime Minister Conte and the Italian G20 Presidency, I will convene a Global Health Summit next year in Italy.

This will show Europeans that our Union is there to protect all.

And this is exactly what we have done when it comes to workers.

When I took office, I vowed to create an instrument to protect workers and businesses from external shocks.

Because I knew from my experience as a Minister for Labour and Social Affairs that these schemes work. They keep people in jobs, skills in companies and SMEs in business. These SMEs are the motor of our economy and will be the engine of our recovery.

This is why the Commission created the SURE programme. And I want to thank this House for working on it in record time.

If Europe has so far avoided mass unemployment seen elsewhere, it is thanks in large part to the fact that around 40 million people applied for short-time work schemes.

This speed and unity of purpose means that 16 countries will soon receive almost 90 billion euros from SURE to support workers and companies.

From Lithuania to Spain, it will give peace of mind to families who need that income to put food on the table or to pay the rent.

And it will help protect millions of jobs, incomes and companies right across our Union.

This is real European solidarity in action. And it reflects the fact that in our Union the dignity of work must be sacred.

But the truth is that for too many people, work no longer pays.

Dumping wages destroys the dignity of work, penalises the entrepreneur who pays decent wages and distorts fair competition in the Single Market.

This is why the Commission will put forward a legal proposal to support Member States to set up a framework for minimum wages. Everyone must have access to minimum wages either through collective agreements or through statutory minimum wages.

I am a strong advocate of collective bargaining and the proposal will fully respect national competencies and traditions.

We have seen in many Member States how a well-negotiated minimum wage secures jobs and creates fairness – both for workers and for the companies who really value them.

Minimum wages work – and it is time that work paid.

The second promise of the social market economy is that of stability.

The European Union and its Member States responded to an unprecedented crisis with an unprecedented response.

By showing it was united and up to the task, Europe provided the stability our economies needed.

The Commission immediately triggered the general escape clause for the first time in our history.

We flexibilisedour European funds and State aid rules.

Authorising more than 3 trillion euro in support to companies and industry: From fishermen in Croatia and farmers in Greece, to SMEs in Italy and freelancers in Denmark.

The European Central Bank took decisive action through its PEPP programme.

The Commission proposed NextGenerationEU and a revamped budget in record time.

It combines investment with much needed reforms.

The Council endorsed it in record time.

This House is working towards voting on it with maximum speed.

For the first time – and for exceptional times - Europe has put in place its own common tools to complement national fiscal stabilisers.

This is a remarkable moment of unity for our Union. This is an achievement that we should take collective pride in.

Now is the time to hold our course.  We have all seen the forecasts. We can expect our economies to start moving again after a 12\% drop in GDP in the second quarter.

But as the virus lingers so does the uncertainty – here in Europe and around the world.  

So this is definitely not the time to withdraw support.

Our economies need continued policy support and a delicate balance will need to be struck between providing financial support and ensuring fiscal sustainability.

In the longer-term there is no greater way to stability and competitiveness than through a stronger Economic and Monetary Union.

Confidence in the euro has never been stronger.                                           

The historic agreement on NextGenerationEU shows the political backing that it has.

We must now use this opportunity to make structural reforms in our economies and complete the Capital Markets Union and the Banking Union.

Deep and liquid capital markets are essential to give businesses access to the finance they need to grow and invest in recovery and in the future.

And they are also a pre-requisite to further strengthen the international role of the euro. So let's get to work and finally complete this generational project.

 

Honourable Members, the third enduring promise is the promise of opportunity.

The pandemic reminded us of many things we may have forgotten or taken for granted.

We were reminded how linked our economies are and how crucial a fully functioning Single Market is to our prosperity and the way we do things.

The Single Market is all about opportunity - for a consumer to get value for money, a company to sell anywhere in Europe and for industry to drive its global competitiveness.

And for all of us, it is about the opportunity to make the most of the freedoms we cherish as Europeans. It gives our companies the scale they need to prosper and is a safe haven for them in times of trouble. We rely on it every day to make our lives easier – and it is critical for managing the crisis and recovering our strength.

Let's give it a boost.

We must tear down the barriers of the Single Market. We must cut red tape. We must step up implementation and enforcement. And we must restore the four freedoms – in full and as fast as possible.

The linchpin of this is a fully functioning Schengen area of free movement. We will work with Parliament and Member States to bring this high up our political agenda and we will propose a new strategy for the future of Schengen.

Based on this strong internal market, the European industry has long powered our economy, providing a stable living for millions and creating the social hubs around which our communities are built.

We presented our new industry strategy in March to ensure industry could lead the twin green and digital transition. The last six months have only accelerated that transformation – at a time when the global competitive landscape is fundamentally changing. This is why we will update our industry strategy in the first half of next year and adapt our competition framework which should also keep pace.

 

PROPELLING EUROPE FORWARD: BUILDING THE WORLD WE WANT TO LIVE IN

Honourable Members,

All of this will ensure Europe gets back to its feet. But as we pull through together, we must also propel ourselves forwards to the world of tomorrow.

There is no more urgent need for acceleration than when it comes to the future of our fragile planet.

While much of the world's activity froze during lockdowns and shutdowns, the planet continued to get dangerously hotter.

We see it all around us: from homes evacuated due to glacier collapse on the Mont Blanc, to fires burning through Oregon, to crops destroyed in Romania by the most severe drought in decades.

But we also saw nature come back into our lives.

We longed for green spaces and cleaner air for our mental health and our physical wellbeing.

We know change is needed – and we also know it is possible.

The European Green Deal is our blueprint to make that transformation.

At the heart of it is our mission to become the first climate-neutral continent by 2050.

But we will not get there with the status quo – we need to go faster and do things better.

We looked in-depth at every sector to see how fast we could go and how to do it in a responsible, evidence-based way.

We held a wide public consultation and conducted an extensive impact assessment.

On this basis, the European Commission is proposing to increase the 2030 target for emission reduction to at least 55\%. 

I recognise that this increase from 40 to 55 is too much for some, and not enough for others.

But our impact assessment clearly shows that our economy and industry can manage this

And they want it too. Just yesterday, 170 business leaders and investors – from SME's to some of the world's biggest companies - wrote to me calling on Europe to set a target of at least 55\%.

Our impact assessment clearly shows that meeting this target would put the EU firmly on track for climate neutrality by 2050 and for meeting our Paris Agreement obligations.

And if others follow our lead, the world will be able to keep warming below 1.5 degrees Celsius.

I am fully aware that many of our partners are far away from that – and I will come back to the Carbon Border Adjustment Mechanism later.

But for us, the 2030 target is ambitious, achievable, and beneficial for Europe.

We can do it.We have already shown we can do it.

While emissions dropped 25\% since 1990, our economy grew by more than 60\%.

The difference is we now have more technology, more expertise and more investment. And we are already embarking towards a circular economy with carbon neutral production.

We have more young people pushing for change. We have more proof that what is good for the climate is good for business and is good for us all. 

And we have a solemn promise to leave no one behind in this transformation. With our Just Transition Fund we will support the regions that have a bigger and more costly change to make.

We have it all. Now it's our responsibility to implement it all and make it happen.

 

Honourable Members,

Meeting this new target will reduce our energy import dependency, create millions of extra jobs and more than halve air pollution.

To get there, we must start now.

By next summer, we will revise all of our climate and energy legislation to make it “fit for 55”.

We will enhance emission trading, boost renewable energy, improve energy efficiency, reform energy taxation. 

But the mission of the European Green Deal involves much more than cutting emissions.

It is about making systemic modernisation across our economy, society and industry. It is about building a stronger world to live in.

Our current levels of consumption of raw materials, energy, water, food and land use are not sustainable.

We need to change how we treat nature, how we produce and consume, live and work, eat and heat, travel and transport.

So we will tackle everything from hazardous chemicals to deforestation to pollution.

This is a plan for a true recovery. It is an investment plan for Europe.

And this is where NextGenerationEU will make a real difference.

Firstly, 37\% of NextGenerationEU will be spent directly on our European Green Deal objectives.

And I will ensure that it also takes green financing to the next level.

We are world leaders in green finance and the largest issuer of green bonds worldwide. We are leading the way in developing a reliable EU Green Bond Standard.

And I can today announce that we will set a target of 30\% of NextGenerationEU's 750 billion euro to be raised through green bonds.

Secondly, NextGenerationEU should invest in lighthouse European projects with the biggest impact: hydrogen, renovation and 1 million electric charging points.

Allow me to explain how this could work: 

Two weeks ago in Sweden, a unique fossil-free steel pilot began test operations. It will replace coal with hydrogen to produce clean steel.

This shows the potential of hydrogen to support our industry with a new, clean, licence to operate.

I want NextGenerationEU to create new European Hydrogen Valleys to modernise our industries, power our vehicles and bring new life to rural areas.

The second example are the buildings we live and work in.

Our buildings generate 40\% of our emissions. They need to become less wasteful, less expensive and more sustainable.

And we know that the construction sector can even be turned from a carbon source into a carbon sink, if organic building materials like wood and smart technologies like AI are applied.

I want NextGenerationEU to kickstart a European renovation wave and make our Union a leader in the circular economy. 

But this is not just an environmental or economic project: it needs to be a new cultural project for Europe. Every movement has its own look and feel. And we need to give our systemic change its own distinct aesthetic – to match style with sustainability.

This is why we will set up a new European Bauhaus – a co-creation space where architects, artists, students, engineers, designers work together to make that happen.

This is NextGenerationEU. This is shaping the world we want to live in.

A world served by an economy that cuts emissions, boosts competitiveness, reduces energy poverty, creates rewarding jobs and improves quality of life.

A world where we use digital technologies to build a healthier, greener society.

This can only be achieved if we all do it together and I will insist that recovery plans don't just bring us out the crisis but also help us propel Europe forward to the world of tomorrow. 

 

Honourable Members,

Imagine for a moment life in this pandemic without digital in our lives. From staying in quarantine – isolated from family and community and cut off from the world of work – to major supply problems. It is in fact not so hard to imagine that this was the case 100 years ago during the last major pandemic.

A century later, modern technology has allowed young people to learn remotely and millions to work from home. They enabled companies to sell their products, factories to keep running and government to deliver crucial public services from afar.  We saw years' worth of digital innovation and transformation in the space of a few weeks.

We are reaching the limits of the things we can do in an analogue way. And this great acceleration is just beginning.

We must make this Europe's Digital Decade.

We need a common plan for digital Europe with clearly defined goals for 2030, such as for connectivity, skills and digital public services. And we need to follow clear principles: the right to privacy and connectivity, freedom of speech, free flow of data and cybersecurity.

But Europe must now lead the way on digital – or it will have to follow the way of others, who are setting these standards for us. This is why we must move fast.  

There are three areas on which I believe we need to focus.

First, data.

On personalized data - business to consumer -  Europe has been too slow and is now dependent on others.

This cannot happen with industrial data. And here the good news is that Europe is in the lead - we have the technology, and crucially we have the industry.

But the race is not yet won. The amount of industrial data in the world will quadruple in the next five years - and so will the opportunities that come with it.  We have to give our companies, SMEs, start-ups and researchers the opportunity to draw on their full potential. And industrial data is worth its weight in gold when it comes to developing new products and services.

But the reality is that 80\% of industrial data is still collected and never used. This is pure waste.

A real data economy, on the other hand, would be a powerful engine for innovation and new jobs.  And this is why we need to secure this data for Europe and make it widely accessible.  We need common data spaces - for example, in the energy or healthcare sectors. This will support innovation ecosystems in which universities, companies and researchers can access and collaborate on data.

And it is why we will build a European cloud as part of NextGenerationEU - based on GaiaX.

The second area we need to focus on is technology - and in particular artificial intelligence.

Whether it's precision farming in agriculture, more accurate medical diagnosis or safe autonomous driving - artificial intelligence will open up new worlds for us. But this world also needs rules.

We want a set of rules that puts people at the centre.  Algorithms must not be a black box and there must be clear rules if something goes wrong. The Commission will propose a law to this effect next year.

This includes control over our personal data which still have far too rarely today. Every time an App or website asks us to create a new digital identity or to easily log on via a big platform, we have no idea what happens to our data in reality.

That is why the Commission will soon propose a secure European e-identity. 

One that we trust and that any citizen can use anywhere in Europe to do anything from paying your taxes to renting a bicycle. A technology where we can control ourselves what data and how data is used.

The third point is the infrastructure.

Data connections must keep pace with the rapid speed of change.

If we are striving for a Europe of equal opportunities, it is unacceptable that 40\% of people in rural areas still do not have access to fast broadband connections.

These connections are now the prerequisite for home working, home learning, online shopping and, increasingly by the day, for new important services.  Without broadband connections, it is now barely possible to build or run a business effectively.

This is a huge opportunity and the prerequisite for revitalising rural areas. Only then can they fully exploit their potential and attract more people and investment.

The investment boost through NextGenerationEU is a unique chance to drive expansion to every village. This is why we want to focus our investments on secure connectivity, on the expansion of 5G, 6G and fiber.

NextGenerationEU is also a unique opportunity to develop a more coherent European approach to connectivity and digital infrastructure deployment.

None of this is an end in itself - it is about Europe's digital sovereignty, on a small and large scale.

In this spirit, I am pleased to announce an investment of 8 billion euros in the next generation of supercomputers - cutting-edge technology made in Europe.

And we want the European industry to develop our own next-generation microprocessor that will allow us to use the increasing data volumes energy-efficient and securely.

This is what Europe's Digital Decade is all about!

 

Honourable Members,

If Europe is to move forward and move fast, we must let go of our hesitancies.

This is about giving Europe more control over its future.

We have everything it takes to bring it to life. And the private sector is desperately waiting for this too.

There has never been a better time to invest in European tech companies with new digital hubs growing everywhere from Sofia to Lisbon to Katowice.  We have the people, the ideas and the strength as a Union to succeed.

And this is why we will invest 20\% of NextGenerationEU on digital.

We want to lead the way, the European way, to the Digital Age: based on our values, our strength, our global ambitions.

 

A VITAL EUROPE IN A FRAGILE WORLD

Honourable Members,

Europe is determined to use this transition to build the world we want to live in. And that of course extends well beyond our borders.

The pandemic has simultaneously shown both the fragility of the global system and the importance of cooperation to tackle collective challenges.

In the face of the crisis, some around the world choose to retreat into isolation. Others actively destabilise the system.

Europe chooses to reach out.

Our leadership is not about self-serving propaganda. It is not about Europe First. It is about being the first to seriously answer the call when it matters.

In the pandemic, European planes delivering thousands of tonnes of protective equipment landed everywhere from Sudan to Afghanistan, Somalia to Venezuela.

None of us will be safe until all of us are safe – wherever we live, whatever we have.

An accessible, affordable and safe vaccine is the world's most promising way to do that.

At the beginning of the pandemic, there was no funding, no global framework for a COVID vaccine – just the rush to be the first to get one.

This is the moment the EU stepped up to lead the global response. With civil society, the G20, WHO and others we brought more than 40 countries together to raise 16 billion euro to finance research on vaccines, tests and treatments for the whole world. This is the EU's unmatched convening power in action.  

But it is not enough to find a vaccine. We need to make sure that European citizens and those around the world have access to it.

Just this month, the EU joined the COVAX global facility and contributed 400 million euro to help ensure that safe vaccines are available not only for those who can afford it – but for everyone who needs it. 

Vaccine nationalism puts lives at risk. Vaccine cooperation saves them.

 

Honourable Members,

We are firm believers in the strength and value of cooperating in international bodies

It is with a strong United Nations that we can find long-term solutions for crises like Libya or Syria.

It is with a strong World Health Organisation that we can better prepare and respond to global pandemics or local outbreaks – be it Corona or Ebola.

And it is with a strong World Trade Organisation that we can ensure fair competition for all.

But the truth is also that the need to revitalise and reform the multilateral system has never been so urgent. Our global system has grown into a creeping paralysis. Major powers are either pulling out of institutions or taking them hostage for their own interests.

Neither road will lead us anywhere. Yes, we want change. But change by design – not by destruction.

And this is why I want the EU to lead reforms of the WTO and WHO so they are fit for today's world. 

But we know that multilateral reforms take time and in the meantime the world will not stop.

Without any doubt, there is a clear need for Europe to take clear positions and quick actions on global affairs.

Two days ago, the latest EU-China leaders meeting took place. 

The relationship between the European Union and China is simultaneously one of the most strategically important and one of the most challenging we have.

From the outset I have said China is a negotiating partner, an economic competitor and a systemic rival.

We have interests in common on issues such as climate change – and China has shown it is willing to engage through a high-level dialogue. But we expect China to live up to its commitments in the Paris Agreement and lead by example.

There is still hard work to do on fair market access for European companies, reciprocity and overcapacity. We continue to have an unbalanced trade and investment partnership.

And there is no doubt that we promote very different systems of governance and society. We believe in the universal value of democracy and the rights of the individual. 

Europe is not without issues – think for example of anti-semitism. But we discuss them publicly. Criticism and opposition are not only accepted but are legally protected.

So we must always call out human rights abuses whenever and wherever they occur – be it on Hong Kong or with the Uyghurs.

But what holds us back? Why are even simple statements on EU values delayed, watered down or held hostage for other motives?

When Member States say Europe is too slow, I say to them be courageous and finally move to qualified majority voting – at least on human rights and sanctions implementation. 

This House has called many times for a European Magnitsky Act – and I can announce that we will now come forward with a proposal.

We need to complete our toolbox.

 

Honourable Members,

Be it in Hong Kong, Moscow or Minsk: Europe must take a clear and swift position.

I want to say it loud and clear: the European Union is on the side of the people of Belarus.

We have all been moved by the immense courage of those peacefully gathering in Independence Square or taking part in the fearless women's march.

The elections that brought them into the street were neither free nor fair. And the brutal response by the government ever since has been shameful.

The people of Belarus must be free to decide their own future for themselves. They are not pieces on someone else's chess board.

To those that advocate closer ties with Russia, I say that the poisoning of Alexei Navalny with an advanced chemical agent is not a one off. We have seen the pattern in Georgia and Ukraine, Syria and Salisbury – and in election meddling around the world. This pattern is not changing – and no pipeline will change that.

Turkey is and will always be an important neighbour. But while we are close together on the map, the distance between us appears to be growing. Yes, Turkey is in a troubled neighbourhood. And yes, it is hosting millions of refugees, for which we support them with considerable funding. But none of this is justification for attempts to intimidate its neighbours.

Our Member States, Cyprus and Greece, can always count on Europe's full solidarity on protecting their legitimate sovereignty rights.  

De-escalation in the Eastern Mediterranean is in our mutual interest. The return of exploratory vessels to Turkish ports in the past few days is a positive step in this direction. This is necessary to create the much needed space for dialogue. Refraining from unilateral actions and resuming talks in genuine good faith is the only path forward.  The only path to stability and lasting solutions.

 

Honourable Members,

As well as responding more assertively to global events, Europe must deepen and refine its partnerships with its friends and allies.

And this starts with revitalising our most enduring of partnerships.

We might not always agree with recent decisions by the White House. But we will always cherish the transatlantic alliance – based on shared values and history, and an unbreakable bond between our people.

So whatever may happen later this year, we are ready to build a new transatlantic agenda. To strengthen our bilateral partnership – be it on trade, tech or taxation.

And we are ready to work together on reforming the international system we built together, jointly with like-minded partners. For our own interests and the interest of the common good.

We need new beginnings with old friends – on both of sides of the Atlantic and on both sides of the Channel.

The scenes in this very room when we held hands and said goodbye with Auld Lang Syne spoke a thousand words. They showed an affection for the British people that will never fade.

But with every day that passes the chances of a timely agreement do start to fade.

Negotiations are always difficult. We are used to that.

And the Commission has the best and most experienced negotiator, Michel Barnier, to navigate us through.

But talks have not progressed as we would have wished. And that leaves us very little time.

As ever, this House will be the first to know and will have the last say. And I can assure you we will continue to update you throughout, just as we did with the Withdrawal Agreement.

That agreement took three years to negotiate and we worked relentlessly on it. Line by line, word by word.

And together we succeeded. The result guarantees our citizens' rights, financial interests, the integrity of the Single Market – and crucially the Good Friday Agreement.

The EU and the UK jointly agreed it was the best and only way for ensuring peace on the island of Ireland.

And we will never backtrack on that. This agreement has been ratified by this House and the House of Commons.

It cannot be unilaterally changed, disregarded or dis-applied. This a matter of law, trust and good faith.

And that is not just me saying it – I remind you of the words of Margaret Thatcher:

“Britain does not break Treaties. It would be bad for Britain, bad for relations with the rest of the world, and bad for any future Treaty on trade”.

This was true then, and it is true today.

Trust is the foundation of any strong partnership.

And Europe will always be ready to build strong partnerships with our closest neighbours. 

That starts with the Western Balkans.

The decision six months ago to open accession negotiations with Albania and North Macedonia was truly historic.

Indeed, the future of the whole region lies in the EU. We share the same history, we share the same destiny.

The Western Balkans are part of Europe - and not just a stopover on the Silk Road. 

We will soon present an economic recovery package for the Western Balkans focusing on a number of regional investment initiatives.

And we will also be there for the Eastern Partnership countries and our partners in the southern neighbourhood– to help create jobs and kickstart their economies.

When I came into office, I chose for the very first trip outside the European Union, to visit the African Union, and it was a natural choice. It was a natural choice and it was a clear message, because we are not just neighbours, we are natural partners.

Three months later, I returned with my entire College to set our priorities for our new strategy with Africa. It is a partnership of equals, where both sides share opportunities and responsibilities.

Africa will be a key partner in building the world we want to live in – whether on climate, digital or trade.

 

Honourable Members,

We will continue to believe in open and fair trade across the world.  Not as an end in itself – but as a way to deliver prosperity at home and promote our values and standards. More than 600,000 jobs in Europe are tied to trade with Japan. And our recent agreement with Vietnam alone helped secure historic labour rights for millions of workers in the country.

We will use our diplomatic strength and economic clout to broker agreements that make a difference – such as designating maritime protected areas in the Antarctica. This would be one of the biggest acts of environmental protection in history.

We will form high ambition coalitions on issues such as digital ethics or fighting deforestation – and develop partnerships with all like-minded partners – from Asian democracies to Australia, Africa, the Americas and anyone else who wants to join.

We will work for just globalisation. But we cannot take this for granted. We must insist on fairness and a level playing field. And Europe will move forward – alone or with partners that want to join.

We are for example working on a Carbon Border Adjustment Mechanism.

Carbon must have its price – because nature cannot pay the price anymore.

This Carbon Border Adjustment Mechanism should motivate foreign producers and EU importers to reduce their carbon emissions, while ensuring that we level the playing field in a WTO-compatible way.

The same principle applies to digital taxation. We will spare no effort to reach agreement in the framework of OECD and G20. But let there be no doubt: should an agreement fall short of a fair tax system that provides long-term sustainable revenues, Europe will come forward with a proposal early next year.

I want Europe to be a global advocate for fairness.

 

A NEW VITALITY FOR EUROPE

Honourable Members,

If Europe is to play this vital role in the world – it must also create a new vitality internally.

And to move forward we must now overcome the differences that have held us back.

The historic agreement on NextGenerationEU shows that it can be done. The speed with which we took decisions on fiscal rules, state aid or for SURE – all this shows it can be done.

So let's do it.

Migration is an issue that has been discussed long enough.

Migration has always been a fact for Europe – and it will always be. Throughout centuries, it has defined our societies, enriched our cultures and shaped many of our lives. And this will always be the case.

As we all know, the 2015 migration crisis caused many deep divisions between Member States – with some of those scars still healing today.

A lot has been done since. But a lot is still missing.

If we are all ready to make compromises – without compromising on our principles – we can find that solution.

Next week, the Commission will put forward its New Pact on Migration.

We will take a human and humane approach. Saving lives at sea is not optional. And those countries who fulfil their legal and moral duties or are more exposed than others, must be able to rely on the solidarity of our whole European Union.

We will ensure a closer link between asylum and return. We have to make a clear distinction between those who have the right to stay and those who do not.

We will take action to fight smugglers, strengthen external borders, deepen external partnerships and create legal pathways.

And we will make sure that people who have the right to stay are integrated and made to feel welcome.

They have a future to build – and skills, energy and talent.

I think of Suadd, the teenage Syrian refugee who arrived in Europe dreaming of being a doctor. Within three years she was awarded a prestigious scholarship from the Royal College of Surgeons in Ireland.

I think of the Libyan and Somalian refugee doctors who offered their medical skills the moment the pandemic struck in France.

Honourable Members, if we think about what they have overcome and what they have achieved, then we simply must be able to manage the question of migration together.

The images of the Moria camp are a painful reminder of the need for Europe to come together.

Everybody has to step up here and take responsibility – and the Commission will do just that. The Commission is now working on a plan for a joint pilot with the Greek authorities for a new camp on Lesvos. We can assist with asylum and return processes and significantly improve the conditions for the refugees.

But I want to be clear: if we step up, then I expect all Member States to step up too.

Migration is a European challenge and all of Europe must do its part.

We must rebuild the trust amongst us and move forward together.

And this trust is at the very heart of our Union and the way we do things together.

It is anchored in our founding values, our democracies and in our Community of Law – as Walter Hallstein used to call it.

This is not an abstract term. The rule of law helps protect people from the rule of the powerful. It is the guarantor of our most basic of every day rights and freedoms. It allows us to give our opinion and be informed by a free press.

Before the end of the month, the Commission will adopt the first annual rule of law report covering all Member States.

It is a preventive tool for early detection of challenges and for finding solutions.

I want this to be a starting point for Commission, Parliament and Member States to ensure there is no backsliding.

The Commission attaches the highest importance to the rule of law. This is why we will ensure that money from our budget and NextGenerationEU is protected against any kind of fraud, corruption and conflict of interest. This is non-negotiable.

But the last months have also reminded us how fragile it can be. We have a duty to always be vigilant to care and nurture for the rule of law.

Breaches of the rule of law cannot be tolerated. I will continue to defend it and the integrity of our European institutions. Be it about the primacy of European law, the freedom of the press, the independence of the judiciary or the sale of golden passports. European values are not for sale.

 

Honourable Members,

These values are more important than ever. I say that because when I think about the state of our Union, I am reminded of the words of John Hume – one of the great Europeans who sadly passed away this year.

If so many people live in peace today on the island of Ireland, it is in large part because of his unwavering belief in humanity and conflict resolution.

He used to say that conflict was about difference and that peace was about respect for difference.

And as he so rightly reminded this House in 1998: “The European visionaries decided that difference is not a threat, difference is natural. Difference is the essence of humanity”.

These words are just as important today as they ever have been.

Because when we look around, we ask ourselves, where is the essence of humanity when three children in Wisconsin watch their father shot by police while they sit in the car?

We ask where is the essence of humanity when anti-semitic carnival costumes openly parade on our streets?

Where is the essence of humanity when every single day Roma people are excluded from society and others are held back simply because of the colour of their skin or their religious belief?

I am proud to live in Europe, in this open society of values and diversity.

But even here in this Union – these stories are a daily reality for so many people.

And this reminds us that progress on fighting racism and hate is fragile – it is hard won but very easily lost.

So now is the moment to make change.

To build a truly anti-racist Union – that goes from condemnation to action. 

And the Commission is putting forward an action plan to start making that happen.

As part of this, we will propose to extend the list of EU crimes to all forms of hate crime and hate speech – whether because of race, religion, gender or sexuality.

Hate is hate – and no one should have to put up with it.

We will strengthen our racial equality laws where there are gaps.

We will use our budget to address discrimination in areas such as employment, housing or healthcare.

We will get tougher on enforcement when implementation lags behind.

Because in this Union, fighting racism will never be optional.

We will improve education and knowledge on the historical, cultural causes of racism.

We will tackle unconscious bias that exists in people, institutions and even in algorithms.

And we will appoint the Commission's first-ever anti-racism coordinator to keep this at the top of our agenda and to work directly with people, civil society and institutions.   

 

Honourable Members,

I will not rest when it comes to building a Union of equality.

A Union where you can be who you are and love who you want – without fear of recrimination or discrimination.

Because being yourself is not your ideology.

It's your identity.

And no one can ever take it away.

So I want to be crystal clear – LGBTQI-free zones are humanity free zones. And they have no place in our Union.

And to make sure that we support the whole community, the Commission will soon put forward a strategy to strengthen LGBTQI rights.

As part of this, I will also push for mutual recognition of family relations in the EU. If you are parent in one country, you are parent in every country.  

 

CONCLUSION

Honourable Members,

This is the world we want to live in.

Where we are united in diversity and adversity. Where we work together to overcome our differences – and pull each other through when times are hard.

Where we build today the healthier, stronger and more respectful world we want our children to live in tomorrow. 

But while we try to teach our children about life, our children are busy teaching us what life is about.

The last year has shown us just how true this really is.

We could speak of the millions of young people asking for change for a better planet.
Or of the hundreds of thousands of beautiful rainbows of solidarity posted in the windows of Europe by our children.

But there is one image that stuck in my mind from the last six difficult months. An image that captures the world through the eyes of our children.

It is the image of Carola and Vittoria. The two young girls playing tennis between the rooftops of Liguria, Italy.

It is not just the courage and talent of the girls that sticks out.

It is the lesson behind it. About not allowing obstacles stand in your way, about not letting conventions hold you back, about seizing the moment.

This is what Carola, Vittoria and all the young people of Europe teach us about life every day. It is what Europe's next generation is all about. This is NextGenerationEU.

This year, Europe took a leaf out of their book and took a leap forward together.

When we had to find a way forward for our future, we did not allow old conventions hold us back. 

When we felt fragility around us, we seized the moment to breathe new vitality into our Union.

When we had a choice to go it alone like we have done in the past, we used the combined strength of the 27 to give all 27 a chance for the future.

We showed that we are in this together and we will get out of this together.

Honourable Members,

The future will be what we make it. And Europe will be what we want it to be.

So let's stop talking it down. And let's get to work for it. Let's make it strong. And let's build the world we want to live in.

Long live Europe!
 \newpage\section{Speech 10 - von der Leyen - 2021-09-15}
\url{https://ec.europa.eu/commission/presscorner/detail/en/SPEECH_21_4701}\\[3mm]
STRENGTHENING THE SOUL OF OUR UNION

 

Introduction

 

Mr President,
Honourable Members,

Many are the people who feel their lives have been on pause while the world has been on fast forward.

The speed of events and the enormity of the challenges are sometimes difficult to grasp.

This has also been a time of soul-searching. From people reassessing their own lives to wider debates on sharing vaccines and on shared values.

But as I look back on this past year, if I look at the state of the Union today, I see a strong soul in everything that we do.

It was Robert Schuman who said: Europe needs a soul, an ideal, and the political will to serve this ideal.

Europe has brought those words to life in the last twelve months.

In the biggest global health crisis for a century, we chose to go it together so that every part of Europe got the same access to a life-saving vaccine.

In the deepest global economic crisis for decades, we chose to go it together with NextGenerationEU.

And in the gravest planetary crisis of all time, again we chose to go it together with the European Green Deal.

We did that together as Commission, as Parliament, as 27 Member States. As one Europe. And we can be proud of it.

But corona times are not over.

There is still much grief in our society as the pandemic lingers. There are hearts we can never mend, life stories we can never finish and time we can never give back to our young. We face new and enduring challenges in a world recovering – and fracturing – unevenly. 

So there is no question: the next year will be yet another test of character.

But I believe that it is when you are tested that your spirit – your soul - truly shines through.

As I look across our Union, I know that Europe will pass that test.

And what gives me that confidence is the inspiration we can draw from Europe's young people.

Because our youth put meaning into empathy and solidarity.
They believe we have a responsibility towards the planet.
And while they are anxious about the future, they are determined to make it better. 

Our Union will be stronger if it is more like our next generation: reflective, determined and caring.  Grounded in values and bold in action.

This spirit will be more important than ever over the next twelve months. This is the message in the Letter of Intent I sent this morning to President Sassoli and Prime Minister Janša to outline our priorities for the year ahead.

 

A EUROPE UNITED THROUGH ADVERSITY AND RECOVERY

Honourable Members,

A year is a long time in a pandemic.

When I stood in front of you 12 months ago, I did not know when – or even if – we would have a safe and effective vaccine against COVID-19.

But today, and against all critics, Europe is among the world leaders.

More than 70 per cent of adults in the EU are fully vaccinated.  We were the only ones to share half of our vaccine production with the rest of the world. We delivered more than 700 million doses to the European people, and we delivered more than another 700 million doses to the rest of the world, to more than 130 countries.

We are the only region in the world to achieve that.

A pandemic is a marathon, not a sprint.

We followed the science.
We delivered to Europe. We delivered to the world.
We did it the right way, because we did it the European way. And it worked!

But while we have every reason to be confident, we have no reason to be complacent.

Our first – and most urgent – priority is to speed up global vaccination.

With less than 1\% of global doses administered in low-income countries, the scale of injustice and the level of urgency are obvious. This is one of the great geopolitical issues of our time.

Team Europe is investing one billion Euro to ramp up mRNA production capacity in Africa. We have already committed to share 250 million doses.

I can announce today that the Commission will add a new donation of another 200 million doses by the middle of next year.

This is an investment in solidarity – but also in global health.

The second priority is to continue our efforts here in Europe.

We see worrisome divergences in vaccination rates in our Union.

So we need to keep up the momentum.

And Europe is ready. We have 1.8 billion additional doses secured. This is enough for us and our neighbourhood when booster shots are needed. Let's do everything possible to ensure that this does not turn into a pandemic of the unvaccinated.

The final priority is to strengthen our pandemic preparedness.

Last year, I said it was time to build a European Health Union. Today we are delivering. With our proposal we get the HERA authority up and running.

This will be a huge asset to deal with future health threats earlier and better.

We have the innovation and scientific capacity, the private sector knowledge, we have competent national authorities. And now we need to bring all of that together, including massive funding. 

So I am proposing a new health preparedness and resilience mission for the whole of the EU. And it should be backed up by Team Europe investment of EUR 50 billion by 2027. 

To make sure that no virus will ever turn a local epidemic into a global pandemic. There is no better return on investment than that. 

 

Honourable Members,

The work on the European Health Union is a big step forward. And I want to thank this House for your support.

We have shown that when we act together, we are able to act fast.

Take the EU digital certificate:

Today more than 400 million certificates have been generated across Europe. 42 countries in 4 continents are plugged in.

We proposed it in March.

You pushed it!
Three months later it was up and running.

Thanks to this joint effort, while the rest of the world talked about it, Europe just did it.

We did a lot of things right. We moved fast to create SURE. This supported over 31 million workers and 2.5 million companies across Europe.

We learned the lessons from the past when we were too divided and too delayed.

And the difference is stark: last time it took 8 years for the Eurozone GDP to get back to pre-crisis levels. 

This time we expect 19 countries to be at pre-pandemic levels this year with the rest following next. Growth in the euro area outpaced both the US and China in the last quarter.

But this is only the beginning. And the lessons from the financial crisis should serve as a cautionary tale. At that time, Europe declared victory too soon and we paid the price for that. And we will not repeat the same mistake.

The good news is that with NextGenerationEU we will now invest in both short-term recovery and long-term prosperity.

We will address structural issues in our economy: from labour market reforms in Spain, to pension reforms in Slovenia or tax reform in Austria.

In an unprecedented manner, we will invest in 5G and fibre. But equally important is the investment in digital skills. This task needs leaders' attention and a structured dialogue at top-level.

Our response provides a clear direction to markets and investors alike.

But, as we look ahead, we also need to reflect on how the crisis has affected the shape of our economy – from increased debt, to uneven impact on different sectors, or new ways of working.

To do that, the Commission will relaunch the discussion on the Economic Governance Review in the coming weeks. The aim is to build a consensus on the way forward well in time for 2023.

 

 

 

Honourable Members,

We will soon celebrate 30 years of the Single Market. For 30 years it has been the great enabler of progress and prosperity in Europe.

At the outset of the pandemic, we defended it against the pressures of erosion and fragmentation. For our recovery, the Single Market is the driver of good jobs and competitiveness.

That is particularly important in the digital single market. 

We have made ambitious proposals in the last year.

To contain the gatekeeper power of major platforms;

To underpin the democratic responsibility of those platforms;

To foster innovation;

To channel the power of artificial intelligence.

 

Digital is the make-or-break issue. And Member States share that view. Digital spending in NextGenerationEU will even overshoot the 20\% target.

That reflects the importance of investing in our European tech sovereignty. We have to double down to shape our digital transformation according to our own rules and values.

Allow me to focus on semi-conductors, those tiny chips that make everything work: from smartphones and electric scooters to trains or entire smart factories.

There is no digital without chips. And while we speak, whole production lines are already working at reduced speed - despite growing demand - because of a shortage of semi-conductors.

But while global demand has exploded, Europe's share across the entire value chain, from design to manufacturing capacity has shrunk. We depend on state-of-the-art chips manufactured in Asia.

So this is not just a matter of our competitiveness. This is also a matter of tech sovereignty. So let's put all of our focus on it.

We will present a new European Chips Act. We need to link together our world-class research, design and testing capacities. We need to coordinate EU and national investment along the value chain.

The aim is to jointly create a state-of-the-art European chip ecosystem, including production. That ensures our security of supply and will develop new markets for ground-breaking European tech.

Yes, this is a daunting task. And I know that some claim it cannot be done.

But they said the same thing about Galileo 20 years ago.

And look what happened. We got our act together. Today European satellites provide the navigation system for more than 2 billion smartphones worldwide. We are world leaders. So let's be bold again, this time with semi-conductors.

 

Honourable Members,

The pandemic has left deep scars that have also left their mark on our social market economy.

For nights on end, we all stood at our windows and doors to applaud critical workers. We felt how much we relied on all those women and men who work for lower wages, fewer protections and less security. 

The applause may have faded away but the strength of feeling cannot.

This is why the implementation of the European Pillar of Social Rights is so important – to ensure decent jobs, fairer working conditions, better healthcare and better balance in people's lives.

If the pandemic taught us one thing, it is that time is precious. And caring for someone you love is the most precious time of all.

We will come forward with a new European Care Strategy to support men and women in finding the best care and the best life balance for them.

But social fairness is not just a question of time. It is also a question of fair taxation.

In our social market economy, it is good for companies to make profits. And they make profits thanks to the quality of our infrastructure, social security and education systems.  So the very least we can expect is that they pay their fair share.

This is why we will continue to crack down on tax avoidance and evasion.
We will put forward a new initiative to address those hiding profits behind shell entities.
And we will do everything in our power to seal the historic global deal on minimum taxation.

Asking big companies to pay the right amount of tax is not only a question of public finances, but above all a question of basic fairness.

 

Honourable Members,

We have all benefited from the principles of our European social market economy – and we must make sure that the next generation can do so to build their future.

This is our most educated, talented and motivated generation. And it has missed out on so much to keep others safe.

Being young is normally a time of discovery, of creating new experiences. A time to meet lifelong friends, to find your own path.  And what did we ask this generation to do? To keep their social distance, to stay locked down and to do school from home. For more than a year.

This is why everything that we do – from the European Green Deal to NextGenerationEU – is about protecting their future. 

That is also why NextGenerationEU must be funded by the new own resources that we are working on.

But we must also caution against creating new divides. Because Europe needs all of its youth.

We must step up our support to those who fall into the gaps – those not in any kind of employment, education or training.

For them, we will put in place a new programme, ALMA.

ALMA will help these young Europeans to find temporary work experience in another Member State.

Because they too deserve an experience like Erasmus. To gain skills, to create bonds and help forge their own European identity.

But if we are to shape our Union in their mould, young people must be able to shape Europe's future.  Our Union needs a soul and a vision they can connect to.

Or as Jacques Delors asked:  How can we ever build Europe if young people do not see it as a collective project and a vision of their own future?

This is why we will propose to make 2022 the Year of European Youth. A year dedicated to empowering those who have dedicated so much to others.

And it is why we will make sure that young people can help lead the debate in the Conference on the Future of Europe.

This is their future and this must be their Conference too.

And as we said when we took office, the Commission will be ready to follow up on what is agreed by the Conference. 

 

A EUROPE UNITED IN RESPONSIBILITY

Honourable Members,

This is a generation with a conscience. They are pushing us to go further and faster to tackle the climate crisis.

And events of the summer only served to explain why. We saw floods in Belgium and Germany. And wildfires burning from the Greek islands to the hills in France.

And if we don't believe our own eyes, we only have to follow the science.

The UN recently published the IPCC report, the Intergovernmental Panel on Climate Change. It is the authority on the science of climate change.

The report leaves no doubt. Climate change is man-made. But since it is man-made, we can do something about it.

As I heard it said recently:  It's warming. It's us. We're sure. It's bad. But we can fix it.

And change is already happening. 

More electric vehicles than diesel cars were registered in Germany in the first half of this year.  Poland is now the EU's largest exporter of car batteries and electric buses. Or take the New European Bauhaus that led to an explosion of creativity of architects, designers, engineers across our Union.

So clearly something is on the move.

And this is what the European Green Deal is all about.

In my speech last year, I announced our target of at least 55\% emission reduction by 2030.
Since then we have together turned our climate goals into legal obligations.
And we are the first major economy to present comprehensive legislation in order to get it done.

You have seen the complexity of the detail. But the goal is simple. We will put a price on pollution. We will clean the energy we use. We will have smarter cars and cleaner airplanes.

And we will make sure that higher climate ambition comes with more social ambition. This must be a fair green transition. This is why we proposed a new Social Climate Fund to tackle the energy poverty that already 34 million Europeans suffer from.

I count on both Parliament and Member States to keep the package and to keep the ambition together.

When it comes to climate change and the nature crisis, Europe can do a lot. And it will support others. I am proud to announce today that the EU will double its external funding for biodiversity, in particular for the most vulnerable countries.

But Europe cannot do it alone. 

The COP26 in Glasgow will be a moment of truth for the global community.

Major economies – from the US to Japan – have set ambitions for climate neutrality in 2050 or shortly after. These need now to be backed up by concrete plans in time for Glasgow. Because current commitments for 2030 will not keep global warming to 1.5°C within reach.

Every country has a responsibility!

The goals that President Xi has set for China are encouraging. But we call for that same leadership on setting out how China will get there. The world would be relieved if they showed they could peak emissions by mid-decade - and move away from coal at home and abroad.

But while every country has a responsibility, major economies do have a special duty to the least developed and most vulnerable countries. Climate finance is essential for them - both for mitigation and adaptation.

In Mexico and in Paris, the major economies committed to provide 100 billion dollars a year until 2025 to the least developed and most vulnerable countries.

We deliver on our commitment. Team Europe contributes 25 billion dollars per year. But others still leave a gaping hole towards reaching the global target.

Closing that gap will increase the chance of success at Glasgow.

My message today is that Europe is ready to do more. We will now propose an additional 4 billion euro for climate finance until 2027. But we expect the United States and our partners to step up too.

Closing the climate finance gap together – the US and the EU – would be a strong signal for global climate leadership. It is time to deliver.

 

Honourable Members,

This climate and economic leadership is central to Europe's global and security objectives.

It also reflects a wider shift in world affairs at a time of transition towards a new international order.

We are entering a new era of hyper-competitiveness.

An era in which some stop at nothing to gain influence: from vaccine promises and high-interest loans, to missiles and misinformation.

An era of regional rivalries and major powers refocusing their attention towards each other.

Recent events in Afghanistan are not the cause of this change – but they are a symptom of it.

And first and foremost, I want to be clear. We stand by the Afghan people. The women and children, prosecutors, journalists and human rights defenders.

I think in particular of women judges who are now in hiding from the men they jailed. They have been put at risk for their contribution to justice and the rule of law. We must support them and we will coordinate all efforts with Member States to bring them to safety.

And we must continue supporting all Afghans in the country and in neighbouring countries. We must do everything to avert the real risk of a major famine and humanitarian disaster. And we will do our part. We will increase again humanitarian aid for Afghanistan by 100 million euro.

This will be part of a new, wider Afghan Support Package that we will present in the next weeks to combine all of our efforts.

 

Honourable Members,

Witnessing events unfold in Afghanistan was profoundly painful for all the families of fallen servicemen and servicewomen.

We bow to the sacrifice of those soldiers, diplomats and aid workers who laid down their lives.

To make sure that their service will never be in vain, we have to reflect on how this mission could end so abruptly.

There are deeply troubling questions that allies will have to tackle within NATO.

But there is simply no security and defence issue where less cooperation is the answer. We need to invest in our joint partnership and to draw on each side's unique strength.

This is why we are working with Secretary-General Jens Stoltenberg on a new EU-NATO Joint Declaration to be presented before the end of the year.

But this is only one part of the equation.

Europe can – and clearly should – be able and willing to do more on its own. But if we are to do more, we first need to explain why. I see three broad categories.

First, we need to provide stability in our neighbourhood and across different regions.

We are connected to the world by narrow straits, stormy seas and vast land borders. Because of that geography, Europe knows better than anyone that if you don't deal in time with the crisis abroad, the crisis comes to you.

Secondly, the nature of the threats we face is evolving rapidly: from hybrid or cyber-attacks to the growing arms race in space.

Disruptive technology has been a great equaliser in the way power can be used today by rogue states or non-state groups. 

You no longer need armies and missiles to cause mass damage. You can paralyse industrial plants, city administrations and hospitals – all you need is your laptop.  You can disrupt entire elections with a smartphone and an internet connection.

The third reason is that the European Union is a unique security provider. There will be missions where NATO or the UN will not be present, but where the EU should be.

On the ground, our soldiers work side-by-side with police officers, lawyers and doctors, with humanitarian workers and human rights defenders, with teachers and engineers.

We can combine military and civilian, along with diplomacy and development – and we have a long history in building and protecting peace.

The good news is that over the past years, we have started to develop a European defence ecosystem.

But what we need is the European Defence Union.

In the last weeks, there have been many discussions on expeditionary forces. On what type and how many we need: battlegroups or EU entry forces.

This is no doubt part of the debate – and I believe it will be part of the solution.

But the more fundamental issue is why this has not worked in the past.

You can have the most advanced forces in the world – but if you are never prepared to use them - of what use are they? 

What has held us back until now is not just a shortfall of capacity – it is the lack of political will.

And if we develop this political will, there is a lot that we can do at EU level.

Allow me to give you three concrete examples:

First, we need to build the foundation for collective decision-making – this is what I call situational awareness.

We fall short if Member States active in the same region, do not share their information on the European level. It is vital that we improve intelligence cooperation.

But this is not just about intelligence in the narrow sense.

It is about bringing together the knowledge from all services and all sources. From space to police trainers, from open source to development agencies. Their work gives us a unique scope and depth of knowledge.

It is out there!

But we can only use that, to make informed decisions if we have the full picture. And this is currently not the case. We have the knowledge, but it is disjoined. Information is fragmented.

This is why the EU could consider its own Joint Situational Awareness Centre to fuse all the different pieces of information. 

And to be better prepared, to be fully informed and to be able to decide.

Secondly, we need to improve interoperability. This is why we are already investing in common European platforms, from fighter jets, to drones and cyber.

But we have to keep thinking of new ways to use all possible synergies.  One example could be to consider waiving VAT when buying defence equipment developed and produced in Europe.

This would not only increase our interoperability, but also decrease our dependencies of today.

Third, we cannot talk about defence without talking about cyber. If everything is connected, everything can be hacked. Given that resources are scarce, we have to bundle our forces. And we should not just be satisfied to address the cyber threat, but also strive to become a leader in cyber security.

It should be here in Europe where cyber defence tools are developed. This is why we need a European Cyber Defence Policy, including legislation on common standards under a new European Cyber Resilience Act.

So, we can do a lot at EU level. But Member States need to do more too.

This starts with a common assessment of the threats we face and a common approach to dealing with them. The upcoming Strategic Compass is a key process of this discussion.  

And we need to decide how we can use all of the possibilities that are already in the Treaty.

This is why, under the French Presidency, President Macron and I will convene a Summit on European defence.

It is time for Europe to step up to the next level.

 

Honourable Members,

In a more contested world, protecting your interests is not only about defending yourself.

It is about forging strong and reliable partnerships. This is not a luxury – it is essential for our future stability, security and prosperity.

This work starts by deepening our partnership with our closest allies.

With the US we will develop our new agenda for global change – from the new Trade and Technology Council to health security and sustainability. 

The EU and the US will always be stronger – together.

The same is true of our neighbours in the Western Balkans.

Before the end of the month, I will travel to the region to send a strong signal of our commitment to the accession process. We owe it to all those young people who believe in a European future.

This is why we are ramping up our support through our new investment and economic plan, worth around a third of the region's GDP. Because an investment in the future of the Western Balkans is an investment in the future of the EU.

And we will also continue investing in our partnerships across our neighbourhood – from stepping up our engagement in the Eastern Partnership to implementing the new Agenda for the Mediterranean and continuing to work on the different aspects of our relationship with Turkey.

 

Honourable Members,

If Europe is to become a more active global player, it also needs to focus on the next generation of partnerships.

In this spirit, today's new EU - Indo-Pacific strategy is a milestone.  It reflects the growing importance of the region to our prosperity and security. But also the fact that autocratic regimes use it to try to expand their influence.  

Europe needs to be more present and more active in the region.

So we will work together to deepen trade links, strengthen global supply chains and develop new investment projects on green and digital technologies. 

This is a template for how Europe can redesign its model to connect the world. 

We are good at financing roads. But it does not make sense for Europe to build a perfect road between a Chinese-owned copper mine and a Chinese-owned harbour.

We have to get smarter when it comes to these kinds of investments.

This is why we will soon present our new connectivity strategy called Global Gateway.

We will build Global Gateway partnerships with countries around the world. We want investments in quality infrastructure, connecting goods, people and services around the world. 

We will take a values-based approach, offering transparency and good governance to our partners.

We want to create links and not dependencies!

And we know how this can work. Since the summer, a new underwater fibre optic cable has connected Brazil to Portugal.

We will invest with Africa to create a market for green hydrogen that connects the two shores of the Mediterranean.

We need a Team Europe approach to make Global Gateway happen. We will connect institutions and investment, banks and the business community. And we will make this a priority for regional summits – starting with the next EU-Africa Summit in February. 

We want to turn Global Gateway into a trusted brand around the world.

And let me be very clear: doing business around the world, global trade – all that is good and necessary. But this can never be done at the expense of people's dignity and freedom.

There are 25 million people out there, who are threatened or coerced into forced labour. We can never accept that they are forced to make products – and that these products then end up for sale in shops here in Europe.

So we will propose a ban on products in our market that have been made by forced labour.

Human rights are not for sale – at any price.

 

A EUROPE UNITED IN FREEDOM AND DIVERSITY

 

And, Honourable Members, human beings are not bargaining chips.

Look at what happened at our borders with Belarus. The regime in Minsk has instrumentalised human beings. They have put people on planes and literally pushed them towards Europe's borders.

This can never be tolerated.

And the quick European reaction shows that. And rest assured, we will continue to stand together with Lithuania, Latvia and Poland.

And, let's call it what it is: this is a hybrid attack to destabilise Europe.

 

Honourable Members,

These are not isolated events. We saw similar incidents at other borders. And we can expect to see it again. This is why, as part of our work on Schengen, we will set out new ways to respond to such aggression and ensure unity in protecting our external borders. 

But as long as we do not find common ground on how to manage migration, our opponents will continue to target that.

Meanwhile, human traffickers continue to exploit people through deadly routes across the Mediterranean.

These events show us that every country has a stake in building a European migration system.

The New Pact on Migration and Asylum gives us everything we need to manage the different types of situations we face.

All the elements are there. This is a balanced and humane system that works for all Member States - in all circumstances. We know that we can find common ground.

But in the year since the Commission presented the Pact, progress has been painfully slow.

I think, this is the moment now for a European migration management policy. So I urge you, in this House and in Member States, to speed up the process.

This ultimately comes down to a question of trust. Trust between Member States. Trust for Europeans that migration can be managed. Trust that Europe will always live up to its enduring duty to the most vulnerable and most in need.

There are many strongly held views on migration in Europe but I believe the common ground is not so far away.

Because if you ask most Europeans, they would agree that we should act to curb irregular migration but also act to provide a refuge for those forced to flee.

They would agree that we should return those who have no right to stay. But that we should welcome those who come here legally and make such a vital contribution to our society and economy.

And we should all agree that the topic of migration should never be used to divide.

I am convinced that there is a way that Europe can build trust amongst us when it comes to migration.

 

Honourable Members,

Societies that build on democracy and common values stand on stable ground.

They have trust in people.

This is how new ideas are formed, how change happens, how injustices are overcome.

Trust in these common values brought our founders together, after World War Two.

And it is these same values that united the freedom fighters who tore down the Iron Curtain over 30 years ago.

They wanted democracy.
They wanted the freedom to choose their government.
They wanted the rule of law and for everyone to be equal before the law.
They wanted freedom of speech and independent media. To no longer be spied on by their governments.
They wanted to combat corruption. And the freedom to be different from the majority.

Or, as former Czech President Václav Havel put it, they wanted all those "great European values". These values come from the cultural, religious and humanist heritage of Europe.

They are part of our soul, part of what defines us today.

These values are now enshrined in our European treaties. This is what we all signed up to when we became part of this Union as free and sovereign countries.

We are determined to defend these values. And we will never waver in that determination.

Our values are guaranteed by our legal order and safeguarded by the judgments of the European Court of Justice. These judgments are binding. We make sure that they are respected. And we do so in every Member State of our Union.

Because protecting the rule of law is not just a noble goal. Protecting the rule of law is also hard work and a constant struggle for improvement.

Our Rule of Law reports are part of this process, with for example justice reforms in Malta or corruption inquiries in Slovakia.

And from 2022, our Rule of Law reports will come with specific recommendations to Member States.

Nevertheless, there are worrying developments in certain Member States. Let me be clear: dialogue always comes first. But dialogue is not an end in itself, it should lead to results.

This is why we take a dual approach of dialogue and decisive action. This is what we did last week. And this is what we will continue to do.

Because people must be able to rely on the right to an independent judiciary. The right to be treated equally before the law. Everywhere in Europe. Whether you belong to a majority or a minority.

 

Honourable Members,

The European budget is the future of our Union cast in figures.

That is why it must be protected. We need to ensure that every euro and every cent is spent for its proper purpose and in line with rule of law principles.

Investments that enable our children to have a better future must not be allowed to seep away into dark channels.

Corruption is not just taxpayer money stolen.  It is investors scared off, big favours bought by big money and democracy undermined by the powerful.

When it comes to protecting our budget, we will pursue every case, with everything in our power. 

 

Honourable Members,

Defending our values is also defending freedom. Freedom to be who you are, freedom to say what's on your mind, freedom to love whoever you want.

But freedom also means freedom from fear. And during the pandemic, too many women were deprived of that freedom. It was an acutely terrifying time for those with nowhere to hide, nowhere to escape from their abusers. We need to shed light on this darkness, we need to show ways out of the pain. Their abusers must be brought to justice.

And those women must have their freedom and their self-determination back.

This is why by the end of year, we will propose a law to combat violence against women – from prevention to protection and effective prosecution, online and offline.

It is about defending the dignity of each individual. It is about justice. Because this is the soul of Europe. And we must make it even stronger.

 

Honourable members,

Allow me to finish with one of the freedoms that gives voice to all other freedoms – media freedom.

Journalists are being targeted simply for doing their job. Some have been threatened, some beaten and, tragically, some murdered. Right here, in our European Union.

Let me mention some of their names: Daphné Caruana Galizia. Ján Kuciak. Peter de Vries.

The details of their stories may be different but what they have in common is that they all fought and died for our right to be informed.

Information is a public good. We must protect those who create transparency – the journalists. That is why today we have put forward a recommendation to give journalists better protection.

And we need to stop those who threaten media freedom. Media companies cannot be treated as just another business.  Their independence is essential. Europe needs a law that safeguards this independence – and the Commission will deliver a Media Freedom Act in the next year.

Defending media freedom means defending our democracy.

 

Conclusion

Honourable Members,

Strengthening Schuman's European ideal that I invoked earlier is a continuous work.

And we should not hide away from our inconsistencies and imperfections.

But imperfect as it might be, our Union is both beautifully unique and uniquely beautiful.

It is a Union where we strengthen our individual liberty through the strength of our community.

A Union shaped as much by our shared history and values as by our different cultures and perspectives.

A Union with a soul.

Trying to find the right words to capture the essence of this feeling is not easy. But it is easier when you borrow them from someone who inspires you. And this is why I have invited a guest of honour to be with us today.

Many of you might know her – a gold medallist from Italy who captured my heart this summer. 

But what you might not know is that only in April, she was told her life was in peril. She went through surgery, she fought back, she recovered.

And only 119 days after she left the hospital, she won Paralympic gold. Honourable Members, please join me in welcoming Beatrice Vio. Bebe has overcome so much, so young.

Her story is one of rising against all odds. Of succeeding thanks to talent, tenacity and unrelenting positivity. She is in the image of her generation: a leader and an advocate for the causes she believes in.

And she has managed to achieve all of that by living up to her belief that - if it seems impossible – then it can be done. Se sembra impossibile, allora si può fare.

This was the spirit of Europe's founders and this is the spirit of Europe's next generation. So let's be inspired by Bebe and by all the young people who change our perception of the possible.


Who show us that you can be what you want to be. And that you can achieve whatever you believe.

 

Honourable Members:

This is the soul of Europe.

This is the future of Europe.

Let's make it stronger together.

 

Viva l'Europa. 
 \newpage\section{Speech 11 - von der Leyen - 2022-09-14}
\url{https://ec.europa.eu/commission/presscorner/detail/en/speech_22_5493}\\[3mm]
A UNION THAT STANDS STRONG TOGETHER

 

INTRODUCTION

Madam President,

Honourable Members,

My fellow Europeans,

Never before has this Parliament debated the State of our Union with war raging on European soil.

We all remember that fateful morning in late February.

Europeans from across our Union woke up dismayed by what they saw. Shaken by the resurgent and ruthless face of evil. Haunted by the sounds of sirens and the sheer brutality of war.

But from that very moment, a whole continent has risen in solidarity.

At the border crossings where refugees found shelter. In our streets, filled with Ukrainian flags. In the classrooms, where Ukrainian children made new friends.

From that very moment, Europeans neither hid nor hesitated.

They found the courage to do the right thing.

And from that very moment, our Union as a whole has risen to the occasion.

Fifteen years ago, during the financial crisis, it took us years to find lasting solutions.

A decade later, when the global pandemic hit, it took us only weeks.

But this year, as soon as Russian troops crossed the border into Ukraine, our response was united, determined and immediate.

And we should be proud of that.

We have brought Europe's inner strength back to the surface.

And we will need all of this strength. The months ahead of us will not be easy. Be it for families who are struggling to make ends meet, or businesses, who are facing tough choices about their future.

Let us be very clear: much is at stake here. Not just for Ukraine – but for all of Europe and the world at large.

And we will be tested. Tested by those who want to exploit any kind of divisions between us.

This is not only a war unleashed by Russia against Ukraine.

This is a war on our energy, a war on our economy, a war on our values and a war on our future.

This is about autocracy against democracy.

And I stand here with the conviction that with courage and solidarity, Putin will fail and Europe will prevail.

 

THE COURAGE TO STAND WITH OUR HEROES

Honourable Members,

Today - courage has a name, and that name is Ukraine.

Courage has a face, the face of Ukrainian men and women who are standing up to Russian aggression.

I remember a moment in the early weeks of the invasion. When the First Lady of Ukraine, Olena Zelenska, gathered the parents of Ukrainian children killed by the invader.

Hundreds of families for whom the war will never end, and for whom life will never go back to what it was before.

We saw the first Lady leading a silent crowd of heartbroken mothers and fathers, and hang small bells in the trees, one for every fallen child.

And now the bells will ring forever in the wind, and forever, the innocent victims of this war will live in our memory.

And she is here with us today!

Dear Olena, it took immense courage to resist Putin's cruelty.

But you found that courage.

And a nation of heroes has risen.

Today, Ukraine stands strong because an entire country has fought street by street, home by home.

Ukraine stands strong because people like your husband, President Zelenskyy, have stayed in Kyiv to lead the resistance – together with you and your children, dear First Lady.

You have given courage to the whole nation. And we have seen in the last days the bravery of Ukrainians paying off.

You have given voice to your people on the global stage.

And you have given hope to all of us.

So today we want to thank you and all Ukrainians.

Glory to a country of European heroes. Slava Ukraini!

Europe's solidarity with Ukraine will remain unshakeable.

From day one, Europe has stood at Ukraine's side. With weapons. With funds. With hospitality for refugees. And with the toughest sanctions the world has ever seen.

Russia's financial sector is on life-support. We have cut off three quarters of Russia's banking sector from international markets.

Nearly one thousand international companies have left the country.

The production of cars fell by three-quarters compared to last year. Aeroflot is grounding planes because there are no more spare parts. The Russian military is taking chips from dishwashers and refrigerators to fix their military hardware, because they ran out of semiconductors. Russia's industry is in tatters.

It is the Kremlin that has put Russia's economy on the path to oblivion.

This is the price for Putin's trail of death and destruction.

And I want to make it very clear, the sanctions are here to stay.

This is the time for us to show resolve, not appeasement.

The same is true for our financial support to Ukraine.

So far Team Europe have provided more than 19 billion euros in financial assistance.

And this is without counting our military support.

And we are in it for the long haul.

Ukraine's reconstruction will require massive resources. For instance, Russian strikes have damaged or destroyed more than 70 schools.

Half a million Ukrainian children have started their school year in the European Union. But many others inside Ukraine simply don't have a classroom to go to.

So today I am announcing that we will work with the First Lady to support the rehabilitation of damaged Ukrainian schools. And that is why we will provide 100 million euros. Because the future of Ukraine begins in its schools.

We will not only support with finance – but also empower Ukraine to make the most of its potential.

Ukraine is already a rising tech hub and home to many innovative young companies.

So I want us to mobilise the full power of our Single Market to help accelerate growth and create opportunities.

In March, we connected successfully Ukraine to our electricity grid. It was initially planned for 2024. But we did it within two weeks. And today, Ukraine is exporting electricity to us. I want to significantly expand this mutually beneficial trade.

We have already suspended import duties on Ukrainian exports to the EU.

We will bring Ukraine into our European free roaming area.

Our solidarity lanes are a big success.

And building on all that, the Commission will work with Ukraine to ensure seamless access to the Single Market. And vice-versa.

Our Single Market is one of Europe's greatest success stories. Now it's time to make it a success story for our Ukrainian friends, too.

And this is why I am going to Kyiv today, to discuss this in detail with President Zelenskyy.

 

Honourable Members,

One lesson from this war is we should have listened to those who know Putin.

To Anna Politkovskaya and all the Russian journalists who exposed the crimes, and paid the ultimate price.

To our friends in Ukraine, Moldova, Georgia, and to the opposition in Belarus.

We should have listened to the voices inside our Union – in Poland, in the Baltics, and all across Central and Eastern Europe.

They have been telling us for years that Putin would not stop.

And they acted accordingly.

Our friends in the Baltics have worked hard to end their dependency on Russia. They have invested in renewable energy, in LNG terminals, and in interconnectors.

This costs a lot. But dependency on Russian fossil fuels comes at a much higher price.

We have to get rid of this dependency all over Europe.

Therefore we agreed on joint storage. We are at 84\% now: we are overshooting our target.

But unfortunately that will not be enough.

We have diversified away from Russia to reliable suppliers. US, Norway, Algeria and others.

Last year, Russian gas accounted for 40\% of our gas imports. Today it's down to 9\% pipeline gas.

But Russia keeps on actively manipulating our energy market. They prefer to flare the gas than to deliver it. This market is not functioning anymore.

In addition the climate crisis is heavily weighing on our bills. Heat waves have boosted electricity demand. Droughts shut down hydro and nuclear plants.

As a result, gas prices have risen by more than 10 times compared to before the pandemic.

Making ends meet is becoming a source of anxiety for millions of businesses and households.

But Europeans are also coping courageously with this.

Workers in ceramics factories in central Italy, have decided to move their shifts to early morning, to benefit from lower energy prices.

Just imagine the parents among them, having to leave home early, when the kids are still sleeping, because of a war they haven't chosen.

This is one example in a million of Europeans adapting to this new reality.

I want our Union to take example from its people. Reducing demand during peak hours will make supply last longer, and it will bring prices down. 

This is why we are putting forward measures for Member States to reduce their overall electricity consumption.

But more targeted supported is needed.

For industries, like glass makers who have to turn off their ovens. Or for single parents facing one daunting bill after another. 

Millions of Europeans need support.

EU Member States have already invested billions of euros to assist vulnerable households.

But we know this will not be enough.

This is why we are proposing a cap on the revenues of companies that produce electricity at a low cost.

These companies are making revenues they never accounted for, they never even dreamt of.

In our social market economy, profits are good.

But in these times it is wrong to receive extraordinary record profits benefitting from war and on the back of consumers.

In these times, profits must be shared and channelled to those who need it the most.

Our proposal will raise more than 140 billion euros for Member States to cushion the blow directly.

And because we are in a fossil fuel crisis, the fossil fuel industry has a special duty, too.

Major oil, gas and coal companies are also making huge profits. So they have to pay a fair share – they have to give a crisis contribution.

These are all emergency and temporary measures we are working on, including our discussions on price caps.

We need to keep working to lower gas prices.

We have to ensure our security of supply and, at the same time, ensure our global competitiveness.

So we will develop with the Member States a set of measures that take into account the specific nature of our relationship with suppliers – ranging from unreliable suppliers such as Russia to reliable friends such as Norway.

I have agreed with Prime Minister Store to set up a task force. Teams have started their work.

Another important topic is on the agenda. Today our gas market has changed dramatically: from pipeline mainly to increasing amounts of LNG.

But the benchmark used in the gas market – the TTF – has not adapted.

This is why the Commission will work on establishing a more representative benchmark.

At the same time we also know that energy companies are facing severe problems with liquidity in electricity futures markets, risking the functioning of our energy system.

We will work with market regulators to ease these problems by amending the rules on collateral - and by taking measures to limit intra-day price volatility.

And we will amend the temporary state aid framework in October to allow for the provision of state guarantees, while preserving a level playing field.

These are all first steps. But as we deal with this immediate crisis, we must also look forward.

The current electricity market design – based on merit order – is not doing justice to consumers anymore.

They should reap the benefits of low-cost renewables.

So, we have to decouple the dominant influence of gas on the price of electricity. This is why we will do a deep and comprehensive reform of the electricity market.

Now - here is an important point. Half a century ago, in the 1970s, the world faced another fossil fuel crisis.

Some of us remember the car-free weekends to save energy. Yet we kept driving on the same road.

We did not get rid of our dependency on oil. And worse, fossil fuels were even massively subsidised.

This was wrong, not just for the climate, but also for our public finances, and our independence. And we are still paying for this today.

Only a few visionaries understood that the real problem was fossil fuels themselves, not just their price.

Among them were our Danish friends.

When the oil crisis hit, Denmark started to invest heavily into harnessing the power of the wind.

They laid the foundations for its global leadership in the sector and created tens of thousands of new jobs.

This is the way to go!

Not just a quick fix, but a change of paradigm, a leap into the future.

 

STAYING THE COURSE AND PREPARING FOR THE FUTURE

Honourable Members,

The good news is: this necessary transformation has started.

It is happening in the North Sea and the Baltic Sea, where our Member States agreed to invest massively into off-shore wind generation.

It is happening in Sicily, where Europe's largest solar factory will soon manufacture the newest generation of panels.

And it is happening in Northern Germany, where local trains now run on green hydrogen.

And hydrogen can be a game changer for Europe.

We need to move our hydrogen economy from niche to scale.

With REPowerEU, we have doubled our 2030 target to produce ten million tons of renewable hydrogen in the EU, each year.

To achieve this, we must create a market maker for hydrogen, in order to bridge the investment gap and connect future supply and demand.

That is why I can today announce that we will create a new European Hydrogen Bank.

It will help guarantee the purchase of hydrogen, notably by using resources from the Innovation Fund.

It will be able to invest 3 billion euros to help building the future market for hydrogen.

This is how we power the economy of the future.

This is the European Green Deal.

 

And we have all seen in the last months just how important the European Green Deal is.

The summer of 2022 will be remembered as a turning point.

We all saw the dry rivers, the burning forests, the impact of the extreme heat.

And under the surface, the situation is far starker.

So far the glaciers in the Alps helped as an emergency reserve for rivers like the Rhine or Rhone.

But with Europe's glaciers melting faster than ever, future droughts will be felt far more acutely.

We must work relentlessly to adapt to our climate – making nature our first ally.

This is why our Union will push for an ambitious global deal for nature at the UN Biodiversity conference in Montreal later this year.

And we will do the same at COP27 in Sharm el-Sheikh.

But in the short term, we also need to be better equipped to handle our changing climate.

The destructive power of extreme weather is too big for any country to fight on its own.

This summer, we sent planes from Greece, Sweden and Italy to fight fires in France and Germany.

But as disasters become more frequent and more intense, Europe will need more capacity.

This is why I can today announce that we will double our firefighting capacity over the next year.

The EU will buy another 10 light amphibious aircrafts and three helicopters to add to the fleet.

This is European solidarity in action.

 

Honourable Members,

The last years have shown how much Europe can achieve when it is united.

After an unprecedented pandemic, our economic output overtook pre-crisis levels in record time.

We went from having no vaccine to securing over 4 billion doses for Europeans and for the world.

And in record time, we came up with SURE – so that people could stay in their jobs even if their companies had run out of work.

We were in the deepest recession since World War 2.

We achieved the fastest recovery since the post-war boom

And that was possible because we all rallied behind a common recovery plan.

NextGenerationEU has been a boost of confidence for our economy.

And its journey has only just begun.

So far, 100 billion euros have been disbursed to Member States. This means: 700 billion euros still haven't flown into our economy.

NextGenerationEU will guarantee a constant stream of investment to sustain jobs and growth.

It means relief for our economy. But most importantly, it means renewal.

It is financing new wind turbines and solar parks, high-speed trains and energy-saving renovations.

We conceived NextGenerationEU almost two years ago, and yet it is exactly what Europe needs today.

So let's stick to the plan.

Let's get the money on the ground.

 

Honourable Members,

The future of our children needs both that we invest in sustainability and that we invest sustainably.

We must finance the transition to a digital and net-zero economy.

And yet we also have to acknowledge a new reality of higher public debt.

We need fiscal rules that allow for strategic investment, while safeguarding fiscal sustainability.

Rules that are fit for the challenges of this decade.

In October, we will come forward with new ideas for our economic governance.

But let me share a few basic principles with you.

Member States should have more flexibility on their debt reduction paths.

But there should be more accountability on the delivery of what we have agreed on.

There should be simpler rules that all can follow.

To open the space for strategic investment and to give financial markets the confidence they need.

Let us chart once again a joint way forward.

With more freedom to invest. And more scrutiny on progress.

More ownership by Member States. And better results for citizens.

Let us rediscover the Maastricht spirit – stability and growth can only go hand in hand.

 

Honourable Members,

As we embark on this transition in our economy, we must rely on the enduring values of our social market economy.

It's the simple idea that Europe's greatest strength lies in each and every one of us.

Our social market economy encourages everyone to excel, but it also takes care of our fragility as human beings.

It rewards performance and guarantees protection. It opens opportunities but also set limits.

We need this even more today.

Because the strength of our social market economy will drive the green and digital transition.

We need an enabling business environment, a workforce with the right skills and access to raw materials our industry needs.

 

Our future competitiveness depends on it.

We must remove the obstacles that still hold our small companies back.

They must be at the centre of this transformation – because they are the backbone of Europe's long history of industrial prowess.

And they have always put their employees first – even and especially in times of crisis.

But inflation and uncertainty are weighing especially hard on them.

This is why we will put forward an SME Relief Package.

It will include a proposal for a single set of tax rules for doing business in Europe – we call it BEFIT.

This will make it easier to do business in our Union. Less red tape means better access to the dynamism of a continental market.

And we will revise the Late Payment Directive – because it is simply not fair that 1 in 4 bankruptcies are due to invoices not being paid on time.

For millions of family businesses, this will be a lifeline in troubled waters.

 

But European companies are also grappling with a shortage of staff.

Unemployment is at a record low, and this is great.

At the same time, job vacancies are at a record high.

Europe lacks truck drivers, waiters and airport workers,

as well as nurses, engineers and IT technicians.

Both low-end and high-end. We need everyone on board.

We need much more focus in our investment on professional education and upskilling.

We need better cooperation with the companies, because they know best what they need.

And we need to match these needs with people's aspirations.

But we also have to attract the right skills to our continent, skills that help companies and strengthen Europe's growth.

As a first important step, we need to speed up and facilitate the recognition of qualifications also of third country nationals.

This will make Europe more attractive for skilled workers.

This is why I am proposing to make 2023 the European Year of Skills.

 

Honourable Members,

My third point for our SMEs and our industry.

Whether we talk about chips for virtual reality or cells for solar panels, the twin transitions will be fuelled by raw materials

Lithium and rare earths are already replacing gas and oil at the heart of our economy.

By 2030, our demand for those rare earth metals will increase fivefold.

And this is a good sign, because it shows that our European Green Deal is moving fast.

The not so good news is – one country dominates the market.

So we have to avoid falling into the same dependency as with oil and gas.

This is where our trade policy comes into play.

New partnerships will advance not only our vital interests – but also our values.

Trade that embraces workers' rights and the highest environmental standards is possible with like-minded partners.

We need to update our links to reliable countries and key growth regions.

And for this reason, I intend to put forward for ratification the agreements with Chile, Mexico and New Zealand.

And advance negotiations with key partners like Australia and India.

 

But securing supplies is only a first step.

The processing of these metals is just as critical.

Today, China controls the global processing industry. Almost 90 \% of rare earths and 60 \% of lithium are processed in China.

We will identify strategic projects all along the supply chain, from extraction to refining, from processing to recycling. And we will build up strategic reserves where supply is at risk.

This is why today I am announcing a European Critical Raw Materials Act.

We know this approach can work.

Five years ago, Europe launched the Battery Alliance. And soon, two third of the batteries we need will be produced in Europe.

Last year I announced the European Chips Act. And the first chips gigafactory will break ground in the coming months.

We now need to replicate this success.

This is also why we will increase our financial participation to Important Projects of Common European Interest.

And for the future, I will push to create a new European Sovereignty Fund.

Let's make sure that the future of industry is made in Europe.

 

STANDING UP FOR OUR DEMOCRACY

Honourable Members,

As we look around at the state of the world today, it can often feel like there is a fading away of what once seemed so permanent.

And in some way, the passing of Queen Elizabeth II last week reminded us of this. 

She is a legend!

She was a constant throughout the turbulent and transforming events in the last 70 years.

Stoic and steadfast in her service.

But more than anything, she always found the right words for every moment in time.

From the calls she made to war evacuees in 1940 to her historic address during the pandemic.

She spoke not only to the heart of her nation but to the soul of the world.

And when I think of the situation we are in today, her words at the height of the pandemic still resonate with me.

She said: “We will succeed – and that success will belong to every one of us”.

She always reminded us that our future is built on new ideas and founded in our oldest values.

Since the end of World War 2, we have pursued the promise of democracy and the rule of law.

And the nations of the world have built together an international system promoting peace and security, justice and economic progress.

Today this is the very target of Russian missiles.

What we saw in the streets of Bucha, in the scorched fields of grain, and now at the gates of Ukraine's largest nuclear plant – is not only a violation of international rules.

It's a deliberate attempt to discard them.

This watershed moment in global politics calls for a rethink of our foreign policy agenda.

This is the time to invest in the power of democracies.

This work begins with the core group of our like-minded partners: our friends in every single democratic nation on this globe.

We see the world with the same eyes. And we should mobilise our collective power to shape global goods.

We should strive to expand this core of democracies. The most immediate way to do so is to deepen our ties and strengthen democracies on our continent.

This starts with those countries that are already on the path to our Union.

We must be at their side every step of the way.

Because the path towards strong democracies and the path towards our Union are one and the same.

So I want the people of the Western Balkans, of Ukraine, Moldova and Georgia to know:

You are part of our family, your future is in our Union, and our Union is not complete without you!

We have also seen that there is a need to reach out to the countries of Europe – beyond the accession process.

This is why I support the call for a European Political Community – and we will set out our ideas to the European Council.

But our future also depends on our ability to engage beyond the core of our democratic partners.

Countries near and far, share an interest in working with us on the great challenges of this century, such as climate change and digitalisation.

This is the main idea behind Global Gateway, the investment plan I announced right here one year ago.

It is already delivering on the ground.

Together with our African partners we are building two factories in Rwanda and Senegal to manufacture mRNA vaccines.

They will be made in Africa, for Africa, with world-class technology.

And we are now replicating this approach across Latin America as part of a larger engagement strategy.

This requires investment on a global scale.

So we will team up with our friends in the US and with other G7 partners to make this happen.

In this spirit, President Biden and I will convene a leaders' meeting to review and announce implementation projects.

 

Honourable Members,

This is part of our work of strengthening our democracies.

But we should not lose sight of the way foreign autocrats are targeting our own countries.

Foreign entities are funding institutes that undermine our values.

Their disinformation is spreading from the internet to the halls of our universities.

Earlier this year, a university in Amsterdam shut down an allegedly independent research centre, which was actually funded by Chinese entities. This centre was publishing so-called research on human rights, dismissing the evidence of forced labour camps for Uyghurs as “rumours”.

These lies are toxic for our democracies.

Think about this: We introduced legislation to screen foreign direct investment in our companies for security concerns.

If we do that for our economy, shouldn't we do the same for our values?

We need to better shield ourselves from malign interference.

This is why we will present a Defence of Democracy package.

It will bring covert foreign influence and shady funding to light.

We will not allow any autocracy's Trojan horses to attack our democracies from within.

 

For more than 70 years, our continent has marched towards democracy. But the gains of our long journey are not assured.

Many of us have taken democracy for granted for too long. Especially those, like me, who have never experienced what it means to live under the fist of an authoritarian regime.

Today we all see that we must fight for our democracies. Every single day.

We must protect them both from the external threats they face, and from the vices that corrode them from within.

It is my Commission's duty and most noble role to protect the rule of law.

So let me assure you: we will keep insisting on judicial independence.

And we will also protect our budget through the conditionality mechanism.

And today I would like to focus on corruption, with all its faces. The face of foreign agents trying to influence our political system. The face of shady companies or foundations abusing public money.

If we want to be credible when we ask candidate countries to strengthen their democracies, we must also eradicate corruption at home.

That is why in the coming year the Commission will present measures to update our legislative framework for fighting corruption.

We will raise standards on offences such as illicit enrichment, trafficking in influence and abuse of power, beyond the more classic offences such as bribery. 

And we will also propose to include corruption in our human rights sanction regime, our new tool to protect our values abroad.

Corruption erodes trust in our institutions. So we must fight back with the full force of the law.

 

Honourable Members,

Our founders only meant to lay the first stone of this democracy.

They always thought that future generations would complete their work.

“Democracy has not gone out of fashion, but it must update itself in order to keep improving people's lives.”

These are the words of David Sassoli – a great European, who we all pay tribute to today.

David Sassoli thought that Europe should always look for new horizons.

And through the adversities of these times, we have started to see what our new horizon might be.

A braver Union.

Closer to its people in times of need.

Bolder in responding to historic challenges and daily concerns of Europeans. And to walk at their side when they deal with the big trials of life.

This is why the Conference on the Future of Europe was so important.

It was a sneak peek of a different kind of citizens' engagement, well beyond election day.

And after Europe listened to its citizens' voice, we now need to deliver.

The Citizens' Panels that were central to the Conference will now become a regular feature of our democratic life.

And in the Letter of Intent that I have sent today to President Metsola and Prime Minister Fiala, I have outlined a number of proposals for the year ahead that stem from the Conference conclusions.

They include for example a new initiative on mental health.

We should take better care of each other. And for many who feel anxious and lost, appropriate, accessible and affordable support can make all the difference.

 

Honourable Members,

Democratic institutions must constantly gain and regain the citizens' trust.

We must live up to the new challenges that history always puts before us.

Just like Europeans did when millions of Ukrainians came knocking on their door.

This is Europe at its best.

A Union of determination and solidarity.

But this determination and drive for solidarity is still missing in our migration debate.

Our actions towards Ukrainian refugees must not be an exception. They can be our blueprint for going forward.

We need fair and quick procedures, a system that is crisis proof and quick to deploy, and a permanent and legally binding mechanism that ensures solidarity.

And at the same time, we need effective control of our external borders, in line with the respect of fundamental rights.

I want a Europe that manages migration with dignity and respect.

I want a Europe where all Member States take responsibility for challenges we all share.

And I want a Europe that shows solidarity to all Member States.

We have progress on the Pact, we now have the Roadmap. And we now need the political will to match.

 

Honourable Members,

Three weeks ago, I had the incredible opportunity of joining 1,500 young people from all over Europe and the world, who gathered in Taizé.

They have different views, they come from different countries, they have different backgrounds, they speak different languages.

And yet, there is something that connects them.

They share a set of values and ideals.

They believe in these values.

They are all passionate about something larger than themselves.

This generation is a generation of dreamers but also of makers.

In my last State of the Union address, I told you that I would like Europe to look more like these young people.

We should put their aspirations at the heart of everything we do.

And the place for this is in our founding Treaties.

Every action that our Union takes should be inspired by a simple principle.

That we should do no harm to our children's future.

That we should leave the world a better place for the next generation.

And therefore, Honourable Members, I believe that it is time to enshrine solidarity between generations in our Treaties.

It is time to renew the European promise.

And we also need to improve the way we do things and the way we decide things.

Some might say this is not the right time. But if we are serious about preparing for the world of tomorrow we must be able to act on the things that matter the most to people.

And as we are serious about a larger union, we also have to be serious about reform.

So as this Parliament has called for, I believe the moment has arrived for a European Convention.

 

CONCLUSION

Honourable Members,

They say that light shines brightest in the dark.

And that was certainly true for the women and the children fleeing Russia's bombs.

They fled a country at war, filled with sadness for what they had left behind, and fear for what may lie ahead.

But they were received with open arms. By many citizens like Magdalena and Agnieszka. Two selfless young women from Poland.

As soon as they heard about trains full of refugees, they rushed to the Warsaw Central Station.

They started to organise.

They set up a tent to assist as many people as possible.

They reached out to supermarket chains for food, and to local authorities to organise buses to hospitality centres.

In a matter of days, they gathered 3000 volunteers, to welcome refugees 24/7.

 

Honourable Members,

Magdalena and Agnieszka are here with us today.

Please join me in applauding them and each and every European who opened their hearts and their homes.

Their story is about everything our Union stands and strives for.

It is a story of heart, character and solidarity.

They showed everyone what Europeans can achieve when we rally around a common mission.

This is Europe's spirit.

A Union that stands strong together.

A Union that prevails together.

Long live Europe.
 \newpage\section{Speech 12 - von der Leyen - 2023-09-13}
\url{https://ec.europa.eu/commission/presscorner/detail/en/SPEECH_20_1655}\\[3mm]
ANSWERING THE CALL OF HISTORY

 

INTRODUCTION – DELIVER TODAY, PREPARE FOR TOMORROW

Honourable Members,

In just under 300 days, Europeans will take to the polls in our unique and remarkable democracy.

As with any election, it will be a time for people to reflect on the State of our Union and the work done by those that represent them.

But it will also be a time to decide on what kind of future and what kind of Europe they want. 

Among them will be millions of first-time voters, the youngest of whom were born in 2008.

As they stand in that polling booth, they will think about what matters to them.

They will think about the war that rages at our borders.

Or the impact of destructive climate change.

About how artificial intelligence will influence their lives.

Or of their chances of getting a house or a job in the years ahead.

Our Union today reflects the vision of those who dreamt of a better future after World War II.  

A future in which a Union of nations, democracies and people would work together to share peace and prosperity.

They believed that Europe was the answer to the call of history. 

When I speak to the new generation of young people, I see that same vision for a better future.

That same burning desire to build something better.

That same belief that in a world of uncertainty, Europe once again must answer the call of history.

And that is what we must do together. 

 

Honourable Members,

This starts with earning the trust of Europeans to deal with their aspirations and anxieties.

And in the next 300 days we must finish the job that they entrusted us with.

I want to thank this House for its leading role in delivering on one of the most ambitious transformations this Union has ever embarked on.

When I stood in front of you in 2019 with my programme for a green, digital and geopolitical Europe I know that some had doubts.

And that was before the world turned upside down with a global pandemic and a brutal war on European soil. 

But look at where Europe is today.

We have seen the birth of a geopolitical Union – supporting Ukraine, standing up to Russia's aggression, responding to an assertive China and investing in partnerships.

We now have a European Green Deal as the centrepiece of our economy and unmatched in ambition. 

We have set the path for the digital transition and become global pioneers in online rights.

We have the historic NextGenerationEU – combining 800 billion euros of investment and reform – and creating decent jobs for today and tomorrow.   

We have set the building blocks for a Health Union, helping to vaccinate an entire continent – and large parts of the world. 

We have started making ourselves more independent in critical sectors, like energy, chips or raw materials.

I would also like to thank you for the ground-breaking and pioneering work we did on gender equality.

As a woman, this means a lot to me.

We have concluded files that many thought would be blocked forever, like the Women on Boards Directive and the historic accession of the EU to the Istanbul Convention.

With the Directive on pay transparency we have cast into law the basic principle that equal work deserves equal pay.

There is not a single argument why – for the same type of work – a woman should be paid less than a man.

But our work is far from over and we must continue pushing for progress together.

I know this house supports our proposal on combating violence against women.

Here too, I would like that we cast into law another basic principle: No, means no.

There can be no true equality without freedom from violence.

 

And thanks to this Parliament, to Member States and to my team of Commissioners, we have delivered over 90\% of the Political Guidelines I presented in 2019.

Together, we have shown that when Europe is bold, it gets things done.

And our work is far from over - so let's stand together. 

Let's deliver today and prepare for tomorrow.

 

EUROPEAN GREEN DEAL

Honourable Members,

Four years ago, the European Green Deal was our answer to the call of history. 

And this summer – the hottest ever on record in Europe – was a stark reminder of that. 

Greece and Spain were struck by ravaging wildfires – and were hit again only a few weeks later by devastating floods.

And we saw the chaos and carnage of extreme weather – from Slovenia to Bulgaria and right across our Union.

This is the reality of a boiling planet. 

The European Green Deal was born out of this necessity to protect our planet.

But it was also designed as an opportunity to preserve our future prosperity.

We started this mandate by setting a long-term perspective with the climate law and the 2050 target.

We shifted the climate agenda to being an economic one.

This has given a clear sense of direction for investment and innovation.

And we have already seen this growth strategy deliveringin the short-term. 

Europe's industry is showing every day that it is ready to power this transition.

Proving that modernisation and decarbonisation can go hand in hand.

In the last five years, the number of clean steel factories in the EU has grown from zero to 38.

We are now attracting more investment in clean hydrogenthan the US and China combined.

And tomorrow I will be in Denmark with Prime Minister Mette Frederiksen to see that innovation first hand.

We will mark the launch of the first container ship, powered by clean methanol made with solar energy.

This is the strength of Europe's response to climate change.

The European Green Deal provides the necessary frame, incentives, and investment – but it is the people, the inventors, the engineers who develop the solutions. 

And this is why, Honourable Members,

as we enter the next phase of the European Green Deal, one thing will never change.

We will keep supporting European industry throughout this transition.  

We started with a package of measures – from the Net-Zero Industry Act to the Critical Raw Materials Act. 

With our Industry Strategy, we are looking at the risks and needs of each ecosystem in this transition.

We need to finish this work.

And with this, we need to develop an approach for each industrial ecosystem.

Therefore, starting from this month, we will hold a series of Clean Transition Dialogues with industry.

The core aim will be to support every sector in building its business model for the decarbonisation of industry.

Because we believe that this transition is essential for our future competitiveness in Europe.

But this is just as much about the people and their jobs of today.

Our wind industry, for instance, is a European success story.

But it is currently facing a unique mix of challenges.

This is why we will put forward a European Wind Power package – working closely with industry and Member States.

We will fast-track permitting even more.

We will improve the auction systems across the EU.

We will focus on skills, access to finance and stable supply chains.

But this is broader than one sector:

From wind to steel, from batteries to electric vehicles, our ambition is crystal clear: The future of our clean tech industry has to be made in Europe. 

 

Honourable Members, 

This shows that when it comes to the European Green Deal: 

We stay the course. 

We stay ambitious. 

We stick to our growth strategy. 

And we will always strive for a fair and just transition!

That means a fair outcome for future generations – to live on healthy planet.

And a fair journey for all those impacted – with decent jobs and a solemn promise to leave no one behind.

Just think about manufacturing jobs and competitiveness: a topic we are discussing a lot these days.

Our industry and tech companies like competition.

They know that global competition is good for business.

And that it creates and protects good jobs here in Europe.

But competition is only true as long as it is fair.

Too often, our companies are excluded from foreign markets or are victims of predatory practices.

They are often undercut by competitors benefitting from huge state subsidies.

We have not forgotten how China's unfair trade practicesaffected our solar industry.

Many young businesses were pushed out by heavily subsidised Chinese competitors.

Pioneering companies had to file for bankruptcy.

Promising talents went searching for fortune abroad.

This is why fairness in the global economy is so important – because it affects lives and livelihoods.

Entire industries and communities depend on it.

So, we have be to be clear-eyed about the risks we face.

Take the electric vehicles sector. 

It is a crucial industry for the clean economy, with a huge potential for Europe.

But global markets are now flooded with cheaper Chinese electric cars.

And their price is kept artificially low by huge state subsidies.

This is distorting our market.

And as we do not accept this from the inside, we do not accept this from the outside.

So I can announce today that the Commission is launching an anti-subsidy investigation into electric vehicles coming from China. 

Europe is open for competition. Not for a race to the bottom.

 

We must defend ourselves against unfair practices.

But equally, it is vital to keep open lines of communication and dialogue with China.

Because there also are topics, where we can and have to cooperate.

De-risk, not decouple – this will be my approach with the Chinese leadership at the EU-China Summit later this year. 

 

Honourable Members,

In the European Union, we are proud of our cultural diversity.

We are a ‘Europe of the Regions' with a unique blend of languages, music, art, traditions, crafts and cuisines.

We are also a continent of unique biological diversity.

Some 6 500 species are found only in Europe.

In northern Europe, we find the Wadden Sea, a world natural heritage site and unique habitat offering a home to rare species of flora and fauna and a vital resource for millions of migratory birds. And with the Baltic Sea we have the largest area of brackish sea in the world.

South of that is the European Plain, characterised by vast tracts of moorland and wetland.

These regions are important allies against ongoing climate change.

Protected moors and wetlands absorb enormous volumes of greenhouse gases, secure regional water cycles and are home to unique biodiversity.

And Europe is a continent of forests.

From the mighty coniferous forests of the North and East, via the last remnants of virgin oak and beech forest in central Europe to the cork oak forests of southern Europe: all these forests are an irreplaceable source of goods and services.

They absorb carbon dioxide, supply wood and other products, generate fertile soils, and filter the air and the water.

Biodiversity and ecosystem services are vital for all of us in Europe.

Loss of nature destroys not only the foundations of our life, but also our feeling of what constitutes home.

We must protect it.

At the same time, food security, in harmony with nature, remains an essential task.

I would like to take this opportunity to express my appreciation to our farmers, to thank them for providing us with food day after day.

For us in Europe, this task of agriculture – producing healthy food – is the foundation of our agricultural policy.

And self-sufficiency in food is also important for us.

That is what our farmers provide.

It is not always an easy task, as the consequences of Russia's aggression against Ukraine, climate change bringing droughts, forest fires and flooding, and new obligations are all having a growing impact on farmers' work and incomes.

We must bear that in mind.

Many are already working towards a more sustainable form of agriculture.

We must work together with the men and women in farming to tackle these new challenges.

That is the only way to secure the supply of food for the future.

We need more dialogue and less polarisation.

That is why we want to launch a strategic dialogue on the future of agriculture in the EU.

I am and remain convinced that agriculture and protection of the natural world can go hand in hand.

We need both.

 

ECONOMY, SOCIAL AND COMPETITIVENESS

Honourable Members,

A fair transition for farmers, families and industry.

This is the hallmark of this Mandate.

And it is all the more important as we face strong economic headwinds.

I see three major economic challenges for our industry in the year ahead: labour and skills shortages, inflation, and making business easier for our companies. 

The first has to do with our labour market. 

We have not forgotten the early days of the global pandemic.

When everyone predicted a new wave of 1930-style mass unemployment.

But we defied this prediction.

With SURE – the first-ever European short-time work initiative – we saved 40 million jobs.

This is Europe's social market economy in action. 

And we can be proud of it!

We then immediately restarted our economic engine thanks to NextGenerationEU.

And today we see the results.

Europe is close to full employment.

Instead of millions of people looking for jobs, millions of jobs are looking for people.

Labour and skills shortages are reaching record levels – both here and across all major economies.

74\% of SMEs are saying they are facing skill shortages.

In the peak of the tourist season, restaurants and bars in Europe are running reduced hours because they cannot find staff.

Hospitals are postponing treatment because of lack of nurses.

And two thirds of European companies are looking for IT specialists.

At the same time millions of parents – mostly mothers – are struggling to reconcile work and family, because there is no childcare.

And 8 million young people are neither in employment, education or training.

Their dreams put on hold, their lives on stand-by.

This is not only the cause of so much personal distress. 

It is also one of the most significant bottlenecks for our competitiveness. 

Because labour shortages hamper the capacity for innovation, growth and prosperity.

So we need to improve access to the labour market.

Most importantly for young people, for women.

And we need qualified migration.

In addition, we need to respond to the deep-rooted shifts in technology, society and demography.

And for that, we should rely on the expertise of businesses and trade unions, our collective bargaining partners. 

It is almost forty years since Jacques Delors convened the Val Duchesse meeting that saw the birth of European social dialogue. 

Since then, social partners have shaped the Union of today – ensuring progress and prosperity for millions.

And as the world around us changes faster than ever, social partners must again be at the heart of our future. 

Together we must focus on the challenges facing the labour market – from skills and labour shortages, to new challenges stemming from AI. 

This is why together with the Belgian Presidency next year, we will convene a new Social Partner Summit once again at Val Duchesse. 

The future of Europe will be built with and by our social partners. 

 

The second major economic challenge: persistent high inflation.

Christine Lagarde and the European Central Bank are working hard to keep inflation under control.

We know that returning to the ECB's medium-term target will take some time.

The good news is that Europe has started bringing energy prices down.

We have not forgotten, Putin's deliberate use of gas as a weapon and how it triggered fears of blackout and an energy crisis like in the 70s.

Many thought, we would not have enough energy to get through the winter.

But we made it.

Because we stayed united – pooling our demand and buying energy together.

And at the same time, different to the 70s, we used the crisis to massively invest in renewables and fast-track the clean transition.

We used Europe's critical mass to bring prices down and secure our supply.

The price for gas in Europe was over 300 euros per MWh one year ago. It is now around 35.

So we need to look at how we can replicate this model of success in other fields like critical raw materials or clean hydrogen.

 

The third challenge for European companies is about making it easier to do business. 

Small companies do not have the capacity to cope with complex administration.

Or they are held back by lengthy processes.

This often means they do less with the time they have – and that they miss out on opportunities to grow.

This is why – before the end of the year – we will appoint an EU SME envoy reporting directly to me.

We want to hear directly from small and medium sized businesses, about their everyday challenges.

For every new piece of legislation we conduct a competitiveness check by an independent board.

And next month, we will make the first legislative proposalstowards reducing reporting obligations at the European level by 25\%.

 

Honourable Members,

Let's be frank – this will not be easy.

And we will need your support.

Because this is a common endeavour for all European institutions.

So we also have to work with Member States, to match the 25\% at national level.

It is time to make business easier in Europe!

 

But European companies also need access to key technologies to innovate, develop and manufacture.

This is a question of European sovereignty as the Leaders underlined in Versailles.

It is an economic and national security imperative to preserve a European edge on critical and emerging technologies.

This European industrial policy also requires common European funding.

This is why – as part of our proposal for a review of our budget – we proposed the STEP platform.

With STEP we can boost, leverage and steer EU funds to invest in everything from microelectronics to quantum computing and AI. 

From biotech to clean tech. 

Our companies need this support now – so I urge for a quick agreement on our budget proposal.

And I know I can count on this House. 

And there is more when it comes to competitiveness.

We have seen real bottlenecks along global supply chains, including because of the deliberate policies of other countries. 

Just think about China's export restrictions on gallium and germanium – which are essential for goods like semiconductors and solar panels.

This shows why it is so important for Europe to step up on economic security.

By de-risking and not decoupling.

And I am very proud that this concept has found broad support from key partners.

From Australia to Japan and the United States.

And many other countries around the world want to work together. 

Many are overly dependent on a single supplier for critical minerals.

Others – from Latin America to Africa – want to develop local industries for processing and refining, instead of just shipping their resources abroad.

This is why later this year we will convene the first meeting of our new Critical Raw Materials Club.

At the same time, we will continue to drive open and fair trade.

So far, we have concluded new free trade agreements with Chile, New Zealand and Kenya.

We should aim to complete deals with Australia, Mexico and Mercosur by the end of this year.

And soon thereafter with India and Indonesia.

Smart trade delivers good jobs and prosperity.

 

Honourable Members,

These three challenges – labour, inflation and business environment – come at a time when we are also asking industry to lead on the clean transition.

So we need to look further ahead and set out how we remain competitive as we do that. 

This is why I have asked Mario Draghi – one of Europe's great economic minds – to prepare a report on the future of European competitiveness.  

Because Europe will do “whatever it takes” to keep its competitive edge. 

 

DIGITAL \& AI

Honourable Members, 

When it comes to making business and life easier, we have seen how important digital technology is. 

It is telling that we have far overshot the 20\% investment target in digital projects of NextGenerationEU. 

Member States have used that investment to digitise their healthcare, justice system or transport network.

At the same time, Europe has led on managing the risks of the digital world. 

The internet was born as an instrument for sharing knowledge, opening minds and connecting people. 

But it has also given rise to serious challenges.

Disinformation, spread of harmful content, risks to the privacy of our data. 

All of this led to a lack of trust and a breach of fundamental rights of people. 

In response, Europe has become the global pioneer of citizen's rights in the digital world. 

The DSA and DMA are creating a safer digital space where fundamental rights are protected. 

And they are ensuring fairness with clear responsibilities for big tech. 

This is a historic achievement – and we should be proud of it. 

The same should be true for artificial intelligence.  

It will improve healthcare, boost productivity, address climate change.

But we also should not underestimate the very real threats. 

Hundreds of leading AI developers, academics and experts warned us recently with the following words:

“Mitigating the risk of extinction from AI should be a global priority alongside other societal-scale risks such as pandemics and nuclear war.”

AI is a general technology that is accessible, powerful and adaptable for a vast range of uses - both civilian andmilitary. 

And it is moving faster than even its developers anticipated.  

So we have a narrowing window of opportunity to guide this technology responsibly.

I believe Europe, together with partners, should lead the way on a new global framework for AI, built on three pillars: guardrails, governance and guiding innovation. 

Firstly, guardrails. 

Our number one priority is to ensure AI develops in a human-centric, transparent and responsible way. 

This is why in my Political Guidelines, I committed to setting out a legislative approach in the first 100 days.

We put forward the AI Act – the world's first comprehensive pro-innovation AI law.

And I want to thank this House and the Council for the tireless work on this groundbreaking law.   

Our AI Act is already a blueprint for the whole world.

We must now focus on adopting the rules as soon as possible and turn to implementation.

 

The second pillar is governance. 

We are now laying the foundations for a single governance system in Europe.

But we should also join forces with our partners to ensure a global approach to understanding the impact of AI in our societies.

Think about the invaluable contribution of the IPCC for climate, a global panel that provides the latest science to policymakers.

I believe we need a similar body for AI – on the risks and its benefits for humanity. 

With scientists, tech companies and independent experts all around the table. 

This will allow us to develop a fast and globally coordinated response – building on the work done by the Hiroshima Process and others.

The third pillar is guiding innovation in a responsible way.

Thanks to our investment in the last years, Europe has now become a leader in supercomputing – with 3 of the 5 most powerful supercomputers in the world.

We need to capitalise on this. 

This is why I can announce today a new initiative to open up our high-performance computers to AI start-ups to train their models.

But this will only be part of our work to guide innovation.

We need an open dialogue with those that develop and deploy AI. 

It happens in the United States, where seven major tech companies have already agreed to voluntary rules around safety, security and trust. 

It happens here, where we will work with AI companies, so that they voluntarily commit to the principles of the AI Actbefore it comes into force.

Now we should bring all of this work together towards minimum global standards for safe and ethical use of AI.

 

GLOBAL, MIGRATION AND SECURITY

Honourable Members,

When I stood here four years ago, I said that if we are united on the inside, nobody will divide us from the outside.

And this was the thinking behind the Geopolitical Commission.

Our Team Europe approach has enabled us to be more strategic, more assertive and more united.

And that is more important than ever.

Our heart bleeds when we see the devastating loss of life in Libya and Morocco after the violent floods and earthquake.

Europe will always stand ready to support in any way we can. 

Or think about the Sahel region, one of the poorest yet fastest growing demographically.

The succession of military coups will make the region more unstable for years ahead.

Russia is both influencing and benefiting from the chaos.

And the region has become fertile ground for the rise in terrorism. 

This is of direct concern for Europe – for our security and prosperity.

So we need to show the same unity of purpose towards Africa as we have shown for Ukraine. 

We need to focus on cooperation with legitimate governments and regional organisations.

And we need to develop a mutually beneficial partnership which focuses on common issues for Europe and Africa.

This is why, together with High Representative Borrell, we will work on a new strategic approach to take forward at the next EU-AU Summit. 

 

Honourable Members,

History is on the move.

Russia is waging a full-scale war against the founding principles of the UN Charter.

This has raised immense concerns in countries from Central Asia to the Indo-Pacific.

They are worried that in a lawless world, they might face the same fate as Ukraine.

We see a clear attempt by some to return to bloc thinking – trying to isolate and influence countries in between.

And it comes at a time when there is a deeper unease by many emerging economies about the way institutions and globalisation work for them.

Those concerns are legitimate.

These emerging economies – with their people and natural assets – are essential allies in building a cleaner, safer and more prosperous world.

Europe will always work with them to reform and improve the international system.

We want to lead efforts to make the rules-based order fairer and make distribution more equal.

This will also mean working with new and old partners to deepen our connections.

And Europe's offer with Global Gateway is truly unique.

Global Gateway is more transparent, more sustainable, and more economically attractive.

Just last week I was in New Delhi to sign the most ambitious project of our generation.

The India-Middle East-Europe Economic Corridor.

It will be the most direct connection to date between India, the Arabian Gulf and Europe: With a rail link, that will make trade between India and Europe 40\% faster.

With an electricity cable and a clean hydrogen pipeline – to foster clean energy trade between Asia, the Middle East and Europe.

With a high-speed data cable, to link some of the most innovative digital ecosystems in the world, and create business opportunities all along the way.

These are state-of-the-art connections for the world of tomorrow.

Faster, shorter, cleaner.

And Global Gateway is making the real difference.

I have seen it in Latin America, South-East Asia and across Africa – from building a local hydrogen economy with Namibia and Kenya to a digital economy with the Philippines.

These are investments in our partners' economy.

And they are investments in Europe's prosperity and security in a fast-changing world.

 

Honourable Members,

Every day, we see that conflict, climate change and instability are pushing people to seek refuge elsewhere.

I have always had a steadfast conviction that migration needs to be managed.

It needs endurance and patient work with key partners.

And it needs unity within our Union.

This is the spirit of the New Pact on Migration and Asylum.

When we took office, there seemed to be no possible compromise in sight.

But with the Pact, we are striking a new balance.

Between protecting borders and protecting people.

Between sovereignty and solidarity.

Between security and humanity.

We listened to all Member States and focused on all routes.

And we have translated the spirit of the Pact into practical solutions.

We were fast and united in responding to the hybrid attack that Belarus launched against us.

We worked closely with our Western Balkan partners and reduced irregular flows.

We have signed a partnership with Tunisia that brings mutual benefits beyond migration – from energy and education, to skills and security.

And we now want to work on similar agreements with other countries.

We stepped up border protection.

European Agencies deepened their cooperation with Member States.

Allow me to thank in particular Bulgaria and Romania for leading the way – showcasing best practices on both asylum and returns.

They have proved it: Bulgaria and Romania are part of our Schengen area.

So let us finally bring them in – without any further delay.

 

Ladies and Gentlemen,

Our work on migration is based on the conviction that unity is within our reach.

An agreement on the pact has never been so close.

Parliament and the Council have a historic opportunity to get it over the line.

Let us show that Europe can manage migration effectively and with compassion.

Let's get this done!

 

Honourable members,

We know that migration requires constant work.

And nowhere is that more vital than in the fight against human smugglers.

They attract desperate people with their lies.

And put them on deadly routes across the desert, or on boats that are unfit for the sea.

The way these smugglers operate is continuously evolving.

But our legislation is over twenty years old and needs an urgent update.

So we need new legislation and a new governance structure.

We need stronger law enforcement, prosecution and a more prominent role for our agencies – Europol, Eurojust and Frontex.

And we need to work with our partners to tackle this global plague of human trafficking.

This is why the Commission will organise an International Conference on fighting people smuggling.

It is time to put an end to this callous and criminal business!

 

UKRAINE

Honourable Members,

On the day when Russian tanks crossed the border into Ukraine, a young Ukrainian mother set off for Prague to bring her child to safety.

When the Czech border official stamped her passport, she started crying.

Her son didn't understand. And he asked his mother why she was crying.

She answered: “Because we are home.”

“But this is not Ukraine,” he argued.

So she explained: “This is Europe.”

On that day, that Ukrainian mother, felt that Europe was her home.

Because “home is where we trust each other”.

And the people of Ukraine could trust their fellow Europeans.

Her name was Victoria Amelina.

She was one of the great young writers of her generation and a tireless activist for justice.

Once her son was safe, Victoria returned to Ukraine to document Russia's war crimes.

One year later she was killed by a Russian ballistic missile, while having dinner with colleagues.

The victim of a Russian war crime, one of countless attacks against innocent civilians.

Amelina was with three friends that day – including Héctor Abad Faciolince, a fellow writer from Colombia.

He is part of a campaign called “Aguanta, Ucrania” – “Resist, Ukraine”, created to tell Latin Americans of Russia's war of aggression and attacks on civilians.

But Héctor could never imagine becoming the target himself.

Afterwards, he said he didn't know why he lived and she died.

But now he is telling the world about Victoria. To save her memory and to end this war.

And I am honoured that Héctor is here with us today.

And I want you to know that we will keep the memory of Victoria – and all other victims – alive.

Aguanta, Ucrania. Slava Ukraini!

 

Honourable Members,

We will be at Ukraine's side every step of the way.

For as long as it takes.

Since the start of the war, four million Ukrainians have found refuge in our Union.

And I want to say to them that they are as welcome now as they were in those fateful first weeks.

We have ensured that they have access to housing, healthcare, the job market and much more.

 

Honourable Members,

this was Europe answering the call of history.

And so I am proud to announce that the Commission will propose to extend our temporary protection to Ukrainians in the EU. 

Our support to Ukraine will endure.

We have provided 12 billion euros this year alone to help pay wages and pensions.

To help keep hospitals, schools and other services running. 

And through our ASAP proposal we are ramping up ammunition production to help match Ukraine's immediate needs.

But we are also looking further ahead.

This is why we have proposed an additional 50 billion euros over four years for investment and reforms.

This will help build Ukraine's future to rebuild a modern and prosperous country.

And that future is clear to see.

This House has said it out loud: The future of Ukraine is in our Union.

The future of the Western Balkans is in our Union.

The future of Moldova is in our Union.

And I know just how important the EU perspective is for so many people in Georgia.

 

Honourable Members,

I started by speaking of Europe responding to the call of history. 

And history is now calling us to work on completing our Union.

In a world where some are trying to pick off countries one by one, we cannot afford to leave our fellow Europeans behind.

In a world where size and weight matters, it is clearly in Europe's strategic and security interests to complete our Union.

But beyond the politics and geopolitics of it, we need to picture what is at stake.

We need to set out a vision for a successful enlargement.

A Union complete with over 500 million people living in a free, democratic and prosperous Union.

A Union complete with young people who can live, study and work in freedom.

A Union complete with vibrant democracies in which judiciaries are independent, oppositions are respected, and journalists are protected.

Because the rule of law and fundamental rights will always be the foundation of our Union – in current and in future Member States.

This is why the Commission has made the Rule of Law Reports a key priority.

We now work closely with Member States to identify progress and concerns – and make recommendations for the year ahead.

This has ensured accountability in front of this House and national parliaments.

It has allowed for dialogue between Member States. 

And it is delivering results.

I believe that it can do the same for future Member States.

This is why I am very happy to announce that we will open the Rule of Law Reports to those accession countries who get up to speed even faster.

This will place them on an equal footing with Member States.

And support them in their reform efforts.

And it will help ensure that our future is a Union of freedom, rights and values for all.

 

Honourable Members,

This is in our shared interest.

Think about the great enlargement of 20 years ago.

We called it the European Day of Welcomes.

And it was a triumph of determination and hope over the burdens of the past.

And in the 20 years since we have seen an economic success story which has improved the lives of millions.

I want us to look forward to the next European Day of Welcomes and the next economic success stories.

We know this is not an easy road.

Accession is merit-based – and the Commission will always defend this principle.

It takes hard work and leadership.

But there is already a lot of progress.

We have seen the great strides Ukraine has already made since we granted them candidate status.

And we have seen the determination of other candidate countries to reform.

 

Honourable Members,

it is now time for us to match that determination.

And that means thinking about how we get ready for a completed Union.

We need to move past old, binary debates about enlargement.

This is not a question of deepening integration or widening the Union.

We can and we must do both.

To give us the geopolitical weight and the capacity to act. 

This is what our Union has always done.

Each wave of enlargement came with a political deepening.

We went from coal and steel towards full economic integration.

And after the fall of the Iron Curtain, we turned an economic project into a true Union of people and states.

I believe that the next enlargement must also be a catalyst for progress.

We have started to build a Health Union at 27.

And I believe we can finish it at 30+.

We have started to build European Defence Union at 27.

And I believe we can finish it at 30+.

We have proven that we can be a Geopolitical Union andshowed we can move fast when we are united.

And I believe that Team Europe also works at 30+.

 

Honourable Members,

I know this House believes the same.

And the European Parliament has always been one of the main drivers of European integration.

It has been so throughout the decades.

And it is once again today. 

And I will always support this House – and all of those who want to reform the EU to make it work better for citizens.

And, yes, that means including through a European Convention and Treaty change if and where it is needed!

But we cannot – and we should not – wait for Treaty change to move ahead with enlargement.

A Union fit for enlargement can be achieved faster.

That means answering practical questions about how a Union of over 30 countries will work in practice.

And in particular about our capacity to act.

The good news is that with every enlargement those who said it would make us less efficient were proven wrong.

Take the last few years.

We agreed on NextGenerationEU at 27.

We agreed to buy vaccines at 27.

We agreed on sanctions in record time – also at 27.

We agreed to purchase natural gas – not only at 27 but including Ukraine, Moldova and Serbia.

So it can be done.

But we need to look closer at each policy and see how they would be affected by a larger Union.

This is why the Commission will start working on a series of pre-enlargement policy reviews to see how each area may need to be adapted to a larger Union.

We will need to think about how our institutions would work – how the Parliament and the Commission would look.

We need to discuss the future of our budget – in terms of what it finances, how it finances it, and how it is financed.

And we need to understand how to ensure credible security commitments in a world where deterrence matters more than ever. 

These are questions we must address today if we want to be ready for tomorrow.

And the Commission will play its part.

This is why we will put forward our ideas to the Leaders' discussion under the Belgian Presidency. 

We will be driven by the belief that completing our Union is the best investment in peace, security and prosperity for our Continent.

So it is time for Europe to once again think big and write our own destiny!

 

CONCLUSION

Honourable Members,

Victoria Amelina believed that it is our collective duty to write a new story for Europe. 

This is where Europe stands today. 

At a time and place where history is written.

The future of our continent depends on the choices we make today. 

On the steps we take to complete our Union.

The people of Europe want a Union that stands up for them in a time of great power competition.

But also one that protects and stands close to them, as a partner and ally in their daily battles.

And we will listen to their voice.

If it matters to Europeans, it matters to Europe.

Think again about the vision and imagination of the young generation I started my speech with.

It is the moment to show them that we can build a continent where you can be who you are, love who you want, and aim as high as you want.

A continent reconciled with nature and leading the way on new technologies.

A continent that is united in freedom and peace. 

Once again – this is Europe's moment to answer the call of history. 

Long live Europe.  
 \newpage
\end{document}
